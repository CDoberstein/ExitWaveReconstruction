%%%%%%%%%%%%%%%%%%%%%%%%%%%%%%%%%%%%%%%%%%%%%%%%%%%%%%%%%%%%%%%%%%%%%%%%%%%%%%%%%%
%%%%%%%%%%%%%%%%%%%%%%%%%%%%%%%%%%%%%%%%%%%%%%%%%%%%%%%%%%%%%%%%%%%%%%%%%%%%%%%%%%
%%%%%%%%%%%%%%%%%%%%%%%%%%%%%%%%%%%%%%%%%%%%%%%%%%%%%%%%%%%%%%%%%%%%%%%%%%%%%%%%%%
%%%%%%%%%%%%%%%%%%%%%%%%%%%%%%%%%%%%%%%%%%%%%%%%%%%%%%%%%%%%%%%%%%%%%%%%%%%%%%%%%%


%                 Chapter The FEOPInterface
%                 written by Marc Droske


%%%%%%%%%%%%%%%%%%%%%%%%%%%%%%%%%%%%%%%%%%%%%%%%%%%%%%%%%%%%%%%%%%%%%%%%%%%%%%%%%%
%%%%%%%%%%%%%%%%%%%%%%%%%%%%%%%%%%%%%%%%%%%%%%%%%%%%%%%%%%%%%%%%%%%%%%%%%%%%%%%%%%
%%%%%%%%%%%%%%%%%%%%%%%%%%%%%%%%%%%%%%%%%%%%%%%%%%%%%%%%%%%%%%%%%%%%%%%%%%%%%%%%%%
%%%%%%%%%%%%%%%%%%%%%%%%%%%%%%%%%%%%%%%%%%%%%%%%%%%%%%%%%%%%%%%%%%%%%%%%%%%%%%%%%%

\chapter{The FEOPInterface}

\newcommand{\dx}{\,\mathrm{d}x}
\newcommand{\V}{{\mathcal V}}

\section{Finite-Element-Operators}

Example: Poisson equation
\begin{eqnarray*}
 u - \Delta u & = & f  \quad \mbox{ in } \Omega \\
\partial_\nu u &=& 0 \quad \mbox{ on } \partial \Omega
\end{eqnarray*}

weak formulation (integration, integration by parts)
\begin{equation}
\int_\Omega  u \varphi \dx + \int_\Omega \nabla u \nabla \varphi \dx = \int_{\partial \Omega } \underbrace{\nabla u \cdot \nu}_{=0} \varphi \dx + \int f \varphi \dx
\end{equation}
for all $\varphi\in H^{1,2}(\Omega)$.
\begin{equation}
 a(u,\varphi) = F( \varphi )
\end{equation}

a continuous formulation (elliptic):

bounded continuous bilinear form $a:\V\times\V:\to \R$
\begin{itemize}
\item boundedness $a(u,v) \leq C \|u\|_{\V}\|v\|_{\V}$
\item coercivity $a(u,u) \geq c \|u\|_{\V}^2$, $c>0$.
\end{itemize}

$F\in \V'$ bounded linearform.

existence and uniqueness by Lax-Milgram.

\section{Discretization}

replace $\V$ by a finite dimensional subspace $\V_h \subset \V$, with basis $(\varphi_i)_i$.
\begin{equation}
u_h = \sum_j \bar U_j \varphi_j
\end{equation}

\begin{eqnarray*}
\int_\Omega \sum_j \bar U_j \varphi_j \varphi_i \dx + \int_\Omega\sum_j \bar U_j \nabla \varphi_j \cdot \nabla \varphi_i \dx  & = & F(\varphi_i)  \\
 \sum_j \bar U_j \underbrace{\int_\Omega \varphi_j \varphi_i \dx}_{=: \mathbf{M}_{ij}} + \sum_j \bar U_j \underbrace{\int_\Omega\nabla \varphi_j \cdot \nabla \varphi_i \dx}_{=:\mathbf{L}_{ij}}  & = & F(\varphi_i) \\
(\mathbf{M} + \alpha \mathbf{L}) \bar U &=& \bar F
\end{eqnarray*}

How to construct finite dimensional subspaces?

\begin{itemize}
\item Partition of $\Omega_h$ into a triangulation $\bar \Omega = \bigcup_{T\in \mathcal{T}} \bar T $
\item local basis on cells $\bar T$ (Lagrange-basis, Hermite-basis, etc. )
\end{itemize}


%\input{ref_cell.pstex_t}

Assembly of $\mathbf{L}$: Traverse all elements, compute for all
$i, j$, such that $\mathrm{supp} \varphi_i \cap \mathrm{supp} \varphi_j \cap T \neq \emptyset$, the integrals $\int_T \nabla \varphi_i \nabla \varphi_j \dx$.

For QuocMeshes $T(x) = h x + b$, hence
\begin{eqnarray*}
\int_T \nabla \hat \varphi_i\circ \phi^{-1} \nabla \hat \varphi_j\circ\phi^{-1}\dx & = &
\frac{|T|}{|\hat T|} \int_{\hat T} \nabla \hat \varphi_i \cdot \nabla  \hat \varphi_j h^{-2}\dx \\
& =  & h^{d-2} \int_{\hat T} \nabla \hat \varphi_i \cdot \nabla  \hat \varphi_j\dx
\end{eqnarray*}

\section{main ingredients}

\begin{itemize}
\item Definition of the discrete function space ($\leadsto$ \id{Configurators})
  \begin{itemize}
  \item iterator over cells $\leadsto$ \id{qc::GridDefinition::OldAllElementIterator}
  \item mapping of local indices to global indices $\leadsto$ \id{qc::FastILexMapper}
  \item quadrature rules on reference elements e.g. $\leadsto$ \id{aol::GaussQuadrature}
  \item definition of basefunction set $\leadsto$ \id{BaseFunctionSet}, the basefunctionset also does cached evaluation
for quadrature.
  \end{itemize}
\item Finite-Element operator related
  \begin{itemize}
    \item assembly of local matrices.
  \end{itemize}
\end{itemize}

{\small
Configurators define and provide:
\begin{myverbatim}
class MyConfiguratorForBilinear2DElements {
  typedef qc::Element                                ElementType;
  typedef qc::GridDefinition::OldAllElementIterator  ElementIteratorType;
  typedef RealType                                   Real;
  typedef qc::GridDefinition                         InitType;

  const ElementIteratorType &begin( ) const;
  inline const ElementIteratorType &end( ) const;
  RealType H( const qc::Element& ) const;

  typedef Vec2<_RealType>     VecType;
  typedef Matrix22<_RealType> MatType;

  typedef qc::FastUniformGridMatrix<_RealType,qc::QC_2D> MatrixType;
  typedef BaseFunctionSetMultiLin<...,_QuadType> BaseFuncSetType;

  static const int maxNumLocalDofs = 4;
  static const qc::Dimension Dim = qc::QC_2D;
  int getNumLocalDofs( const qc::Element & ) const;
  int getNumGlobalDofs( ) const;
  const BaseFuncSetType& getBaseFunctionSet( const qc::Element &El ) const;
  int localToGlobal( const qc::Element &El, const int localIndex ) const;
  MatrixType* createNewMatrix( );
};
\end{myverbatim}
}

In the quocmesh library exist default configurator classes by using (nested) traits:
\begin{myverbatim}
typedef
qc::QuocConfiguratorTraitMultiLin<
         REAL,         // RealType
         qc::QC_2D,    // Dimension
         aol::GaussQuadrature<REAL,qc::QC_2D,3> >
ConfType;
\end{myverbatim}
The last line specifies the type of Quadrature to be used, here
Gauss-Quadrature of order $3$.

with this at hand the setup of a stiffness or mass matrix is all that easy:
\begin{myverbatim}
 aol::StiffOp<ConfType> stiff( grid, MODE );
 aol::MassOp<ConfType>  mass( grid, MODE );
\end{myverbatim}
where \id{MODE} is either \id{aol::ASSEMBLED} or \id{aol::ONTHEFLY}.
On the fly operators don't need any memory but are slower.

\section{Support for lumped mass-matrices}

standard lumped mass matrix defined by
\begin{equation}
(\mathbf{M}_h)_{ij} := \int_\Omega \mathcal{I}_h(\varphi_i \varphi_j)) \dx
\end{equation}

\begin{myverbatim}
aol::LumpedMassOp<ConfType>
     lumpedMass( grid, INVERT_MODE );
\end{myverbatim}
\id{INVERT\_MODE} is either \id{aol::INVERT} or \id{aol::DO\_NOT\_INVERT}


Consider heat equation $(\mathbf{M} + \tau \mathbf{L})\bar U^{n+1} = \mathbf{M}U^{n}$.
\begin{myverbatim}
aol::StiffOp<ConfType> stiff( grid );
aol::MassOp<ConfType>  mass( grid );
aol::LinCombOp<aol::Vector<REAL> > op;
op.append( mass );
op.append( stiff, tau );
aol::CGInverse<aol::Vector<REAL> > inv( op );
mass.apply( u_old, rhs );
inv.apply( rhs, u_new );
\end{myverbatim}

That's all.. :-)


Alternative: FEOp's can assemble themselves into other matrices:
\begin{myverbatim}
aol::StiffOp<ConfType> stiff( grid );
aol::MassOp<ConfType>  mass( grid );
aol::SparseMatrix<REAL> mat( grid );
stiff.assembleAddMatrix( mat );
mat *= tau;
mass.assembleAddMatrix( mat );
aol::CGInverse<aol::Vector<REAL> > inv( mat );
mass.apply( u_old, rhs );
inv.apply( rhs, u_new );
\end{myverbatim}


There exist easy to use interfaces for operators of the form
\begin{itemize}
\item $\div( a(x) \nabla u )$ $a$ scalar, $\leadsto$ \id{FELinScalarWeightedStiffInterface}
\item $\div( A(x) \nabla u )$ $A$ a matrix, $\leadsto$ \id{FELinMatrixWeightedStiffInterface}
\item $a(x) u$, $\leadsto$ \id{FELinScalarWeightedMassInterface}
\end{itemize}

Generation of a discrete function given a vector of coefficients.
\begin{myverbatim}
aol::DiscreteFunction<ConfType> discFunc( grid, coeffs );
discFunc.evaluate( El, locCoords );     // slow
discFunc.evaluateAtQuadPoint( El, i );  // fast
discFunc.evaluateGradient( El, locCoords, grad );     // slow
discFunc.evaluateGradientAtQuadPoint( El, i, grad );  // fast
\end{myverbatim}

\section{Customization: example mean curvature flow}
\begin{multline*}
\partial_t u + \div\left\{ \frac {\nabla u}{\|\nabla u\|} \right\} \|\nabla u\| = 0 \quad \Rightarrow \quad
\left(\boldsymbol{M}+ \tau \boldsymbol{L}\right) \bar U^{n+1} = \boldsymbol{M} U^n \\
\mbox{ where } \quad \boldsymbol{M}_{ij} = \int_\Omega \frac {\varphi_i \varphi_j}{\|\nabla U^{n} \|_\epsilon } \dx \quad \mbox{ and } \quad \boldsymbol{L}_{ij} = \int_\Omega \frac {\nabla \varphi_i \cdot \nabla \varphi_j}{\|\nabla U^{n} \|_\epsilon } \dx
\end{multline*}

How to implement the matrices with the \id{FEOpInterface}-classes?

\begin{itemize}
\item $\boldsymbol{M}$ $\leadsto$ derive from \id{FELinScalarWeightedMassInterface}
\item $\boldsymbol{L}$ $\leadsto$ derive from \id{FELinScalarWeightedStiffInterface}
\end{itemize}

{\small
\begin{myverbatim}
template <typename Conf_T>
class MCMStiffOp :
public aol::FELinScalarWeightedStiffInterface<Conf_T, MCMStiffOp<Conf_T> > {
public:
  typedef typename Conf_T::RealType RealType;
protected:
  aol::DiscreteFunctionDefault<Conf_T> *_discFunc;
  RealType _eps;
public:
  MCMStiffOp( const typename Conf_T::InitType &Initializer,
              aol::OperatorType OpType = aol::ONTHEFLY,
              RealType Epsilon = 1. )
: aol::FELinScalarWeightedStiffInterface<Conf_T, MCMStiffOp<Conf_T> >( Initializer, OpType ),
      _discFunc( NULL ), _eps( Epsilon )  {
  }

  void setImage( const aol::Vector<RealType> &Image ) { [...] }

  inline RealType getCoeff( const qc::Element &El, int QuadPoint,
                            const typename Conf_T::VecType& RefCoord ) const {
    if ( !_discFunc ) {  throw aol::Exception( "first!", __FILE__, __LINE__ );  }
    typename Conf_T::VecType grad;
    _discFunc.evaluateGradientAtQuadPoint( El, QuadPoint, grad );
    return 1. / sqrt( grad.normSqr() + _eps*_eps );
  }
};
\end{myverbatim}
}


\section{Advantages}
\begin{enumerate}
\item minimization of code duplication $\leadsto$ concentrate on your problem.
\item effiency due to inlined \id{getCoeff} function
\item works automatically in 2D as well as 3D!
\item works for arbitrary finite element spaces
\item works for arbitrary quadrature rules
\item works for arbitrary grids.
\end{enumerate}
\section{this class (and others) are already implemented in \id{mcm.h}}



\section{Other classes---for right hand sides:}
\begin{itemize}
\item for \id{Vector}'s
  \begin{itemize}
  \item \id{FENonlinOpInterface} $\leadsto$ $\int_\Omega f(U)\varphi_i\dx$
  \item \id{FENonlinDiffOpInterface} $\leadsto$ $\int_\Omega \vec f(U)\cdot \nabla \varphi_i\dx$
  \end{itemize}
\item for \id{MultiVector}'s
  \begin{itemize}
  \item \id{FENonlinVectorOpInterface} $\leadsto$ $\int_\Omega \vec f(U)\varphi_i\dx$
  \item \id{FENonlinVectorDiffOpInterface} $\leadsto$ $\int_\Omega \boldsymbol{F}(U)\cdot \nabla \varphi_i\dx$
  \end{itemize}
\end{itemize}


\section{Overview over Integral-Interface-classes}
Here, an overview over the finite element interface classes is given.
We will distinguish between linear operator interfaces (which additionally
to "apply(...)" and "applyAdd(...)" have a method "assembleAddMatrix(...)" to assemble a system matrix)
and nonlinear operators (which cannot assemble a matrix).
The first table enlists the linear operator interfaces, where the second column displays the matrix assembled in "assembleAddMatrix(...)"
(the result of "apply(...)" then is just this matrix multiplied by the vector of function values, passed to "apply(...)" as the first argument),
the left column gives the corresponding class names, and the right column enlists the methods to be overloaded.
The second table shows all nonlinear operator interfaces and has the same structure, only the middle column shows the result of the method "apply(...)".
In both cases, $\phi$ shall represent the discretized function, which is passed to "apply(...)" or "applyAdd(...)" as the first argument,
$\varphi_i$ represents the $i$th finite element base function, $w$, $f$, $A$ are functions, which have to be implemented by the overload method
in the derived class. An arrow over a function signifies that the function is vector-valued, two arrows
symbolize a matrix. The implementation of $f\left(\phi(x),\nabla\phi(x),x\right)$ gets $x$ and the function $\phi$ as argument, where $\phi$
itself may be evaluated at $x$, or its gradient, or both. $n_g$ shall denote the number of components of a vector function $\vec g$.

{
\renewcommand{\arraystretch}{2}
\resizebox{\textwidth}{!}{
\begin{tabular}{llll}
  \hline
    \textbf{FELin...Interface} & \textbf{expression} & \textbf{overload method} & \textbf{comment} \\
  \hline
    ScalarWeightedMass      & $\left(\int_\Omega w(x) \varphi_j(x) \varphi_i(x) dx\right)_{ij}$                                                            & $w$: getCoeff() \\
    ScalarWeightedStiff     & $\left(\int_\Omega w(x) \nabla\varphi_j(x) \cdot \nabla\varphi_i(x)dx \right)_{ij}$                                          & $w$: getCoeff() \\
    MatrixWeightedStiff     & $\left(\int_\Omega A(x)\nabla \varphi_j \cdot \nabla \varphi_i dx\right)_{ij}$                                               & $A$: getCoeffMatrix() & $A$ symmetric \\
    AsymMatrixWeightedStiff & $\left(\int_\Omega A(x)\nabla \varphi_j \cdot \nabla \varphi_i dx\right)_{ij}$                                               & $A$: getCoeffMatrix() & $A$ asymmetric \\
    ScalarWeightedMixedDiff & $\left(\int_\Omega w(x) \frac{\partial\varphi_j(x)}{\partial x_s} \frac{\partial\varphi_i(x)}{\partial x_t} dx\right)_{ij}$  & $w$: getCoeff() \\
    ScalarWeightedSemiDiff  & $\left(\int_\Omega \varphi_j(x) \frac{\partial\varphi_i(x)}{\partial x_k} w(x)\,dx\right)_{ij} $                             & $w$: getCoeff() & can also return the transpose \\
    VectorWeightedSemiDiff  & $\left(\int_\Omega \varphi_j(x) (\nabla \varphi_i(x) \cdot \vec{w}(x))\,dx\right)_{ij} $                                     & $\vec{w}$: getCoefficientVector() & can also return the transpose, \\
                            &                                                                                                                              & & $n_w=$domain dimension \\
  \hline
\end{tabular}
}
}

\vspace*{1cm}
{
\renewcommand{\arraystretch}{2}
\resizebox{\textwidth}{!}{
\begin{tabular}{llll}
  \hline
    \textbf{FENonlin...Interface} & \textbf{expression} & \textbf{overload method} & \textbf{comment} \\
  \hline
    Op                       & $\left(\int_\Omega f\left(\phi(x),\nabla\phi(x),x\right) \varphi_i(x) dx\right)_i $                                           & $f$: getNonlinearity() \\
    DiffOp                   & $\left(\int_\Omega \vec{f}\left(\phi(x),\nabla\phi(x),x\right)\cdot \nabla\varphi_i(x) dx\right)_i $                          & $\vec{f}$: getNonlinearity() \\
    VectorOp                 & $\left(\int_\Omega \vec{f}\left(\vec\phi(x),\nabla\vec\phi(x),x\right)\cdot\vec\varphi_i(x) dx\right)_i $                     & $\vec{f}$: getNonlinearity() & $n_f$, $n_\phi$ are template parameters \\
    VectorDiffOp             & $\left(\!\!\int_\Omega \!\!\vec{\vec{f}}\left(\vec\phi(x),\nabla\vec\phi(x),x\right):\nabla \vec\varphi_i(x) dx\!\!\right)_i$ & $\vec{\vec{f}}$:getNonlinearity() \\
    IntegrationScalar        & $\int_\Omega f\left(\phi(x),\nabla\phi(x),x\right) dx$                                                                        & $f$: evaluateIntegrand() \\
    IntegrationVector        & $\int_\Omega f\left(\vec\phi(x),\nabla\vec\phi(x),x\right) dx$                                                                & $f$: evaluateIntegrand() & $n_\phi$ is template parameter \\
    IntegrationVectorGeneral & $\int_\Omega f\left(\vec\phi(x),\nabla\vec\phi(x),x\right) dx$                                                                & $f$: evaluateIntegrand() & $n_\phi\leq$domain dimension is variable \\
  \hline
\end{tabular}
}
}
VectorFENonlinIntegrationVectorInterface can be used for vector valued integrands.



%%% Local Variables:
%%% mode: latex
%%% TeX-master: "manual"
%%% End:
