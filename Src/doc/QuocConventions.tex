% *******************************************************************************
%     File of conventions for programming in the quocmesh-library.
% *******************************************************************************

The following rules apply to all modules, examples and tools. You should observe them in projects, too.

\paragraph{class- and filenames}

\begin{itemize}
  \item class names: use \verb|CamelCase|, starting with upper case letter
  \item member function: use \verb|camelCase|, starting with lower case letter
  \item file name: use \verb|camelCase|, starting with lower case letter (exemption: start with upper case,
        if first word is a proper name, e. g. \verb|ArmijoSearch.h|), \\
        use only the following characters:
        \begin{itemize}
          \item upper and lower case letters (in particular no umlauts) (\texttt{a..z, A..Z})
          \item digits (\texttt{0..9})
          \item underscore and hyphen (\texttt{\_,-})
          \item periods/full stops (\texttt{.})
        \end{itemize}
  \item please use correct English names
  \item avoid name conflicts with system header files (e. g. stl headers)
  \item include each new module header file to corresponding selfTest
\end{itemize}


\paragraph{templates}

\begin{itemize}
  \item template parameters: must contain lowercase letters (\verb|realType| and \verb|RealType|
        are okay, but \verb|REAL| is not)
  \item Naming standard for re-exported template parameters: \\
        \verb|template< typename _DataType > ... typedef _DataType DataType;|
  \item For template parameters to be reexported: use \verb|_SomeType| as template parameter and
        \verb|public typedef _SomeType SomeType|
  \item If declaration and implementation are separate, the template parameters must have the
        same name in both cases, that is \verb|_SomeType|. In the implementation, both versions
        may be used.
\end{itemize}


\paragraph{preprocessor directives}

\begin{itemize}
    \item after \verb|#define|, USE CAPITAL LETTERS
          \begin{verbatim}
#ifndef __BLA_H
#define __BLA_H
// contents of bla.h
#endif         \end{verbatim}
    \item \#ifdef and similar preprocessor directives are not indented.
    \item include guards must be used in all headers, the format is \verb|__AOL_H|
    \item openmp critical sections should be named according to the scheme
\begin{verbatim}
namespace_class_{method,other useful identifier}[_number, if needed]
\end{verbatim}
      e.g.
\begin{verbatim}
#pragma omp critical (aol_RandomGenerator_getNextRnd)
\end{verbatim}
\end{itemize}


\paragraph{structure of externals}

\begin{itemize}
\item standard externals should contain
  \begin{itemize}
  \item \texttt{makefile.local} that sets include and link paths
  \item provide an include header, marked as a system header
\begin{verbatim}
#ifdef __GNUC__
#pragma GCC system_header
#endif
\end{verbatim}
    to prevent compiler warnings for external code
  \item a short description
  \end{itemize}
\item if the external is selected in
  \texttt{makefile.selection.default}, define
  \texttt{USE$\_$EXTERNAL$\_$...} is set automatically.
\item nonstandard externals may contain (small amounts of) code that
  is compiled automatically (if necessary) by the make mechanism
  (\texttt{go} and \texttt{clean} scripts) or by an appropriate visual
  c++ project
\end{itemize}

\paragraph{using externals}
\begin{itemize}
\item all modules (except for those obviously fully depending on an
  external) must compile without the external being used
\item Programs that use external code also have to compile if the
  corresponding external is switched off. The executables should then
  give a useful and informative error message like ''This program
  can't be used without (\textit{corresponding external})''. To achieve
  this, enclose your header and cpp-files in the following
  ifdef-construction (external is GRAPE in this example):
\begin{verbatim}
#ifdef USE_EXTERNAL_GRAPE
    ... (code that uses externals) ...
#endif
\end{verbatim}
In case of executables add the follwing else-part (or a similar one):
\begin{verbatim}
#ifdef USE_EXTERNAL_GRAPE
  ... (code that uses externals) ...
#else
  int main ( int, char** ) {
    cerr << "Without grape external, this program is useless" << endl;
    return ( 0 ) ;
  }
#endif
\end{verbatim}
\end{itemize}


\paragraph{style (indentation, spaces~\ldots)}

\begin{itemize}
    \item 2 spaces are used for indentation (no tabs, not 4 spaces etc.)
    \item preprocessor directives are not indented at all.
    \item \texttt{public:} and similar are not indented relative to the class.
          In both cases, that is the current astyle standard.
    \item brackets (placement in lines and spacing around brackets) are used according to the following scheme:
          \begin{verbatim}
dummy_method ( aol::Vector<RealType> &vec, RealType factor ) {
  for ( int i = 0; i < vec.size(); ++i ) {
    vec[i] = factor * vec[i];
  }
}              \end{verbatim}
          and can be enforced automatically by using util/indent which in turn uses astyle
\end{itemize}


\paragraph{Name convention for methods that import or export data, e. g. aol::Mat A, B inversion (same for transposition etc.)}

\begin{itemize}
  \item \texttt{void A.invert()}: writes $A = A^{-1}$
  \item \texttt{B = A.inverse()}: compute and return inverse of $A$, do not modify $A$
  \item \texttt{A.invertFrom(B)}: $A = B^{-1}$, $B$ unmodified
  \item \texttt{A.invertTo(B)}: $B = A^{-1}$, $A$ unmodified
\end{itemize}


\paragraph{miscellaneous}

\begin{itemize}
  \item comments have to be written in english (except in your own projects, there you can do whatever you want)
  \item use special characters only in your own projects and only on your own risk
\end{itemize}


\paragraph{data sets}

\begin{itemize}
  \item don't commit any data sets (images etc.) except very small data sets for examples or selfTests
        (keep those in directory \texttt{examples/testdata}, files here must be sufficiently free to be usable under quoc license)
\end{itemize}



\paragraph{No convention on~\ldots}

\begin{itemize}
  \item the position of member variables, they may be at the beginning or at the end of a class
  \item No general rule on whether implementation should be inside or outside class definition.
\end{itemize}


\paragraph{Rules for subversion}

\begin{itemize}
  \item Moving code and changing code (e. g. moving implementation out of class and changing it)
        should be committed separately to allow diffing.
  \item use svn:ignore to ignore files that will automatically occur when compiling and typical temp files of editors and IDEs,
        not for personal temporary copies like aol.hold
\end{itemize}


\paragraph{Very special things}
\begin{itemize}
  \item use \texttt{aol::Abs} instead of \texttt{fabs}.
        But: template specialization is necessary for unsigned data types when needed.
  \item Instead of \texttt{M$\_$PI} the expressions \texttt{aol::PI} or
        \texttt{aol::NumberTrait<RealType>::pi} should be used. Analogously for other mathematical
        constants (if not available define own NumberTrait).
  \item \texttt{apply(x, x)} is nowhere forbidden, but produces (mostly) unpredictable output.
        Apply should check for this and throw exception or contain comment that \texttt{apply(x, x)}
        works, we will not change this in all old apply methods now.
  \item Use \texttt{for}-loops where possible, even simple things like \\
        \texttt{a[0] = expression ( 0 ); a[1] = expression ( 1 )} \\
        should be done in a \texttt{for}-loop.
  \item In methods like \texttt{getMinValue()} or \texttt{getMaxValue()} don't initialize the
        first value with $\pm$ infinity, but with \texttt{vector[0]} (otherwise, if the size of
        the vector is $0$, it might happen that $\pm$ infinity is returned). \\
        If it's not really really obvious that in no case anything can ever go wrong with the
        \texttt{[]}-operator of the vector, use \texttt{get} and \texttt{set} (then bounds-checking
        is applied in the debug-mode).
  \item use \texttt{NON\_PARALLEL\_STATIC} if \texttt{static} variables should not be static when using
        parallelization (due to conflicting write access), e.g.\ if \texttt{static} is only used for
        performance reasons. If they always need to be static, prevent parallel write access.
\end{itemize}


