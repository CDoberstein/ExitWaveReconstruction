

\chapter{Solver (Direct, Iterative)}
\section{Implemented solver}

%\vspace{-.7cm}
\begin{minipage}{\linewidth}
\begin{minipage}[t]{0.5\linewidth}
\underline{ direct solver: }
\begin{itemize}
\item LUInverse
\item (QRDecomposeGivernsBase)
\item (QRDecomposeHouseholderMatrix)
\end{itemize}
\underline{ iterative solver: }
\begin{itemize}
\item CGInverse
\item PCGInverse
\item BiCGInverse
\item PBiCGInverse
\item PBiCGStabInverse
\item GMRESInverse
\item GaussSeidelInverse
\item (JacobiSmoother)
\item (abstractMultigrid)
\end{itemize}
\end{minipage}
\begin{minipage}[t]{0.5\linewidth}
\underline{ preconditioner: }
\begin{itemize}
\item ILU0Preconditioner
\item SSORPreconditioner
\end{itemize}
\underline{ relicts (not for use): }
\begin{itemize}
\item EllipticSolver
\item ParabolicTimestep
\item ParabImplEuler
\item CGInverseProjection
\item ApproxCGInverse
\end{itemize}
\underline{ not implemented: }
\begin{itemize}
\item multigraph
\end{itemize}
\end{minipage}
\end{minipage}

%%%%%%%%%%%%%%%%%%%%%%%%%%%%%%%%%%%%%%%%%%%%%%%%%%%%%%%%%%%%%%%%%%%%%%%%%%%%%%%

\subsection { QRDecompose}
\subsubsection{ QRDecomposeGivensBase, QRDecomposeHouseholderMatrix: } in \texttt{aol/QRDecomposition.h}
\begin{itemize}
\item \textbf{no solver!}
\item QRDecomposeGivensBase: \\
protected method \texttt{eliminate( R, col, toEliminate, eliminateFrom )}:\\
eliminates (toEliminate) element in column (col) with (eliminateFrom) \\
not for use ( abstract function \texttt{transform} )
\item QRDecomposeHouseholderMatrix: derived from QRDecomposeGivensBase\\
for right upper triangle matrices with occupied secondary diagonal
\item using: \begin{quote}
\texttt{aol::QRDecomposeHouseholderMatrix< RealType > QRH; \\
QRH.transform( H, R, Q );} \end{quote}
with \texttt{H} H-matrix of GMRES-method
\item with matrices \texttt{FullMatrix< RealType > }
\begin{flushright} \texttt{H( row, col ), R( row, col ), Q( row, row ); } \end{flushright}
\end{itemize}


%%%%%%%%%%%%%%%%%%%%%%%%%%%%%%%%%%%%%%%%%%%%%%%%%%%%%%%%%%%%%%%%%%%%%%%%%%%%%%%

\subsection {Using solver}
we want to solve the system of linear equations $Ax = b$ with a properly solver

$//$ declaration and instantiation:\\
$//$ operator\\
\texttt{{\em solverclass}< {\em templateparameter} > A( {\em parameterlist} ); }\\[2ex]
$//$\texttt{b}: right hand side \\
$//$\texttt{x}: initial value $\rightarrow$ solution \\
\texttt{VectorType b( {\em parameterlist} ), x( {\em parameterlist} ); \\
initialise\underline{ }rhs( b ); \\ initialise\underline{ }start( x );}\\[2ex]
$//$ solve equation Ax=b\\
\texttt{A.apply( b, x );}

after iteration: \texttt{x} = solution of equation


%%%%%%%%%%%%%%%%%%%%%%%%%%%%%%%%%%%%%%%%%%%%%%%%%%%%%%%%%%%%%%%%%%%%%%%%%%%%%%%

\section {Direct solver for $Ax=b$}
\subsection{ LUInverse: }
\begin{itemize}
\item derived from \texttt{Matrix< RealType >} in \texttt{project$/$bemesh$/$luinverse.h}
\item uses methods \texttt{makeLU} and \texttt{solveLU} of \texttt{FullMatrix}
\item using: \begin{quote}
\texttt{aol::LUInverse< RealType > luSolver(  Matrix\underline{ }A  ); \\
luSolver.apply( b, x );} \end{quote}
\item with matrix \texttt{aol::Matrix< RealType > Matrix\underline{ }A( row, col );}
\end{itemize}

%%%%%%%%%%%%%%%%%%%%%%%%%%%%%%%%%%%%%%%%%%%%%%%%%%%%%%%%%%%%%%%%%%%%%%%%%%%%%%%

\section {Iterative solver for $Ax = b$}
\label{sec:iterativeSolvers}
\subsection{ CGInverse: }
\begin{itemize}
\item derived from \texttt{InverseOp }in \texttt{aol::Solver.h}
\item Conjugate Gradient method, most effective for spd matrices (symmetric, positive definit)
\item using: \begin{quote}
\texttt{aol::CGInverse< VecType, OpType = Op< VecType > >}
\begin{flushright} \texttt{cgSolver( Operator\underline{ }A, Epsilon, MaxIter );} \end{flushright}
\texttt{cgSolver.apply( b, x ); } \end{quote}
\item default values: \\
\texttt{Epsilon = 1.e-16, \\ MaxIter = 1000}
\end{itemize}

%%%%%%%%%%%%%%%%%%%%%%%%%%%%%%%%%%%%%%%%%%%%%%%%%%%%%%%%%%%%%%%%%%%%%%%%%%%%%%%

\subsection{ PCGInverse: }
\begin{itemize}
\item derived from \texttt{InverseOp }in \texttt{aol::Solver.h}
\item CG method with preconditioning
\item using: \begin{quote}
\texttt{aol::PCGInverse< VecType,}
$\begin{array}[t]{l}
\texttt{OpType = Op< VecType >,} \\ \texttt{iOpType = Op< VecType > > }
\end{array}$
\begin{flushright} \texttt{pcgSolver( Operator\underline{ }A, approxInverseOperator\underline{ }IA, Epsilon, MaxIter ); }
\end{flushright}
\texttt{pcgSolver.apply( b, x );} \end{quote}
\item default values: \\
\texttt{Epsilon = 1.e-16, \\ MaxIter = 50}
\end{itemize}

%%%%%%%%%%%%%%%%%%%%%%%%%%%%%%%%%%%%%%%%%%%%%%%%%%%%%%%%%%%%%%%%%%%%%%%%%%%%%%%

\subsection{ BiCGInverse:}
\begin{itemize}
\item derived from \texttt{InverseOp }in \texttt{aol::Solver.h}
\item BIorthogonal CG, for non symmetric, non singular matrices
\item using: \begin{quote}
\texttt{aol::BiCGInverse< VecType > }
\begin{flushright} \texttt{bicgSolver( Operator\underline{ }A, transposeOperator\underline{ }AT, Epsilon, MaxIter );} \end{flushright}
\texttt{bicgSolver.apply( b, x); } \end{quote}
\item default values: \\
\texttt{Epsilon = 1.e-16, \\ MaxIter = 50}
\end{itemize}

%%%%%%%%%%%%%%%%%%%%%%%%%%%%%%%%%%%%%%%%%%%%%%%%%%%%%%%%%%%%%%%%%%%%%%%%%%%%%%%

\subsection{ PBiCGSolver:}
\begin{itemize}
\item derived from \texttt{InverseOp }in \texttt{aol::Solver.h}
\item BiCG with preconditioning
\item using: \begin{quote}
\texttt{aol::PBiCGInverse< VecType > }
\begin{flushright} \texttt{pbicgSolver( Operator\underline{ }A, transposeOperator\underline{ }AT, approxInverseOperator\underline{ }IA, Epsilon, MaxIter );}
\end{flushright}
\texttt{pbicgSolver.apply( b, x );}\end{quote}
\item default values: \\
\texttt{Epsilon = 1.e-16, \\ MaxIter = 50}
\end{itemize}

%%%%%%%%%%%%%%%%%%%%%%%%%%%%%%%%%%%%%%%%%%%%%%%%%%%%%%%%%%%%%%%%%%%%%%%%%%%%%%%


\subsection{ PBiCGStabInverse: }
\begin{itemize}
\item derived from \texttt{InverseOp }in \texttt{aol::Solver.h}
\item variation of BiCG, with preconditioning $\Rightarrow$ more stable
\item using: \begin{quote}
\texttt{aol::PBiCGStabInverse< VecType > \\
pbicgstabSolver( Operator\underline{ }A, approxInverseOperator\underline{ }IA, Epsilon, MaxIter ); \\
pbicgstabSolver.apply( b, x ); }
\end{quote}
\item default values: \\
\texttt{Epsilon = 1.e-16, \\ MaxIter = 50}
\end{itemize}

%%%%%%%%%%%%%%%%%%%%%%%%%%%%%%%%%%%%%%%%%%%%%%%%%%%%%%%%%%%%%%%%%%%%%%%%%%%%%%%


\subsection{ GMRESInverse: }
\begin{itemize}
\item derived from \texttt{InverseOp }in \texttt{aol::Solver.h}
\item  Generalized Minimum RESidual method, for non symmetric, non singular matrices\\
can become unstable
\item using: \begin{quote}
\texttt{aol::GMRESInverse< VecType > }
\begin{flushright} \texttt{ gmresSolver( Operator\underline{ }A, Epsilon, MaxInnerIter, MaxIter );} \end{flushright}
\texttt{gmresSolver.apply( b, x );}
\end{quote}
\item default values: \\
\texttt{Epsilon = 1.e-16, \\ MaxInnerIter = 10, \\ MaxIter = 50}
\end{itemize}

%%%%%%%%%%%%%%%%%%%%%%%%%%%%%%%%%%%%%%%%%%%%%%%%%%%%%%%%%%%%%%%%%%%%%%%%%%%%%%%


\subsection{ GaussSeidelInverse: }
\begin{itemize}
\item derived from \texttt{InverseOp } in \texttt{aol::sparseSolver.h}
\item using: \begin{quote}
\texttt{GaussSeidelInverse< VecType, MatType > }
\begin{flushright} \texttt{gaussseidelSolver( Matrix\underline{ }A, Epsilon, MaxIter, Relaxation  );}
\end{flushright}
\texttt{gaussseidelSolver.apply( b, x );}
\end{quote}
\item \textbf{for matrices only!}
\item default values: \\
\texttt{Epsilon = 1.e-16, \\ MaxIter = 50, \\ Relaxation = 1.0}
\item \texttt{Relaxation $\not=$ 1 $\Rightarrow$ } SOR
\item \texttt{MatType} must support \texttt{makeRowEntries }
\end{itemize}

%%%%%%%%%%%%%%%%%%%%%%%%%%%%%%%%%%%%%%%%%%%%%%%%%%%%%%%%%%%%%%%%%%%%%%%%%%%%%%%

\subsection {comments}

solver derived from \texttt{InverseOp: }
\begin{itemize}
\item no guarantee for convergence with nonsymmetric operators
\item solver for \texttt{Vector< RealType>} AND \texttt{MultiVector< RealType> }
\item classes derived from \texttt{aol::Op< VecType >}:
\begin{itemize}
\item \texttt{aol::Matrix }
\item \texttt{aol::FEOp, aol::FEOpInterface }
\item \texttt{aol::LinCombOp, aol::CompositeOp, ... }
\end{itemize}
\item quiet-mode: default = false \\
change with method \texttt{setQuietMode( bool )} for CGInverse, PCGInverse
\item change iteration steps with method \texttt{changeMaxIterations} for CGInverse, PCGInverse
\end{itemize}


%%%%%%%%%%%%%%%%%%%%%%%%%%%%%%%%%%%%%%%%%%%%%%%%%%%%%%%%%%%%%%%%%%%%%%%%%%%%%%%

\subsection{JacobiSmoother: }
\begin{itemize}
\item derived from \texttt{Op< Vector< Realtype > > } in \texttt{aol::sparseSolver.h}
\item used for multigrid (pre- and postsmoothing)
\item \textbf{no abort condition on Epsilon! always computes \texttt{MaxIter} iteration steps}
\item using: \begin{quote}\texttt{JacobiSmoother< RealType, MatType > }
\begin{flushright} \texttt{jacobiSolver( Matrix\underline{ }A, Epsilon, MaxIter, Relaxation );} \end{flushright}
\texttt{jacobiSolver.apply( b, x );}
\end{quote}
\item default values: \\
\texttt{Epsilon = 1.e-16, \\ MaxIter = 50, \\ Relaxation = 1.0}
\item \texttt{Relaxation $\not=$ 1 $\Rightarrow$ } Richardson
\item \texttt{MatType} must support \texttt{makeRowEntries }
\item quiet-mode: default = true
\item for \texttt{Vector< RealType >} only
\end{itemize}


%%%%%%%%%%%%%%%%%%%%%%%%%%%%%%%%%%%%%%%%%%%%%%%%%%%%%%%%%%%%%%%%%%%%%%%%%%%%%%%

\chapter {Preconditioner}
\section{SSORPreconditioner: }
\begin{itemize}
\item derived from \texttt{Op< Vector< RealType > >} in \texttt{aol::sparseSolver.h}
\item for positive definite matrices
\item one step of ssor-iteration
\item using: \begin{quote}
\item \texttt{aol::SSORPreconditioner< RealType, MatType > ssorSolver( Matrix\underline{ }A, Omega );\\
ssorSolver.apply( b, x );}
\end{quote}
\item \texttt{MatType} must support \texttt{makeRowEntries }
\item default value \texttt{Omega = 1.2}
\end{itemize}

%%%%%%%%%%%%%%%%%%%%%%%%%%%%%%%%%%%%%%%%%%%%%%%%%%%%%%%%%%%%%%%%%%%%%%%%%%%%%%%


\section{ILU0Preconditioner: }
\begin{itemize}
\item derived from \texttt{Op< Vector< RealType > >} in \texttt{aol::sparseSolver.h}
\item Incomplete LU decomposition, without modification of non diagonal elements
\item using: \begin{quote}
\texttt{ILU0Preconditioner< RealType, MatType > ilu0Solver( MatrixA ); \\
ilu0Solver.apply( b, x );} \end{quote}
\item \texttt{MatType} must support \texttt{makeRowEntries} and \texttt{makeSortedRowEntries}
\end{itemize}

%%%%%%%%%%%%%%%%%%%%%%%%%%%%%%%%%%%%%%%%%%%%%%%%%%%%%%%%%%%%%%%%%%%%%%%%%%%%%%%

\section {Sparse matrices: }
\texttt{MatType} must support \texttt{makeRowEntries} and \texttt{makeSortedRowEntries} (for \texttt{ILU0Preconditioner})\\
$\Rightarrow$ use one from the following matrices ( or derived )
\begin{itemize}
\item in \texttt{sparseMartrices.h}:
\begin{itemize}
\item \texttt{aol::GenSparseMatrix< RealType > A( row, column);} or \\ \texttt{aol::GenSparseMatrix< RealType > A( Grid );}\\
public methods \texttt{A.set( i, j, value);} and \texttt{A.set( i, j, value);}
\item \texttt{aol::SparseMatrix< RealType >} derived from \texttt{aol::GenSparseMatrix}
\item \texttt{aol::UniformGridSparseMatrix< RealType >} derived from \texttt{aol::GenSparseMatrix}
\end{itemize}
\item in \texttt{qc::fastUniformGridMatrix.h}:
\begin{itemize}
\item \texttt{qc::FastUniformGridMatrix< RealType, dimension > A( Grid );} \\
public methods \texttt{A.set( i, j, value);} and \texttt{A.set( i, j, value);}\\
\textbf{NO} method \texttt{makeSortedRowEntries} $\Rightarrow$ \textbf{NOT} for \texttt{ILU0Preconditioner}!
\end{itemize}
\end{itemize}

%%%%%%%%%%%%%%%%%%%%%%%%%%%%%%%%%%%%%%%%%%%%%%%%%%%%%%%%%%%%%%%%%%%%%%%%%%%%%%%

\subsection{ example: heat equation }

\[ \partial_t u - \kappa \triangle u = f \]

weak formulation and time discretization (implicit euler):
\begin{eqnarray*} \int \frac{ \left(u^{i+1} - u^i\right) \phi}{\tau} & = & - \int \kappa \nabla u^{i+1} \nabla \phi + \int f \phi \\
\Rightarrow (M + \tau \kappa L) U^{i+1} & = & M U^i + \tau M F
\end{eqnarray*}

for one timestep we have to solve \qquad $ Ax = b $ \\ [1cm]
with $x = U^{i+1}, A = M + \tau \kappa L$ and $ b = M U^i + \tau M F$

for simplicity: $\kappa = 1, f \equiv 0 \quad \Rightarrow \quad (M + \tau L)U^{i+1} = M U^i$


%%%%%%%%%%%%%%%%%%%%%%%%%%%%%%%%%%%%%%%%%%%%%%%%%%%%%%%%%%%%%%%%%%%%%%%%%%%%%%%

%%%%%%%%%%%%%%%%%%%%%%%%%%%%%%%%%%%%%%%%%%%%%%%%%%%%%%%%%%%%%%%%%%%%%%%%%%%%%%%


%%% Local Variables:
%%% mode: latex
%%% TeX-master: "manual"
%%% End:
