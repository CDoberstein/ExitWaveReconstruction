% General properties
\documentclass [twoside, a4paper, 12pt] {scrbook}
\usepackage {graphicx}

\usepackage {amssymb, latexsym}
\usepackage {amsmath}
\usepackage {color}
\usepackage {dsfont}
\usepackage [english] {babel}
\usepackage {times}

\usepackage {listings}
\lstset {language=C++, tabsize=2,
         keywordstyle=\color{rot}\bfseries,
         basicstyle=\ttfamily\small,
         commentstyle=\rmfamily\itshape,
         aboveskip=1cm,
         numbers=left, numberstyle=\tiny, stepnumber=1, numbersep=0.5cm,
         emph={Segment, Particle, Boundary, Vector, Vec2, aol, qc, bm,
               ReferenceSegment, ParaParticle, RectParticle, list, AllSegmentIterator, ConstAllSegmentIterator, PipingIterator,
               ParaSegment, ConstParaSegment, ParaSegmentIterator, ConstParaSegmentIterator,
               std, vector, RectSegment, ConstRectSegment, RectSegmentIterator, ConstRectSegmentIterator
               QRDecomposeHouseholderMatrix, QRInverse, Matrix, string, map, TypePtr,
               ParameterParser, istream, ostream, iterator, IntegralOperator, FullMatrix,
               SingleLayerPotential, LocalOperator, FEOpInterface, PanelCluster, Op,
               ParticleTree, PanelClusteringOperator, ConstLeafIterator, ConstSegmentIteratorType, ConstClusterIterator,
               File, Circle, Rectangle, Polygon, Complex, Date, FILE, String, Student, PhDStudent, Professor, Element,
               Point, Triangle, AlignedQuadrilateral, Square, DiscreteFunction, Grid, quoc,
               Exception, IOException, ParameterRangeException, FileFormatMismatch, StrItInterface,
               StringIterator, StringIteratorEndProxy, SurfMesh, istringstream},
         emphstyle=\color{blau}\bfseries,
         emph={[2]Implementation, IteratorType, ConstIteratorType, SegmentType, ConstSegmentType,
               DataType, ParticleType, IndexType, RangeType, DomainType,
               BoundaryType, ParticleIteratorType, SegmentIteratorType, _DataType, _IndexType, LocalOperatorType, _ParticleType},
         emphstyle={[2]\color{gruen}\bfseries}}

% Page layout
\typearea [12mm] {13}
\pagestyle {headings}
\frenchspacing
\setlength {\parindent} {0mm}

\definecolor {rot} {rgb} {0.4, 0, 0}
\definecolor {blau} {rgb} {0, 0, 0.4}
\definecolor {gruen} {rgb} {0, 0.4, 0}

\newcommand {\R} {\mathds {R}}
\newcommand {\id} [1] {{\lstinline$#1$}}
\renewcommand {\div} {\mathrm{div}}
\newcommand{\missing}[1]{\textbf{[missing:}{#1} \textbf{]}}

\lstnewenvironment {myverbatim} {} {}

\begin {document}

\selectlanguage {english}

\title {Using and Programming the QuocMesh Library}
\author {QuocMesh Collective}

\maketitle

\tableofcontents

\chapter{Source Code Conventions}
% *******************************************************************************
%     File of conventions for programming in the quocmesh-library.
% *******************************************************************************

The following rules apply to all modules, examples and tools. You should observe them in projects, too.

\paragraph{class- and filenames}

\begin{itemize}
  \item class names: use \verb|CamelCase|, starting with upper case letter
  \item member function: use \verb|camelCase|, starting with lower case letter
  \item file name: use \verb|camelCase|, starting with lower case letter (exemption: start with upper case,
        if first word is a proper name, e. g. \verb|ArmijoSearch.h|), \\
        use only the following characters:
        \begin{itemize}
          \item upper and lower case letters (in particular no umlauts) (\texttt{a..z, A..Z})
          \item digits (\texttt{0..9})
          \item underscore and hyphen (\texttt{\_,-})
          \item periods/full stops (\texttt{.})
        \end{itemize}
  \item please use correct English names
  \item avoid name conflicts with system header files (e. g. stl headers)
  \item include each new module header file to corresponding selfTest
\end{itemize}


\paragraph{templates}

\begin{itemize}
  \item template parameters: must contain lowercase letters (\verb|realType| and \verb|RealType|
        are okay, but \verb|REAL| is not)
  \item Naming standard for re-exported template parameters: \\
        \verb|template< typename _DataType > ... typedef _DataType DataType;|
  \item For template parameters to be reexported: use \verb|_SomeType| as template parameter and
        \verb|public typedef _SomeType SomeType|
  \item If declaration and implementation are separate, the template parameters must have the
        same name in both cases, that is \verb|_SomeType|. In the implementation, both versions
        may be used.
\end{itemize}


\paragraph{preprocessor directives}

\begin{itemize}
    \item after \verb|#define|, USE CAPITAL LETTERS
          \begin{verbatim}
#ifndef __BLA_H
#define __BLA_H
// contents of bla.h
#endif         \end{verbatim}
    \item \#ifdef and similar preprocessor directives are not indented.
    \item include guards must be used in all headers, the format is \verb|__AOL_H|
    \item openmp critical sections should be named according to the scheme
\begin{verbatim}
namespace_class_{method,other useful identifier}[_number, if needed]
\end{verbatim}
      e.g.
\begin{verbatim}
#pragma omp critical (aol_RandomGenerator_getNextRnd)
\end{verbatim}
\end{itemize}


\paragraph{structure of externals}

\begin{itemize}
\item standard externals should contain
  \begin{itemize}
  \item \texttt{makefile.local} that sets include and link paths
  \item provide an include header, marked as a system header
\begin{verbatim}
#ifdef __GNUC__
#pragma GCC system_header
#endif
\end{verbatim}
    to prevent compiler warnings for external code
  \item a short description
  \end{itemize}
\item if the external is selected in
  \texttt{makefile.selection.default}, define
  \texttt{USE$\_$EXTERNAL$\_$...} is set automatically.
\item nonstandard externals may contain (small amounts of) code that
  is compiled automatically (if necessary) by the make mechanism
  (\texttt{go} and \texttt{clean} scripts) or by an appropriate visual
  c++ project
\end{itemize}

\paragraph{using externals}
\begin{itemize}
\item all modules (except for those obviously fully depending on an
  external) must compile without the external being used
\item Programs that use external code also have to compile if the
  corresponding external is switched off. The executables should then
  give a useful and informative error message like ''This program
  can't be used without (\textit{corresponding external})''. To achieve
  this, enclose your header and cpp-files in the following
  ifdef-construction (external is GRAPE in this example):
\begin{verbatim}
#ifdef USE_EXTERNAL_GRAPE
    ... (code that uses externals) ...
#endif
\end{verbatim}
In case of executables add the follwing else-part (or a similar one):
\begin{verbatim}
#ifdef USE_EXTERNAL_GRAPE
  ... (code that uses externals) ...
#else
  int main ( int, char** ) {
    cerr << "Without grape external, this program is useless" << endl;
    return ( 0 ) ;
  }
#endif
\end{verbatim}
\end{itemize}


\paragraph{style (indentation, spaces~\ldots)}

\begin{itemize}
    \item 2 spaces are used for indentation (no tabs, not 4 spaces etc.)
    \item preprocessor directives are not indented at all.
    \item \texttt{public:} and similar are not indented relative to the class.
          In both cases, that is the current astyle standard.
    \item brackets (placement in lines and spacing around brackets) are used according to the following scheme:
          \begin{verbatim}
dummy_method ( aol::Vector<RealType> &vec, RealType factor ) {
  for ( int i = 0; i < vec.size(); ++i ) {
    vec[i] = factor * vec[i];
  }
}              \end{verbatim}
          and can be enforced automatically by using util/indent which in turn uses astyle
\end{itemize}


\paragraph{Name convention for methods that import or export data, e. g. aol::Mat A, B inversion (same for transposition etc.)}

\begin{itemize}
  \item \texttt{void A.invert()}: writes $A = A^{-1}$
  \item \texttt{B = A.inverse()}: compute and return inverse of $A$, do not modify $A$
  \item \texttt{A.invertFrom(B)}: $A = B^{-1}$, $B$ unmodified
  \item \texttt{A.invertTo(B)}: $B = A^{-1}$, $A$ unmodified
\end{itemize}


\paragraph{miscellaneous}

\begin{itemize}
  \item comments have to be written in english (except in your own projects, there you can do whatever you want)
  \item use special characters only in your own projects and only on your own risk
\end{itemize}


\paragraph{data sets}

\begin{itemize}
  \item don't commit any data sets (images etc.) except very small data sets for examples or selfTests
        (keep those in directory \texttt{examples/testdata}, files here must be sufficiently free to be usable under quoc license)
\end{itemize}



\paragraph{No convention on~\ldots}

\begin{itemize}
  \item the position of member variables, they may be at the beginning or at the end of a class
  \item No general rule on whether implementation should be inside or outside class definition.
\end{itemize}


\paragraph{Rules for subversion}

\begin{itemize}
  \item Moving code and changing code (e. g. moving implementation out of class and changing it)
        should be committed separately to allow diffing.
  \item use svn:ignore to ignore files that will automatically occur when compiling and typical temp files of editors and IDEs,
        not for personal temporary copies like aol.hold
\end{itemize}


\paragraph{Very special things}
\begin{itemize}
  \item use \texttt{aol::Abs} instead of \texttt{fabs}.
        But: template specialization is necessary for unsigned data types when needed.
  \item Instead of \texttt{M$\_$PI} the expressions \texttt{aol::PI} or
        \texttt{aol::NumberTrait<RealType>::pi} should be used. Analogously for other mathematical
        constants (if not available define own NumberTrait).
  \item \texttt{apply(x, x)} is nowhere forbidden, but produces (mostly) unpredictable output.
        Apply should check for this and throw exception or contain comment that \texttt{apply(x, x)}
        works, we will not change this in all old apply methods now.
  \item Use \texttt{for}-loops where possible, even simple things like \\
        \texttt{a[0] = expression ( 0 ); a[1] = expression ( 1 )} \\
        should be done in a \texttt{for}-loop.
  \item In methods like \texttt{getMinValue()} or \texttt{getMaxValue()} don't initialize the
        first value with $\pm$ infinity, but with \texttt{vector[0]} (otherwise, if the size of
        the vector is $0$, it might happen that $\pm$ infinity is returned). \\
        If it's not really really obvious that in no case anything can ever go wrong with the
        \texttt{[]}-operator of the vector, use \texttt{get} and \texttt{set} (then bounds-checking
        is applied in the debug-mode).
  \item use \texttt{NON\_PARALLEL\_STATIC} if \texttt{static} variables should not be static when using
        parallelization (due to conflicting write access), e.g.\ if \texttt{static} is only used for
        performance reasons. If they always need to be static, prevent parallel write access.
\end{itemize}




\InputIfFileExists{DocOOProgramming.tex}{}{}

%%%%%%%%%%%%%%%%%%%%%%%%%%%%%%%%%%%%%%%%%%%%%%%%%%%%%%%%%%%%%%%%%%%%%%%%%%%%%%%%%%
%%%%%%%%%%%%%%%%%%%%%%%%%%%%%%%%%%%%%%%%%%%%%%%%%%%%%%%%%%%%%%%%%%%%%%%%%%%%%%%%%%
%%%%%%%%%%%%%%%%%%%%%%%%%%%%%%%%%%%%%%%%%%%%%%%%%%%%%%%%%%%%%%%%%%%%%%%%%%%%%%%%%%
%%%%%%%%%%%%%%%%%%%%%%%%%%%%%%%%%%%%%%%%%%%%%%%%%%%%%%%%%%%%%%%%%%%%%%%%%%%%%%%%%%


%                 Chapter Templates, Traits, Interfaces, STL
%                 written Marc Droske


%%%%%%%%%%%%%%%%%%%%%%%%%%%%%%%%%%%%%%%%%%%%%%%%%%%%%%%%%%%%%%%%%%%%%%%%%%%%%%%%%%
%%%%%%%%%%%%%%%%%%%%%%%%%%%%%%%%%%%%%%%%%%%%%%%%%%%%%%%%%%%%%%%%%%%%%%%%%%%%%%%%%%
%%%%%%%%%%%%%%%%%%%%%%%%%%%%%%%%%%%%%%%%%%%%%%%%%%%%%%%%%%%%%%%%%%%%%%%%%%%%%%%%%%
%%%%%%%%%%%%%%%%%%%%%%%%%%%%%%%%%%%%%%%%%%%%%%%%%%%%%%%%%%%%%%%%%%%%%%%%%%%%%%%%%%
\chapter{Templates, Traits, Interfaces, STL}

\section{Generic (OO) programming concepts}

\begin{itemize}
\item Avoid code duplication.
\item Represent independent concepts separately.
\item Design your classes to be used as flexibly as possible.
\item Problem: type flexibility, robustness \& efficiency
\item \bfseries{templates} provide a mechanism for \emph{generic programming}
\end{itemize}


\section{Class parametrization via templates}
Consider a simple class to story some \id{int}s:
\begin{myverbatim}
class int_array {
  private:
    int data[1000];
  public:
    int get( int i ) { ... }
    void set( int i ) { ... }
    // dot-product of this array and other array
    int operator*( const int_array &other ) const { ... }
};
\end{myverbatim}

Later, we want to have an array of \id{float}s, \id{double}s, \id{complex} numbers,
\id{short[3]}s or other types (which can be classes) and specify a different length.


Consider the following parametrization of class \id{array} with a type \id{T}:

\begin{myverbatim}
template <typename T>
class array {
  private:
    T data[1000];
  public:
    T get( int i ) const { ... }
    void set( int i, T v ) { ... }
    T operator*( const array<T> &other ) const {
      T dot=static_cast<T>(0);
      for ( int i=0; i<1000; i++ ) {
        dot += this->get( i ) * other.get( i );
      }
      return dot;
    }
};
\end{myverbatim}


\section{Instantiation}

Instantiation of templated class is very simple:
\begin{myverbatim}
array<float> a, b;
array<int> c, d;
[...] // fill the arrays with really nice values.
float dotf = a * b;
int doti = c * d;
\end{myverbatim}

\begin{itemize}
\item \id{array<A>} and \id{array<B>} are distinct types if and only if \id{A} and \id{B} are distinct types!
\item \id{array} itself is not a class, but just a template of a class, which has to be parametrized by the type \id{T}.
\end{itemize}



\section{Parametrization by values}

We now want the array to be able to store data of different sizes.
\begin{myverbatim}
template <typename T, int SIZE = 1000>
class array {
  private:
    T data[ SIZE ];
  public:
    [...]
    T operator*( const array<T,SIZE> &other ) const {
      T dot=static_cast<T>(0);
      for ( int i=0; i<SIZE; i++ ) {
        dot += this->get( i ) * other.get( i );
      }
      return dot;
    }
};
\end{myverbatim}

{\bfseries{Template arguments have to be known at compile time.}}

Again, the type of the templated class is determined
by it's own type and all of it's template arguments..

\begin{myverbatim}
array<float,1000> a;
array<float,1001> b;
array<double,1000> c;
float dot = a * A; // fails!
float dot = a * B; // fails!

typedef float MYTYPE;
array<MYTYPE> d;

float dot = a * d;
// works.. typedefs and default args are resolved
\end{myverbatim}



\section{(Member) Function Templates}
Allow functions to be passed parameters of various types:
\begin{myverbatim}
template <typename T>
T sqr( const T& v ) { return v*v; }

class A {
  [...]
  A operator*( const A& other ) const { [...]; }
};

A a;

sqr( 4. );
sqr( 2 );
sqr( a ); // hooray!
\end{myverbatim}

or we want to have a class to fill some arrays..
\begin{myverbatim}
template <class T, int SIZE = 1000>
class array {
  [...];
  T& operator[]( int i ) { return data[i]; }
  const T& operator[]( int i ) const { return data[i]; }
};

template <typename ARR_TYPE>
void fill_array( ARR_TYPE &a ) {
  a[0] = 2.;  a[1] = 3.;  a[2] = 5.;  a[3] = 7.;  a[4] = 11.;
}

double d[5];
array<float,5> f;
fill_array( d ); // works..
fill_array( f ); // works too.. ;)
\end{myverbatim}



\section{Template specializations}

Maybe we want to define a special behaviour of the class for
some special template arguments:

\begin{myverbatim}
template <int SIZE = 1000>
class array<bool> {
  private:
    int data[ (SIZE / sizeof(int)) + (SIZE % sizeof(int) == 0) ? 0 : 1 ];
  public:
    bool get( int i ) const { /* fancy bitwise storage goes here */ };
    void set( int i, bool T ) { [...] }
};
cerr << sizeof( array<float> ) << endl;
// 1000 * sizeof(float);

cerr << sizeof( array<bool> ) << endl;
// 1000 / sizeof(int);
\end{myverbatim}


\section{Traits} Use template specializations to map compile-time types or values to
other types or values.
\begin{myverbatim}
template <typename T> class avg_trait {  // default
public:
  typedef T AVG_TYPE;
};

template<> class avg_trait<int> {  // special behaviour
public:
  typedef float AVG_TYPE;
};

template <typename T, int SIZE = 1000>
class array_with_average : public array<T,SIZE> {
public:
  avg_trait<T>::AVG_TYPE average( ) const { [...]; }
};
\end{myverbatim}


\section{Type promoters with traits:}
\begin{myverbatim}
template<typename T1, typename T2>
Vector<???> operator+(const Vector<T1> &, Vector<T2> &);
\end{myverbatim}
How to determine the return value?
\begin{myverbatim}
template <typename T1, typename T2>
struct promote_trait { };  // empty..

template <> struct promote_trait<int,float> {
  typedef float T_promote;
};
template <> struct promote_trait<char,double> {
  typedef double T_promote;
};

template<typename T1, typename T2>
Vector<promote_trait<T1,T2>::T_promote>
operator+(const Vector<T1> &, Vector<T2> &);
\end{myverbatim}


\section{Definition of member functions outside classes:}
\begin{myverbatim}
template <typename T>
class A {
public:
  void foo( const T& ) const;
};

template <typename T>
void A<T>::foo( const T& t ) const {
  [...];
}
\end{myverbatim}
It is possible to define templated member functions in \id{.cpp} files,
but the compiler has to known for which types they have to be compiled.
\begin{myverbatim}
template class A<int>;
template class A<float>;
template class A<B>;
\end{myverbatim}
If a-priori list of template arguments is known, one should always define
member functions in the header files! \\
This also reduces the size of compiled object files, precompilation into a lib
is not possible though.


\section{Interfaces: Static vs. dynamic Polymorphism}

Motivation
\begin{enumerate}
\item huge variety of libraries and special classes, each having
advantages and disadvantages in different environments
\item simplify problem-related programming by the spefication
  of a high-level interface. the programmer who is implementing
actual problems only has to know the interface, not the details
about the different libraries
\item Usually there is a significant loss of efficiency due to the interface,
 due to wrapper functions, proxy-objects and structural differences.
\end{enumerate}

How can interfaces be designed efficiently and still be easy to use?
\textbf{Interfaces: Static vs. dynamic Polymorphism} \\
(cf. T. Veldhuizen, {\em Techniques for Scientific C++})\\
{Dynamic Polymorphism:}
\begin{myverbatim}
template <typename T> class Matrix { //
public:
  virtual T operator( int i, int j ) const = 0 { };
  virtual T frobeniusNorm( ) const { [..] // call get a lot of times; }
  virtual T mult( const Vector<T> &arg, Vector<T> &dest );
};

template <typename T> class FullMatrix {
public:
  virtual T operator( int i, int j ) const { [...] };
};

template <typename T> class SymmetricMatrix {
public:
  virtual T operator( int i, int j ) const { [...] };
};
\end{myverbatim}

Now we want to use the matrices dynamically..
\begin{myverbatim}
Matrix *matrices[5];
matrices[0] = new FullMatrix<double>;
matrices[1] = new SymmetricMatrix<double>;
// etc.

double sum_frob = 0;
for (int i=0; i<5; i++ ) {
  sum_frob += matrices[i]->frobeniusNorm();
}
\end{myverbatim}

{This will be slow!!} the virtual function \id{operator( int i, int j )} will be called
often, each time a pointer lookup is necessary. \\
Resort: Implement the (virtual) function \id{frobeniusNorm} on all derived classes,
but this involves to implement a lot of similar code..

\section{Static polymorphism:} \textbf{engines}
\begin{myverbatim}
template <typename T>
class FullMatrix_engine { // storage/get/set };
template <typename T>
class SymmetricMatrix_engine { // storage/get/set };

template <T_engine>
class Matrix {
private:
  T_engine engine;
public:
  [...]
  T frobeniusNorm() const {
    T r=0;
    for ( ... ) { r += sqr( engine( i, j ) ); }
  }
};
\end{myverbatim}

\section{Static polymorphism: } \textbf{The Barton and Nackman trick}
\begin{myverbatim}
// we still want to put matrices into a container
template <typename Imp>
class A_Interface {
protected:
  Imp &asImp() { return static_cast<Imp&>( *this ); }
  const Imp &asImp() const { return static_cast<const Imp&>( *this ); }
public:
  void foo( ) { asImp().foo(); }

  void bar( ) { foo(); foo(); }
};

class A : public A_Interface<A> {
public:
  void foo( ) { // do something }
};
\end{myverbatim}


\section{Static polymorphism: } \textbf{The Barton and Nackman trick}
{\small
\begin{myverbatim}
// we still want to put matrices into a container
template <typename T, typename Imp>
class MatrixInterface {
protected:
  Imp &asImp() { return static_cast<Imp&>( *this ); }
  const Imp &asImp() const { return static_cast<const Imp&>( *this ); }
public:
  T operator( int i, int j ) const { return asImp()( i, j ); }
  T frobeniusNorm( ) const { return asImp().frobeniusNorm(); }
};

template <typename T, typename Imp>
class MatrixDefault : MatrixInterface<T, Imp>{
  T frobeniusNorm( ) const {
    T r=0;
    for ( ... ) { r += sqr((*this)( i, j ))); }
  }
};
\end{myverbatim}

Actual implementations:
\begin{myverbatim}
template <typename T>
class FullMatrix : public MatrixDefault<T,FullMatrix<T> > {
  T data[1000][1000];
public:
  T operator( int i, int j ) const { return data[i][j]; }
};

template <typename T>
class SymmetricMatrix : public MatrixDefault<T,FullMatrix<T> > {
public:
  T operator( int i, int j ) const { [..] }
  T frobeniusNorm( ) const { [..] // do some tricks here.. ; }
};

\end{myverbatim}
}


\section{Advantages}
\begin{enumerate}
\item Interface is clearly defined and more transparent.
\item No instanciations of auxiliary classes necessary.
\item Allows more conventional inheritance hierarchy.
\item Methods can be selectively overloaded in derived classes, i.~e. it is easy to
implement and change standard behaviour in a larger hierarchy.
\end{enumerate}

\textbf{Warning: Typical error} If the member function of the derived class is not exactly of the
same type (e.~g. forgot to declare as \id{const}), the call to
\id{asImp().function()} get's lost in infinite recursions..


\subsection{Rules of thumb}

\begin{itemize}
\item \id{virtual} functions \textbf{can} be slow, but the difference to inlined functions is miniscule. {When the function body is longer than a few lines, the difference may not be noticed.} Then dynamic polymorphism should be chosen over templated polymorphism for the sake of maintainability and lower compilation time.
\item What parts can I implement with static polymorphism and which not? \\
Static polymorphism has it's limitations. Everything that is known to you, the programmer, when you write the program, i.~e., at the stage of compile time, can be implemented with static polymorphism. Not more!
\end{itemize}


\section{The standard template library (STL)}


\begin{itemize}
\item Library of classes, algorithms and iterators for the generic storage and processing of instances.
\item It is entirely parametrizable by templates.
\item The heart of the STL are container classes, i.~e., classes to dynamically manage sets of objects.
\end{itemize}
Example:
\begin{myverbatim}
// create a vector of 1000 integers
vector<int> vec_of_ints( 1000 );
vec_of_ints[0] = 2;
vec_of_ints[1] = 3;
vec_of_ints[2] = vec_of_ints[0] + vec_of_ints[1];
\end{myverbatim}


\subsection{STL concepts}
\begin{itemize}
\item \textbf{storage classes/containers}. The main container classes in the STL are \id{vector}, \id{list}, \id{deque}, \id{set}, \id{map}, \id{hash\_set}, \id{hash\_map}, \id{stack}, etc. They are parametrized by the type of objects they should store and possibly by a memory manager.
\item \textbf{iterators} provide easy to use functional for the sequential traversal of the elements of a container or a subset. Iterators represent the method of choice for accessing elements of containers.
\item \textbf{algorithms}. The STL contains a variety of auxiliary algorithms which are often based on iterators, such as sorting routines, search, reversal, merging, splicing, heap operations, filling, shuffling etc. etc. The STL is considered state-of-the-art with respect to efficiency and flexibility and should be used wherever it is applicable.
\end{itemize}


\centerline{\large Overview on container classes}
\begin{itemize}
\item \textbf{vectors} represent linearly organized sequences of objects, which provides (unlike most others) \textbf{random access} to it's elements.
\item \textbf{lists} are doubly-linked lists, i.~e., access to the predecessor and successor, as well as insertion at arbitrary positions is an O(1) operation.
\item \textbf{deque} is like a vector but supports constant time insertion at the beginning.
\item \textbf{sets} represent the unstructered storage of (unique) elements. It is well suited for set operations, like $\cap$, $\cup$, $\setminus$. Sets are always sorted.
\item \textbf{maps} are parametrized by the types \id{Key} and \id{Data}, which can be understood as domain resp. range-types of a mathematical mapping. Keys are unique.
\item \textbf{stacks} provide a {\em last-in-first-out} (LIFO) in constant time class.
\item \textbf{hash\_sets} and \textbf{hash\_maps} are principally like \textbf{sets}, \textbf{maps}, but thanks to the additional specification of a hashing-function,
access to elements is in constant time (depending on the choice of the hash-function).

\item \textbf{(hash\_)multisets} and \textbf{(hash\_)multimaps} differ from the non-multi variants by the fact that they can contain multiple elements being equal, i.~e., elements are mapped to subsets of the range set, not atoms.

\item \textbf{string} classes.

\end{itemize}


\centerline{\large Iterators}

Sequential traversal of half-open intervals: \id{[begin,.........,end)}
\begin{itemize}
\item \textbf{Input iterators} read-access, but not necessarily write-access.
\item \textbf{Ouput iterators} write-access, but not necessarily read-access.
\item \textbf{Forward iterators} traversal from begin to end
\item \textbf{Reverse iterators} traversal in reverse order
\item \textbf{Bidirectonal iterators} traversal in both directions possible.
\item \textbf{Random-Access iterators} also allow index arithmetic.

\end{itemize}


\subsection{list-example}

\begin{myverbatim}
template <typename T_cont>
void dump( const T_cont& c ) {
  for ( typename T_cont::const_iterator it=c.begin();
    it!=c.end(); ++it ) { cerr << *it << " "; }
  cerr << endl;
}

list<int> l;
l.push_back( 8 ); l.push_back( 2 ); l.push_back( 6 );
dump( l );
list<int>::iterator it=l.begin();
l.insert( ++++it, 1 ); dump( l );
l.sort( );             dump( l );
Output:
8 2 6
8 2 1 6
1 2 6 8
\end{myverbatim}

\subsection{map-example}
{\small
\begin{myverbatim}
map<const char*, int, ltstr> months;
months["january"] = 31;
[...]
months["december"] = 31;

cout << "june -> " << months["june"] << endl;
map<const char*, int, ltstr>::iterator cur  = months.find("june");
map<const char*, int, ltstr>::iterator prev = cur;
map<const char*, int, ltstr>::iterator next = cur;
++next;
--prev;
cout << "Previous (alphabetically) is " << (*prev).first << endl;
cout << "Next (alphabetically) is " << (*next).first << endl;
\end{myverbatim}
Output:
\begin{myverbatim}
june -> 30
Previous (alphabetically) is july
Next (alphabetically) is march
\end{myverbatim}
}


\subsection{Memory management}

The STL has efficient built-in memory-management routines, which can
be customized (rarely-necessary).

Actually reserved memory may be larger than the amount needed by
the stored elements.
\id{vector} provides various member functions.
\begin{itemize}
\item \id{size( )} returns actual length of vector.
\item \id{capacity( )} returns how many elements fit into the reserved
memory. larger or equal than size.
\item \id{reserve( size\_t n )} ensures there is enough memory for \id{n} elements.
\item \id{resize( n, T t=T() )} resizes the vector such that length becomes \id{n}
\end{itemize}


%%% Local Variables:
%%% mode: latex
%%% TeX-master: "manual"
%%% End:


%%%%%%%%%%%%%%%%%%%%%%%%%%%%%%%%%%%%%%%%%%%%%%%%%%%%%%%%%%%%%%%%%%%%%%%%%%%%%%%%%%
%%%%%%%%%%%%%%%%%%%%%%%%%%%%%%%%%%%%%%%%%%%%%%%%%%%%%%%%%%%%%%%%%%%%%%%%%%%%%%%%%%
%%%%%%%%%%%%%%%%%%%%%%%%%%%%%%%%%%%%%%%%%%%%%%%%%%%%%%%%%%%%%%%%%%%%%%%%%%%%%%%%%%
%%%%%%%%%%%%%%%%%%%%%%%%%%%%%%%%%%%%%%%%%%%%%%%%%%%%%%%%%%%%%%%%%%%%%%%%%%%%%%%%%%


%                 Chapter The FEOPInterface
%                 written by Marc Droske


%%%%%%%%%%%%%%%%%%%%%%%%%%%%%%%%%%%%%%%%%%%%%%%%%%%%%%%%%%%%%%%%%%%%%%%%%%%%%%%%%%
%%%%%%%%%%%%%%%%%%%%%%%%%%%%%%%%%%%%%%%%%%%%%%%%%%%%%%%%%%%%%%%%%%%%%%%%%%%%%%%%%%
%%%%%%%%%%%%%%%%%%%%%%%%%%%%%%%%%%%%%%%%%%%%%%%%%%%%%%%%%%%%%%%%%%%%%%%%%%%%%%%%%%
%%%%%%%%%%%%%%%%%%%%%%%%%%%%%%%%%%%%%%%%%%%%%%%%%%%%%%%%%%%%%%%%%%%%%%%%%%%%%%%%%%

\chapter{The FEOPInterface}

\newcommand{\dx}{\,\mathrm{d}x}
\newcommand{\V}{{\mathcal V}}

\section{Finite-Element-Operators}

Example: Poisson equation
\begin{eqnarray*}
 u - \Delta u & = & f  \quad \mbox{ in } \Omega \\
\partial_\nu u &=& 0 \quad \mbox{ on } \partial \Omega
\end{eqnarray*}

weak formulation (integration, integration by parts)
\begin{equation}
\int_\Omega  u \varphi \dx + \int_\Omega \nabla u \nabla \varphi \dx = \int_{\partial \Omega } \underbrace{\nabla u \cdot \nu}_{=0} \varphi \dx + \int f \varphi \dx
\end{equation}
for all $\varphi\in H^{1,2}(\Omega)$.
\begin{equation}
 a(u,\varphi) = F( \varphi )
\end{equation}

a continuous formulation (elliptic):

bounded continuous bilinear form $a:\V\times\V:\to \R$
\begin{itemize}
\item boundedness $a(u,v) \leq C \|u\|_{\V}\|v\|_{\V}$
\item coercivity $a(u,u) \geq c \|u\|_{\V}^2$, $c>0$.
\end{itemize}

$F\in \V'$ bounded linearform.

existence and uniqueness by Lax-Milgram.

\section{Discretization}

replace $\V$ by a finite dimensional subspace $\V_h \subset \V$, with basis $(\varphi_i)_i$.
\begin{equation}
u_h = \sum_j \bar U_j \varphi_j
\end{equation}

\begin{eqnarray*}
\int_\Omega \sum_j \bar U_j \varphi_j \varphi_i \dx + \int_\Omega\sum_j \bar U_j \nabla \varphi_j \cdot \nabla \varphi_i \dx  & = & F(\varphi_i)  \\
 \sum_j \bar U_j \underbrace{\int_\Omega \varphi_j \varphi_i \dx}_{=: \mathbf{M}_{ij}} + \sum_j \bar U_j \underbrace{\int_\Omega\nabla \varphi_j \cdot \nabla \varphi_i \dx}_{=:\mathbf{L}_{ij}}  & = & F(\varphi_i) \\
(\mathbf{M} + \alpha \mathbf{L}) \bar U &=& \bar F
\end{eqnarray*}

How to construct finite dimensional subspaces?

\begin{itemize}
\item Partition of $\Omega_h$ into a triangulation $\bar \Omega = \bigcup_{T\in \mathcal{T}} \bar T $
\item local basis on cells $\bar T$ (Lagrange-basis, Hermite-basis, etc. )
\end{itemize}


%\input{ref_cell.pstex_t}

Assembly of $\mathbf{L}$: Traverse all elements, compute for all
$i, j$, such that $\mathrm{supp} \varphi_i \cap \mathrm{supp} \varphi_j \cap T \neq \emptyset$, the integrals $\int_T \nabla \varphi_i \nabla \varphi_j \dx$.

For QuocMeshes $T(x) = h x + b$, hence
\begin{eqnarray*}
\int_T \nabla \hat \varphi_i\circ \phi^{-1} \nabla \hat \varphi_j\circ\phi^{-1}\dx & = &
\frac{|T|}{|\hat T|} \int_{\hat T} \nabla \hat \varphi_i \cdot \nabla  \hat \varphi_j h^{-2}\dx \\
& =  & h^{d-2} \int_{\hat T} \nabla \hat \varphi_i \cdot \nabla  \hat \varphi_j\dx
\end{eqnarray*}

\section{main ingredients}

\begin{itemize}
\item Definition of the discrete function space ($\leadsto$ \id{Configurators})
  \begin{itemize}
  \item iterator over cells $\leadsto$ \id{qc::GridDefinition::OldAllElementIterator}
  \item mapping of local indices to global indices $\leadsto$ \id{qc::FastILexMapper}
  \item quadrature rules on reference elements e.g. $\leadsto$ \id{aol::GaussQuadrature}
  \item definition of basefunction set $\leadsto$ \id{BaseFunctionSet}, the basefunctionset also does cached evaluation
for quadrature.
  \end{itemize}
\item Finite-Element operator related
  \begin{itemize}
    \item assembly of local matrices.
  \end{itemize}
\end{itemize}

{\small
Configurators define and provide:
\begin{myverbatim}
class MyConfiguratorForBilinear2DElements {
  typedef qc::Element                                ElementType;
  typedef qc::GridDefinition::OldAllElementIterator  ElementIteratorType;
  typedef RealType                                   Real;
  typedef qc::GridDefinition                         InitType;

  const ElementIteratorType &begin( ) const;
  inline const ElementIteratorType &end( ) const;
  RealType H( const qc::Element& ) const;

  typedef Vec2<_RealType>     VecType;
  typedef Matrix22<_RealType> MatType;

  typedef qc::FastUniformGridMatrix<_RealType,qc::QC_2D> MatrixType;
  typedef BaseFunctionSetMultiLin<...,_QuadType> BaseFuncSetType;

  static const int maxNumLocalDofs = 4;
  static const qc::Dimension Dim = qc::QC_2D;
  int getNumLocalDofs( const qc::Element & ) const;
  int getNumGlobalDofs( ) const;
  const BaseFuncSetType& getBaseFunctionSet( const qc::Element &El ) const;
  int localToGlobal( const qc::Element &El, const int localIndex ) const;
  MatrixType* createNewMatrix( );
};
\end{myverbatim}
}

In the quocmesh library exist default configurator classes by using (nested) traits:
\begin{myverbatim}
typedef
qc::QuocConfiguratorTraitMultiLin<
         REAL,         // RealType
         qc::QC_2D,    // Dimension
         aol::GaussQuadrature<REAL,qc::QC_2D,3> >
ConfType;
\end{myverbatim}
The last line specifies the type of Quadrature to be used, here
Gauss-Quadrature of order $3$.

with this at hand the setup of a stiffness or mass matrix is all that easy:
\begin{myverbatim}
 aol::StiffOp<ConfType> stiff( grid, MODE );
 aol::MassOp<ConfType>  mass( grid, MODE );
\end{myverbatim}
where \id{MODE} is either \id{aol::ASSEMBLED} or \id{aol::ONTHEFLY}.
On the fly operators don't need any memory but are slower.

\section{Support for lumped mass-matrices}

standard lumped mass matrix defined by
\begin{equation}
(\mathbf{M}_h)_{ij} := \int_\Omega \mathcal{I}_h(\varphi_i \varphi_j)) \dx
\end{equation}

\begin{myverbatim}
aol::LumpedMassOp<ConfType>
     lumpedMass( grid, INVERT_MODE );
\end{myverbatim}
\id{INVERT\_MODE} is either \id{aol::INVERT} or \id{aol::DO\_NOT\_INVERT}


Consider heat equation $(\mathbf{M} + \tau \mathbf{L})\bar U^{n+1} = \mathbf{M}U^{n}$.
\begin{myverbatim}
aol::StiffOp<ConfType> stiff( grid );
aol::MassOp<ConfType>  mass( grid );
aol::LinCombOp<aol::Vector<REAL> > op;
op.append( mass );
op.append( stiff, tau );
aol::CGInverse<aol::Vector<REAL> > inv( op );
mass.apply( u_old, rhs );
inv.apply( rhs, u_new );
\end{myverbatim}

That's all.. :-)


Alternative: FEOp's can assemble themselves into other matrices:
\begin{myverbatim}
aol::StiffOp<ConfType> stiff( grid );
aol::MassOp<ConfType>  mass( grid );
aol::SparseMatrix<REAL> mat( grid );
stiff.assembleAddMatrix( mat );
mat *= tau;
mass.assembleAddMatrix( mat );
aol::CGInverse<aol::Vector<REAL> > inv( mat );
mass.apply( u_old, rhs );
inv.apply( rhs, u_new );
\end{myverbatim}


There exist easy to use interfaces for operators of the form
\begin{itemize}
\item $\div( a(x) \nabla u )$ $a$ scalar, $\leadsto$ \id{FELinScalarWeightedStiffInterface}
\item $\div( A(x) \nabla u )$ $A$ a matrix, $\leadsto$ \id{FELinMatrixWeightedStiffInterface}
\item $a(x) u$, $\leadsto$ \id{FELinScalarWeightedMassInterface}
\end{itemize}

Generation of a discrete function given a vector of coefficients.
\begin{myverbatim}
aol::DiscreteFunction<ConfType> discFunc( grid, coeffs );
discFunc.evaluate( El, locCoords );     // slow
discFunc.evaluateAtQuadPoint( El, i );  // fast
discFunc.evaluateGradient( El, locCoords, grad );     // slow
discFunc.evaluateGradientAtQuadPoint( El, i, grad );  // fast
\end{myverbatim}

\section{Customization: example mean curvature flow}
\begin{multline*}
\partial_t u + \div\left\{ \frac {\nabla u}{\|\nabla u\|} \right\} \|\nabla u\| = 0 \quad \Rightarrow \quad
\left(\boldsymbol{M}+ \tau \boldsymbol{L}\right) \bar U^{n+1} = \boldsymbol{M} U^n \\
\mbox{ where } \quad \boldsymbol{M}_{ij} = \int_\Omega \frac {\varphi_i \varphi_j}{\|\nabla U^{n} \|_\epsilon } \dx \quad \mbox{ and } \quad \boldsymbol{L}_{ij} = \int_\Omega \frac {\nabla \varphi_i \cdot \nabla \varphi_j}{\|\nabla U^{n} \|_\epsilon } \dx
\end{multline*}

How to implement the matrices with the \id{FEOpInterface}-classes?

\begin{itemize}
\item $\boldsymbol{M}$ $\leadsto$ derive from \id{FELinScalarWeightedMassInterface}
\item $\boldsymbol{L}$ $\leadsto$ derive from \id{FELinScalarWeightedStiffInterface}
\end{itemize}

{\small
\begin{myverbatim}
template <typename Conf_T>
class MCMStiffOp :
public aol::FELinScalarWeightedStiffInterface<Conf_T, MCMStiffOp<Conf_T> > {
public:
  typedef typename Conf_T::RealType RealType;
protected:
  aol::DiscreteFunctionDefault<Conf_T> *_discFunc;
  RealType _eps;
public:
  MCMStiffOp( const typename Conf_T::InitType &Initializer,
              aol::OperatorType OpType = aol::ONTHEFLY,
              RealType Epsilon = 1. )
: aol::FELinScalarWeightedStiffInterface<Conf_T, MCMStiffOp<Conf_T> >( Initializer, OpType ),
      _discFunc( NULL ), _eps( Epsilon )  {
  }

  void setImage( const aol::Vector<RealType> &Image ) { [...] }

  inline RealType getCoeff( const qc::Element &El, int QuadPoint,
                            const typename Conf_T::VecType& RefCoord ) const {
    if ( !_discFunc ) {  throw aol::Exception( "first!", __FILE__, __LINE__ );  }
    typename Conf_T::VecType grad;
    _discFunc.evaluateGradientAtQuadPoint( El, QuadPoint, grad );
    return 1. / sqrt( grad.normSqr() + _eps*_eps );
  }
};
\end{myverbatim}
}


\section{Advantages}
\begin{enumerate}
\item minimization of code duplication $\leadsto$ concentrate on your problem.
\item effiency due to inlined \id{getCoeff} function
\item works automatically in 2D as well as 3D!
\item works for arbitrary finite element spaces
\item works for arbitrary quadrature rules
\item works for arbitrary grids.
\end{enumerate}
\section{this class (and others) are already implemented in \id{mcm.h}}



\section{Other classes---for right hand sides:}
\begin{itemize}
\item for \id{Vector}'s
  \begin{itemize}
  \item \id{FENonlinOpInterface} $\leadsto$ $\int_\Omega f(U)\varphi_i\dx$
  \item \id{FENonlinDiffOpInterface} $\leadsto$ $\int_\Omega \vec f(U)\cdot \nabla \varphi_i\dx$
  \end{itemize}
\item for \id{MultiVector}'s
  \begin{itemize}
  \item \id{FENonlinVectorOpInterface} $\leadsto$ $\int_\Omega \vec f(U)\varphi_i\dx$
  \item \id{FENonlinVectorDiffOpInterface} $\leadsto$ $\int_\Omega \boldsymbol{F}(U)\cdot \nabla \varphi_i\dx$
  \end{itemize}
\end{itemize}


\section{Overview over Integral-Interface-classes}
Here, an overview over the finite element interface classes is given.
We will distinguish between linear operator interfaces (which additionally
to "apply(...)" and "applyAdd(...)" have a method "assembleAddMatrix(...)" to assemble a system matrix)
and nonlinear operators (which cannot assemble a matrix).
The first table enlists the linear operator interfaces, where the second column displays the matrix assembled in "assembleAddMatrix(...)"
(the result of "apply(...)" then is just this matrix multiplied by the vector of function values, passed to "apply(...)" as the first argument),
the left column gives the corresponding class names, and the right column enlists the methods to be overloaded.
The second table shows all nonlinear operator interfaces and has the same structure, only the middle column shows the result of the method "apply(...)".
In both cases, $\phi$ shall represent the discretized function, which is passed to "apply(...)" or "applyAdd(...)" as the first argument,
$\varphi_i$ represents the $i$th finite element base function, $w$, $f$, $A$ are functions, which have to be implemented by the overload method
in the derived class. An arrow over a function signifies that the function is vector-valued, two arrows
symbolize a matrix. The implementation of $f\left(\phi(x),\nabla\phi(x),x\right)$ gets $x$ and the function $\phi$ as argument, where $\phi$
itself may be evaluated at $x$, or its gradient, or both. $n_g$ shall denote the number of components of a vector function $\vec g$.

{
\renewcommand{\arraystretch}{2}
\resizebox{\textwidth}{!}{
\begin{tabular}{llll}
  \hline
    \textbf{FELin...Interface} & \textbf{expression} & \textbf{overload method} & \textbf{comment} \\
  \hline
    ScalarWeightedMass      & $\left(\int_\Omega w(x) \varphi_j(x) \varphi_i(x) dx\right)_{ij}$                                                            & $w$: getCoeff() \\
    ScalarWeightedStiff     & $\left(\int_\Omega w(x) \nabla\varphi_j(x) \cdot \nabla\varphi_i(x)dx \right)_{ij}$                                          & $w$: getCoeff() \\
    MatrixWeightedStiff     & $\left(\int_\Omega A(x)\nabla \varphi_j \cdot \nabla \varphi_i dx\right)_{ij}$                                               & $A$: getCoeffMatrix() & $A$ symmetric \\
    AsymMatrixWeightedStiff & $\left(\int_\Omega A(x)\nabla \varphi_j \cdot \nabla \varphi_i dx\right)_{ij}$                                               & $A$: getCoeffMatrix() & $A$ asymmetric \\
    ScalarWeightedMixedDiff & $\left(\int_\Omega w(x) \frac{\partial\varphi_j(x)}{\partial x_s} \frac{\partial\varphi_i(x)}{\partial x_t} dx\right)_{ij}$  & $w$: getCoeff() \\
    ScalarWeightedSemiDiff  & $\left(\int_\Omega \varphi_j(x) \frac{\partial\varphi_i(x)}{\partial x_k} w(x)\,dx\right)_{ij} $                             & $w$: getCoeff() & can also return the transpose \\
    VectorWeightedSemiDiff  & $\left(\int_\Omega \varphi_j(x) (\nabla \varphi_i(x) \cdot \vec{w}(x))\,dx\right)_{ij} $                                     & $\vec{w}$: getCoefficientVector() & can also return the transpose, \\
                            &                                                                                                                              & & $n_w=$domain dimension \\
  \hline
\end{tabular}
}
}

\vspace*{1cm}
{
\renewcommand{\arraystretch}{2}
\resizebox{\textwidth}{!}{
\begin{tabular}{llll}
  \hline
    \textbf{FENonlin...Interface} & \textbf{expression} & \textbf{overload method} & \textbf{comment} \\
  \hline
    Op                       & $\left(\int_\Omega f\left(\phi(x),\nabla\phi(x),x\right) \varphi_i(x) dx\right)_i $                                           & $f$: getNonlinearity() \\
    DiffOp                   & $\left(\int_\Omega \vec{f}\left(\phi(x),\nabla\phi(x),x\right)\cdot \nabla\varphi_i(x) dx\right)_i $                          & $\vec{f}$: getNonlinearity() \\
    VectorOp                 & $\left(\int_\Omega \vec{f}\left(\vec\phi(x),\nabla\vec\phi(x),x\right)\cdot\vec\varphi_i(x) dx\right)_i $                     & $\vec{f}$: getNonlinearity() & $n_f$, $n_\phi$ are template parameters \\
    VectorDiffOp             & $\left(\!\!\int_\Omega \!\!\vec{\vec{f}}\left(\vec\phi(x),\nabla\vec\phi(x),x\right):\nabla \vec\varphi_i(x) dx\!\!\right)_i$ & $\vec{\vec{f}}$:getNonlinearity() \\
    IntegrationScalar        & $\int_\Omega f\left(\phi(x),\nabla\phi(x),x\right) dx$                                                                        & $f$: evaluateIntegrand() \\
    IntegrationVector        & $\int_\Omega f\left(\vec\phi(x),\nabla\vec\phi(x),x\right) dx$                                                                & $f$: evaluateIntegrand() & $n_\phi$ is template parameter \\
    IntegrationVectorGeneral & $\int_\Omega f\left(\vec\phi(x),\nabla\vec\phi(x),x\right) dx$                                                                & $f$: evaluateIntegrand() & $n_\phi\leq$domain dimension is variable \\
  \hline
\end{tabular}
}
}
VectorFENonlinIntegrationVectorInterface can be used for vector valued integrands.



%%% Local Variables:
%%% mode: latex
%%% TeX-master: "manual"
%%% End:


\InputIfFileExists{DocSelectedProblems.tex}{}{}


%%%%%%%%%%%%%%%%%%%%%%%%%%%%%%%%%%%%%%%%%%%%%%%%%%%%%%%%%%%%%%%%%%%%%%%%%%%%%%%%%%
%%%%%%%%%%%%%%%%%%%%%%%%%%%%%%%%%%%%%%%%%%%%%%%%%%%%%%%%%%%%%%%%%%%%%%%%%%%%%%%%%%
%%%%%%%%%%%%%%%%%%%%%%%%%%%%%%%%%%%%%%%%%%%%%%%%%%%%%%%%%%%%%%%%%%%%%%%%%%%%%%%%%%
%%%%%%%%%%%%%%%%%%%%%%%%%%%%%%%%%%%%%%%%%%%%%%%%%%%%%%%%%%%%%%%%%%%%%%%%%%%%%%%%%%


%                 Chapter "Boundary Element Methods"
%                 written by Martin Lenz


%%%%%%%%%%%%%%%%%%%%%%%%%%%%%%%%%%%%%%%%%%%%%%%%%%%%%%%%%%%%%%%%%%%%%%%%%%%%%%%%%%
%%%%%%%%%%%%%%%%%%%%%%%%%%%%%%%%%%%%%%%%%%%%%%%%%%%%%%%%%%%%%%%%%%%%%%%%%%%%%%%%%%
%%%%%%%%%%%%%%%%%%%%%%%%%%%%%%%%%%%%%%%%%%%%%%%%%%%%%%%%%%%%%%%%%%%%%%%%%%%%%%%%%%
%%%%%%%%%%%%%%%%%%%%%%%%%%%%%%%%%%%%%%%%%%%%%%%%%%%%%%%%%%%%%%%%%%%%%%%%%%%%%%%%%%



\chapter {Boundary Element Methods}

\section {Overview}

Framework for Boundary Element Methods in 2D, Namespace \lstinline$bm$

\begin {itemize}
\item Boundary Discretization\\
(framework, polygons, rectangles)
\item Boundary Integral Operators\\
(laplacian, elasticity)
\item Utilities
\end {itemize}

\section {Model Problem}

Consider a harmonic function $u$, i.e. $\Delta u = 0$ on $\Omega$.\\
Let $\psi_y (x) = -\frac{1}{2\pi }\ln \left| x-y\right|$ be the fundamental solution to the Laplacian in $\R^2$ centered at $y$.
Now we have for $y \in \partial \Omega$ the boundary integral equation ($\nu$ the outer normal)
\[ \frac {1}{2} u(y) = \int_{\partial \Omega} u(x) \partial_\nu \psi_y (x) - \partial_\nu u(x) \psi_y(x) \, ds_x \,, \]
that relates Dirichlet and Neumann boundary conditions.

This allows us to solve boundary value problems by considering an integral equation only on the boundary.

\section {Boundary Discretization}

Consider Integral Equations of the type
\[ (K u)(y) = f(y) \,, \qquad K u = \int_\Gamma u(x) \psi_y(x) \, ds_x \]
with some integral kernel $\psi_y$ that is usually nonlocal and
possibly singular at $y \in \Gamma$.

Ansatz: Approximate $u$ in some basis: $u (x) = \sum_j u_j \phi_j (x)$. Choose some points
$y_i \in \Gamma$, where we will assume the integral equation to be solved exactly. This yields
a linear system that is to be solved:
\[ \sum_j K_{ij} u_j = f(y_i) \,, \qquad K_{ij} = \int_\Gamma \phi_j (x) \psi_{y_i} (x) \, ds_x \]

\subsection {Abstract Framework}

For the target application, the boundary consists of numerous closed fragments.

So basic building block is a \lstinline$Particle$, i.e. a closed part of the boundary.

Particles are assumed to be polygons, that means one can iterate through the segments.

The set of all particles is organized in a list.



Particles have Iterators over segments\\
Barton--Nackman Interface, file \lstinline$particle.h$

\begin{myverbatim}
template <class Implementation, class IteratorType,
 class ConstIteratorType, class SegmentType,
 class ConstSegmentType, class DataType>
  class Particle {
  public:
    IteratorType beginSegment ();
    IteratorType endSegment ();
    ConstIteratorType beginSegment () const;
    ConstIteratorType endSegment () const;
    int getNumberOfSegments () const;
    void getLengths (Vector<DataType>& lengths) const;
    // ...
}
\end{myverbatim}



General segment Interface, file \lstinline$segment.h$

\begin{myverbatim}
template <class Implementation, class DataType>
  class Segment {
  public:
    aol::Vec2<DataType> getStart () const;
    aol::Vec2<DataType> getEnd () const;
    aol::Vec2<DataType> getDirection () const;
    aol::Vec2<DataType> getNormal () const;
    DataType getLength () const;
}
\end{myverbatim}



Usually one wants to store the real information in the particles.
Segments only contain a reference to a particle and an index. Most basic operations (including iterating
over them) are quite easy and efficient.

\begin{myverbatim}
template <class Implementation, class ParticleType>
  class ReferenceSegment
  : public Segment<Implementation,
   typename ParticleType::DataType> {
  public:
    typedef typename ParticleType::IndexType IndexType;
    bool operator == (const ReferenceSegment& segment) const;
    bool isFollowedBy (const ReferenceSegment& segment) const;
  protected:
    ParticleType* _particle;
    IndexType _index;
}
\end{myverbatim}



Boundary is a list of particles, file \lstinline$boundary.h$

\begin{myverbatim}
template <class ParticleType>
  class Boundary
  : public list<ParticleType> {
  public:
    typedef typename ParticleType::DataType DataType;
    AllSegmentIterator<ParticleType> beginSegment ();
    AllSegmentIterator<ParticleType> endSegment ();
    ConstAllSegmentIterator<ParticleType> beginSegment () const;
    ConstAllSegmentIterator<ParticleType> endSegment () const;
    int getNumberOfSegments () const;
    void getLengths (Vector<DataType>& lengths) const;
    // ...
}
\end{myverbatim}



The segment iterators on the boundary iterate over the particles, then over the segments
within. For this general concept of nested iterators, there is a utility class. Due to templatization, this is abstract
with respect to the const-ness of segments.

\begin{myverbatim}
template <class BoundaryType, class ParticleIteratorType,
 class SegmentIteratorType, class SegmentType>
  class PipingIterator {
  public:
    // Iterator interface

  protected:
    ParticleIteratorType _particleBegin,
                         _particleEnd, _particleCurrent;
    SegmentIteratorType _segmentBegin,
                        _segmentEnd, _segmentCurrent;
}
\end{myverbatim}



Example: Implementation of \lstinline$PipingIterator::operator++$

\begin{myverbatim}
  template <...>
    inline PipingIterator<...>& PipingIterator<...>::operator ++ ()
    {
      ++_segmentCurrent;
      if (_segmentCurrent == _segmentEnd) {
        ++_particleCurrent;
        updateSegment (false);
      }

      return *this;
    }
\end{myverbatim}



\begin{myverbatim}
  // Update segment iterators after particle changed
  template <...>
    inline void PipingIterator<...>::updateSegment (bool end)
    {
      if (_particleCurrent == _particleEnd) return;

      _segmentBegin = _particleCurrent->beginSegment ();
      _segmentEnd = _particleCurrent->endSegment ();
      _segmentCurrent = end ? _segmentEnd : _segmentBegin;
    }
\end{myverbatim}



Implementation of \lstinline$AllSegmentIterator$ and \lstinline$ConstAllSegmentIterator$
is now trivial with almost no code:

\begin{myverbatim}
template <class ParticleType>
  class AllSegmentIterator
  : public PipingIterator<Boundary<ParticleType>,
   typename Boundary<ParticleType>::iterator,
   typename ParticleType::SegmentIteratorType,
   typename ParticleType::SegmentType> {
  public:
    typedef typename ParticleType::SegmentType SegmentType;
    SegmentType& operator * () {
      return *_segmentCurrent;
    }
}
\end{myverbatim}

\subsection {General Implementation: Polygons}

The particle contains a list of points and
some interface functions. File \lstinline$parametric.h$

\begin{myverbatim}
template <class _DataType = double, class _IndexType = int>
  class ParaParticle
  : public Particle<ParaParticle<_DataType,_IndexType>,
   ParaSegmentIterator<_DataType,_IndexType>,
   ConstParaSegmentIterator<_DataType,_IndexType>,
   ParaSegment<_DataType,_IndexType>,
   ConstParaSegment<_DataType,_IndexType>,_DataType> {
   public:
     aol::Vec2<DataType>& getPoint (IndexType index);
   private:
     std::vector<aol::Vec2<DataType> > _points;
}
\end{myverbatim}



The \lstinline$ParaSegment$ class must contain the code for the
\lstinline$Segment$ interface using the list of points.

\begin{myverbatim}
template <class _DataType = double, class IndexType = int>
  class ParaSegment
  : public ReferenceSegment<ParaSegment<_DataType,IndexType>,
   ParaParticle<_DataType,IndexType> > {
  protected:
    void updatePoints ()
    {
      _start = _particle->getPoint (_index);
      _end = _particle->getPoint (_index + 1);
    }
}
\end{myverbatim}

\subsection {Space- and Time-Optimized Implementation: Rectangles}

A rectangle needs only to store two points. File \lstinline$rectangle.h$

\begin{myverbatim}
template <class _DataType = double>
  class RectParticle
  : public Particle<RectParticle<_DataType>,
   RectSegmentIterator<_DataType>,
   ConstRectSegmentIterator<_DataType>,
   RectSegment<_DataType>,
   ConstRectSegment<_DataType>,_DataType> {
  private:
    aol::Vec2<DataType> _lowerLeft, _upperRight;
}
\end{myverbatim}



\begin{myverbatim}
template <class DataType>
void RectSegment<DataType>::updatePoints () {
  aol::Vec2<DataType> ll = _particle->getLowerLeft ();
  aol::Vec2<DataType> ur = _particle->getUpperRight ();

  DataType top = ur.y (), left = ll.x ();
  DataType bottom = ll.y (), right = ur.x ();

  switch (_index) {
  case TOP:
    _start = aol::Vec2<DataType> (right, top);
    _end = aol::Vec2<DataType> (left, top);
    break;
  // ...
  }
}
\end{myverbatim}

\section {Boundary Integral Operators}

Discrete Integral operators are full matrices, constructed with some\\
\lstinline$LocalOperatorType$
that defines the evaluation of the integral of the kernel and some basis function:
\[ A_{ij} = \int_\Gamma \psi_{y_i} (x) \phi_j (x) \, ds_x \]

\begin{myverbatim}
template <class LocalOperatorType>
  class IntegralOperator
  : public aol::FullMatrix<typename LocalOperatorType
                           ::ParticleType::DataType> {
  IntegralOperator (const LocalOperatorType& localOp,
                    const Boundary<typename LocalOperatorType
                                   ::ParticleType>& boundary);
}
\end{myverbatim}

\subsection {Concrete Operators}

Our ansatz is based on integrating the kernel explicitly, so that we must provide
the kernel function itself and the analytical integrals needed for the matrix construction.
The basic local operator considers piecewise constant ansatz functions.

\begin{myverbatim}
template <class _ParticleType>
  class SingleLayerPotential
  : public LocalOperator<SingleLayerPotential<_ParticleType>,
                         _ParticleType> {

  public:
    typedef _ParticleType ParticleType;
    typedef typename ParticleType::DataType DataType;
    typedef typename ParticleType::ConstSegmentType
                                   ConstSegmentType;

    using LocalOperator<...>::evaluate;
    using LocalOperator<...>::evaluateLocally;

    DataType evaluate (aol::Vec2<DataType> point,
                       aol::Vec2<DataType> center) const;
    DataType evaluateLocally (const ConstSegmentType& integrate,
                              const aol::Vec2<DataType>& evaluate,
                              bool isstart, bool isinterior, bool isend) const;
  };
\end{myverbatim}

\lstinline$evaluate (p, c)$ return $\psi_c(p)$

\lstinline$evaluateLocally (i, e, ...)$ returns $\int_\Gamma \psi_{e} (x) \phi_i (x) \, ds_s$, where $\psi_e$ is centered at the middle of the segment $e$,
$\phi_i$ is the characteristic function of the segment $i$ (which then is obviously the integrational domain), and the bools have to be set appropriately
when the singularity of $\psi_e$ is somewhere on $i$.



There are other types of operators for piecewise linear ansatz functions and for different types of integral kernels.

\begin {itemize}

\item Single layer, double layer, and hypersingular operator

\item Laplacian and anisotropic elasticity

\item Piecewise constant and linear ansatz functions

\end {itemize}

\section {Utilities}

\subsection {Parameter Parser}

Parser for parameter files of the following type
\begin{myverbatim}
int     numerOfSteps  100
double  tau           1E-3
string  filename      data
bool    doit          1
\end{myverbatim}

Extensible to other types, by simply provinding an appropriate \lstinline$operator >>$
and adding one line to \lstinline$ParameterParser::read$
\begin{myverbatim}
map <string,TypePtr> types;
types["bool"]                  = &typeid (bool);
types["aol::Vector<double>"]   = &typeid (aol::Vector<double>);
\end{myverbatim}



Usage in main program is simple: With no runtime argument, the parser reads from
the default file \lstinline$parameter$, otherwise from the file with the provided filename.

\begin {myverbatim}
int main (int argc, char* argv[])
{
  aol::ParameterParser parameter (argc, argv);
  double tau;
  parameter.get ("tau", tau);
}
\end{myverbatim}

On initialization, the type of variables is identified by the string in the \lstinline$types$ map,
and the value is read to a string.
Then \lstinline$ParameterParser::get$ is overloaded for any type. It ensures that it is called exactly
for the right type by a \lstinline$typeid$ comparison and the converts the data via \lstinline$operator >>$.

\subsection {IO Utilities}

Functions for binary IO with streams\\
(encapsulates some ugly casting), file \lstinline$genmesh.h$

\begin{myverbatim}
//! Binary input from stream,
//! nothing is done about byte order
template <class DataType>
  void readbinary (istream& is, DataType& data);

//! Binary output to stream,
//! nothing is done about byte order
template <class DataType>
  void writebinary (ostream& os, const DataType& data);
\end{myverbatim}

\subsection {Memory Usage}

For space complexity estimates: memory usage of running program\\
(adds the right values of the \lstinline$mallinfo$ structure), file \lstinline$genmesh.h$

\begin{myverbatim}
//! Returns number of bytes used by program
int memusage ();
\end{myverbatim}

%%% Local Variables:
%%% mode: latex
%%% TeX-master: "manual"
%%% End:


\InputIfFileExists{DocNewtonMethod.tex}{}{}

\InputIfFileExists{DocTimestepping.tex}{}{}

\InputIfFileExists{DocGradientflow.tex}{}{}



\chapter{Solver (Direct, Iterative)}
\section{Implemented solver}

%\vspace{-.7cm}
\begin{minipage}{\linewidth}
\begin{minipage}[t]{0.5\linewidth}
\underline{ direct solver: }
\begin{itemize}
\item LUInverse
\item (QRDecomposeGivernsBase)
\item (QRDecomposeHouseholderMatrix)
\end{itemize}
\underline{ iterative solver: }
\begin{itemize}
\item CGInverse
\item PCGInverse
\item BiCGInverse
\item PBiCGInverse
\item PBiCGStabInverse
\item GMRESInverse
\item GaussSeidelInverse
\item (JacobiSmoother)
\item (abstractMultigrid)
\end{itemize}
\end{minipage}
\begin{minipage}[t]{0.5\linewidth}
\underline{ preconditioner: }
\begin{itemize}
\item ILU0Preconditioner
\item SSORPreconditioner
\end{itemize}
\underline{ relicts (not for use): }
\begin{itemize}
\item EllipticSolver
\item ParabolicTimestep
\item ParabImplEuler
\item CGInverseProjection
\item ApproxCGInverse
\end{itemize}
\underline{ not implemented: }
\begin{itemize}
\item multigraph
\end{itemize}
\end{minipage}
\end{minipage}

%%%%%%%%%%%%%%%%%%%%%%%%%%%%%%%%%%%%%%%%%%%%%%%%%%%%%%%%%%%%%%%%%%%%%%%%%%%%%%%

\subsection { QRDecompose}
\subsubsection{ QRDecomposeGivensBase, QRDecomposeHouseholderMatrix: } in \texttt{aol/QRDecomposition.h}
\begin{itemize}
\item \textbf{no solver!}
\item QRDecomposeGivensBase: \\
protected method \texttt{eliminate( R, col, toEliminate, eliminateFrom )}:\\
eliminates (toEliminate) element in column (col) with (eliminateFrom) \\
not for use ( abstract function \texttt{transform} )
\item QRDecomposeHouseholderMatrix: derived from QRDecomposeGivensBase\\
for right upper triangle matrices with occupied secondary diagonal
\item using: \begin{quote}
\texttt{aol::QRDecomposeHouseholderMatrix< RealType > QRH; \\
QRH.transform( H, R, Q );} \end{quote}
with \texttt{H} H-matrix of GMRES-method
\item with matrices \texttt{FullMatrix< RealType > }
\begin{flushright} \texttt{H( row, col ), R( row, col ), Q( row, row ); } \end{flushright}
\end{itemize}


%%%%%%%%%%%%%%%%%%%%%%%%%%%%%%%%%%%%%%%%%%%%%%%%%%%%%%%%%%%%%%%%%%%%%%%%%%%%%%%

\subsection {Using solver}
we want to solve the system of linear equations $Ax = b$ with a properly solver

$//$ declaration and instantiation:\\
$//$ operator\\
\texttt{{\em solverclass}< {\em templateparameter} > A( {\em parameterlist} ); }\\[2ex]
$//$\texttt{b}: right hand side \\
$//$\texttt{x}: initial value $\rightarrow$ solution \\
\texttt{VectorType b( {\em parameterlist} ), x( {\em parameterlist} ); \\
initialise\underline{ }rhs( b ); \\ initialise\underline{ }start( x );}\\[2ex]
$//$ solve equation Ax=b\\
\texttt{A.apply( b, x );}

after iteration: \texttt{x} = solution of equation


%%%%%%%%%%%%%%%%%%%%%%%%%%%%%%%%%%%%%%%%%%%%%%%%%%%%%%%%%%%%%%%%%%%%%%%%%%%%%%%

\section {Direct solver for $Ax=b$}
\subsection{ LUInverse: }
\begin{itemize}
\item derived from \texttt{Matrix< RealType >} in \texttt{project$/$bemesh$/$luinverse.h}
\item uses methods \texttt{makeLU} and \texttt{solveLU} of \texttt{FullMatrix}
\item using: \begin{quote}
\texttt{aol::LUInverse< RealType > luSolver(  Matrix\underline{ }A  ); \\
luSolver.apply( b, x );} \end{quote}
\item with matrix \texttt{aol::Matrix< RealType > Matrix\underline{ }A( row, col );}
\end{itemize}

%%%%%%%%%%%%%%%%%%%%%%%%%%%%%%%%%%%%%%%%%%%%%%%%%%%%%%%%%%%%%%%%%%%%%%%%%%%%%%%

\section {Iterative solver for $Ax = b$}
\label{sec:iterativeSolvers}
\subsection{ CGInverse: }
\begin{itemize}
\item derived from \texttt{InverseOp }in \texttt{aol::Solver.h}
\item Conjugate Gradient method, most effective for spd matrices (symmetric, positive definit)
\item using: \begin{quote}
\texttt{aol::CGInverse< VecType, OpType = Op< VecType > >}
\begin{flushright} \texttt{cgSolver( Operator\underline{ }A, Epsilon, MaxIter );} \end{flushright}
\texttt{cgSolver.apply( b, x ); } \end{quote}
\item default values: \\
\texttt{Epsilon = 1.e-16, \\ MaxIter = 1000}
\end{itemize}

%%%%%%%%%%%%%%%%%%%%%%%%%%%%%%%%%%%%%%%%%%%%%%%%%%%%%%%%%%%%%%%%%%%%%%%%%%%%%%%

\subsection{ PCGInverse: }
\begin{itemize}
\item derived from \texttt{InverseOp }in \texttt{aol::Solver.h}
\item CG method with preconditioning
\item using: \begin{quote}
\texttt{aol::PCGInverse< VecType,}
$\begin{array}[t]{l}
\texttt{OpType = Op< VecType >,} \\ \texttt{iOpType = Op< VecType > > }
\end{array}$
\begin{flushright} \texttt{pcgSolver( Operator\underline{ }A, approxInverseOperator\underline{ }IA, Epsilon, MaxIter ); }
\end{flushright}
\texttt{pcgSolver.apply( b, x );} \end{quote}
\item default values: \\
\texttt{Epsilon = 1.e-16, \\ MaxIter = 50}
\end{itemize}

%%%%%%%%%%%%%%%%%%%%%%%%%%%%%%%%%%%%%%%%%%%%%%%%%%%%%%%%%%%%%%%%%%%%%%%%%%%%%%%

\subsection{ BiCGInverse:}
\begin{itemize}
\item derived from \texttt{InverseOp }in \texttt{aol::Solver.h}
\item BIorthogonal CG, for non symmetric, non singular matrices
\item using: \begin{quote}
\texttt{aol::BiCGInverse< VecType > }
\begin{flushright} \texttt{bicgSolver( Operator\underline{ }A, transposeOperator\underline{ }AT, Epsilon, MaxIter );} \end{flushright}
\texttt{bicgSolver.apply( b, x); } \end{quote}
\item default values: \\
\texttt{Epsilon = 1.e-16, \\ MaxIter = 50}
\end{itemize}

%%%%%%%%%%%%%%%%%%%%%%%%%%%%%%%%%%%%%%%%%%%%%%%%%%%%%%%%%%%%%%%%%%%%%%%%%%%%%%%

\subsection{ PBiCGSolver:}
\begin{itemize}
\item derived from \texttt{InverseOp }in \texttt{aol::Solver.h}
\item BiCG with preconditioning
\item using: \begin{quote}
\texttt{aol::PBiCGInverse< VecType > }
\begin{flushright} \texttt{pbicgSolver( Operator\underline{ }A, transposeOperator\underline{ }AT, approxInverseOperator\underline{ }IA, Epsilon, MaxIter );}
\end{flushright}
\texttt{pbicgSolver.apply( b, x );}\end{quote}
\item default values: \\
\texttt{Epsilon = 1.e-16, \\ MaxIter = 50}
\end{itemize}

%%%%%%%%%%%%%%%%%%%%%%%%%%%%%%%%%%%%%%%%%%%%%%%%%%%%%%%%%%%%%%%%%%%%%%%%%%%%%%%


\subsection{ PBiCGStabInverse: }
\begin{itemize}
\item derived from \texttt{InverseOp }in \texttt{aol::Solver.h}
\item variation of BiCG, with preconditioning $\Rightarrow$ more stable
\item using: \begin{quote}
\texttt{aol::PBiCGStabInverse< VecType > \\
pbicgstabSolver( Operator\underline{ }A, approxInverseOperator\underline{ }IA, Epsilon, MaxIter ); \\
pbicgstabSolver.apply( b, x ); }
\end{quote}
\item default values: \\
\texttt{Epsilon = 1.e-16, \\ MaxIter = 50}
\end{itemize}

%%%%%%%%%%%%%%%%%%%%%%%%%%%%%%%%%%%%%%%%%%%%%%%%%%%%%%%%%%%%%%%%%%%%%%%%%%%%%%%


\subsection{ GMRESInverse: }
\begin{itemize}
\item derived from \texttt{InverseOp }in \texttt{aol::Solver.h}
\item  Generalized Minimum RESidual method, for non symmetric, non singular matrices\\
can become unstable
\item using: \begin{quote}
\texttt{aol::GMRESInverse< VecType > }
\begin{flushright} \texttt{ gmresSolver( Operator\underline{ }A, Epsilon, MaxInnerIter, MaxIter );} \end{flushright}
\texttt{gmresSolver.apply( b, x );}
\end{quote}
\item default values: \\
\texttt{Epsilon = 1.e-16, \\ MaxInnerIter = 10, \\ MaxIter = 50}
\end{itemize}

%%%%%%%%%%%%%%%%%%%%%%%%%%%%%%%%%%%%%%%%%%%%%%%%%%%%%%%%%%%%%%%%%%%%%%%%%%%%%%%


\subsection{ GaussSeidelInverse: }
\begin{itemize}
\item derived from \texttt{InverseOp } in \texttt{aol::sparseSolver.h}
\item using: \begin{quote}
\texttt{GaussSeidelInverse< VecType, MatType > }
\begin{flushright} \texttt{gaussseidelSolver( Matrix\underline{ }A, Epsilon, MaxIter, Relaxation  );}
\end{flushright}
\texttt{gaussseidelSolver.apply( b, x );}
\end{quote}
\item \textbf{for matrices only!}
\item default values: \\
\texttt{Epsilon = 1.e-16, \\ MaxIter = 50, \\ Relaxation = 1.0}
\item \texttt{Relaxation $\not=$ 1 $\Rightarrow$ } SOR
\item \texttt{MatType} must support \texttt{makeRowEntries }
\end{itemize}

%%%%%%%%%%%%%%%%%%%%%%%%%%%%%%%%%%%%%%%%%%%%%%%%%%%%%%%%%%%%%%%%%%%%%%%%%%%%%%%

\subsection {comments}

solver derived from \texttt{InverseOp: }
\begin{itemize}
\item no guarantee for convergence with nonsymmetric operators
\item solver for \texttt{Vector< RealType>} AND \texttt{MultiVector< RealType> }
\item classes derived from \texttt{aol::Op< VecType >}:
\begin{itemize}
\item \texttt{aol::Matrix }
\item \texttt{aol::FEOp, aol::FEOpInterface }
\item \texttt{aol::LinCombOp, aol::CompositeOp, ... }
\end{itemize}
\item quiet-mode: default = false \\
change with method \texttt{setQuietMode( bool )} for CGInverse, PCGInverse
\item change iteration steps with method \texttt{changeMaxIterations} for CGInverse, PCGInverse
\end{itemize}


%%%%%%%%%%%%%%%%%%%%%%%%%%%%%%%%%%%%%%%%%%%%%%%%%%%%%%%%%%%%%%%%%%%%%%%%%%%%%%%

\subsection{JacobiSmoother: }
\begin{itemize}
\item derived from \texttt{Op< Vector< Realtype > > } in \texttt{aol::sparseSolver.h}
\item used for multigrid (pre- and postsmoothing)
\item \textbf{no abort condition on Epsilon! always computes \texttt{MaxIter} iteration steps}
\item using: \begin{quote}\texttt{JacobiSmoother< RealType, MatType > }
\begin{flushright} \texttt{jacobiSolver( Matrix\underline{ }A, Epsilon, MaxIter, Relaxation );} \end{flushright}
\texttt{jacobiSolver.apply( b, x );}
\end{quote}
\item default values: \\
\texttt{Epsilon = 1.e-16, \\ MaxIter = 50, \\ Relaxation = 1.0}
\item \texttt{Relaxation $\not=$ 1 $\Rightarrow$ } Richardson
\item \texttt{MatType} must support \texttt{makeRowEntries }
\item quiet-mode: default = true
\item for \texttt{Vector< RealType >} only
\end{itemize}


%%%%%%%%%%%%%%%%%%%%%%%%%%%%%%%%%%%%%%%%%%%%%%%%%%%%%%%%%%%%%%%%%%%%%%%%%%%%%%%

\chapter {Preconditioner}
\section{SSORPreconditioner: }
\begin{itemize}
\item derived from \texttt{Op< Vector< RealType > >} in \texttt{aol::sparseSolver.h}
\item for positive definite matrices
\item one step of ssor-iteration
\item using: \begin{quote}
\item \texttt{aol::SSORPreconditioner< RealType, MatType > ssorSolver( Matrix\underline{ }A, Omega );\\
ssorSolver.apply( b, x );}
\end{quote}
\item \texttt{MatType} must support \texttt{makeRowEntries }
\item default value \texttt{Omega = 1.2}
\end{itemize}

%%%%%%%%%%%%%%%%%%%%%%%%%%%%%%%%%%%%%%%%%%%%%%%%%%%%%%%%%%%%%%%%%%%%%%%%%%%%%%%


\section{ILU0Preconditioner: }
\begin{itemize}
\item derived from \texttt{Op< Vector< RealType > >} in \texttt{aol::sparseSolver.h}
\item Incomplete LU decomposition, without modification of non diagonal elements
\item using: \begin{quote}
\texttt{ILU0Preconditioner< RealType, MatType > ilu0Solver( MatrixA ); \\
ilu0Solver.apply( b, x );} \end{quote}
\item \texttt{MatType} must support \texttt{makeRowEntries} and \texttt{makeSortedRowEntries}
\end{itemize}

%%%%%%%%%%%%%%%%%%%%%%%%%%%%%%%%%%%%%%%%%%%%%%%%%%%%%%%%%%%%%%%%%%%%%%%%%%%%%%%

\section {Sparse matrices: }
\texttt{MatType} must support \texttt{makeRowEntries} and \texttt{makeSortedRowEntries} (for \texttt{ILU0Preconditioner})\\
$\Rightarrow$ use one from the following matrices ( or derived )
\begin{itemize}
\item in \texttt{sparseMartrices.h}:
\begin{itemize}
\item \texttt{aol::GenSparseMatrix< RealType > A( row, column);} or \\ \texttt{aol::GenSparseMatrix< RealType > A( Grid );}\\
public methods \texttt{A.set( i, j, value);} and \texttt{A.set( i, j, value);}
\item \texttt{aol::SparseMatrix< RealType >} derived from \texttt{aol::GenSparseMatrix}
\item \texttt{aol::UniformGridSparseMatrix< RealType >} derived from \texttt{aol::GenSparseMatrix}
\end{itemize}
\item in \texttt{qc::fastUniformGridMatrix.h}:
\begin{itemize}
\item \texttt{qc::FastUniformGridMatrix< RealType, dimension > A( Grid );} \\
public methods \texttt{A.set( i, j, value);} and \texttt{A.set( i, j, value);}\\
\textbf{NO} method \texttt{makeSortedRowEntries} $\Rightarrow$ \textbf{NOT} for \texttt{ILU0Preconditioner}!
\end{itemize}
\end{itemize}

%%%%%%%%%%%%%%%%%%%%%%%%%%%%%%%%%%%%%%%%%%%%%%%%%%%%%%%%%%%%%%%%%%%%%%%%%%%%%%%

\subsection{ example: heat equation }

\[ \partial_t u - \kappa \triangle u = f \]

weak formulation and time discretization (implicit euler):
\begin{eqnarray*} \int \frac{ \left(u^{i+1} - u^i\right) \phi}{\tau} & = & - \int \kappa \nabla u^{i+1} \nabla \phi + \int f \phi \\
\Rightarrow (M + \tau \kappa L) U^{i+1} & = & M U^i + \tau M F
\end{eqnarray*}

for one timestep we have to solve \qquad $ Ax = b $ \\ [1cm]
with $x = U^{i+1}, A = M + \tau \kappa L$ and $ b = M U^i + \tau M F$

for simplicity: $\kappa = 1, f \equiv 0 \quad \Rightarrow \quad (M + \tau L)U^{i+1} = M U^i$


%%%%%%%%%%%%%%%%%%%%%%%%%%%%%%%%%%%%%%%%%%%%%%%%%%%%%%%%%%%%%%%%%%%%%%%%%%%%%%%

%%%%%%%%%%%%%%%%%%%%%%%%%%%%%%%%%%%%%%%%%%%%%%%%%%%%%%%%%%%%%%%%%%%%%%%%%%%%%%%


%%% Local Variables:
%%% mode: latex
%%% TeX-master: "manual"
%%% End:


\InputIfFileExists{DocDTGrids.tex}{}{}

\end {document}

%%% Local Variables:
%%% mode: latex
%%% TeX-master:
%%% End:
