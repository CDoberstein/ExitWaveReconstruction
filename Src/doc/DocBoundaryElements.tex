
%%%%%%%%%%%%%%%%%%%%%%%%%%%%%%%%%%%%%%%%%%%%%%%%%%%%%%%%%%%%%%%%%%%%%%%%%%%%%%%%%%
%%%%%%%%%%%%%%%%%%%%%%%%%%%%%%%%%%%%%%%%%%%%%%%%%%%%%%%%%%%%%%%%%%%%%%%%%%%%%%%%%%
%%%%%%%%%%%%%%%%%%%%%%%%%%%%%%%%%%%%%%%%%%%%%%%%%%%%%%%%%%%%%%%%%%%%%%%%%%%%%%%%%%
%%%%%%%%%%%%%%%%%%%%%%%%%%%%%%%%%%%%%%%%%%%%%%%%%%%%%%%%%%%%%%%%%%%%%%%%%%%%%%%%%%


%                 Chapter "Boundary Element Methods"
%                 written by Martin Lenz


%%%%%%%%%%%%%%%%%%%%%%%%%%%%%%%%%%%%%%%%%%%%%%%%%%%%%%%%%%%%%%%%%%%%%%%%%%%%%%%%%%
%%%%%%%%%%%%%%%%%%%%%%%%%%%%%%%%%%%%%%%%%%%%%%%%%%%%%%%%%%%%%%%%%%%%%%%%%%%%%%%%%%
%%%%%%%%%%%%%%%%%%%%%%%%%%%%%%%%%%%%%%%%%%%%%%%%%%%%%%%%%%%%%%%%%%%%%%%%%%%%%%%%%%
%%%%%%%%%%%%%%%%%%%%%%%%%%%%%%%%%%%%%%%%%%%%%%%%%%%%%%%%%%%%%%%%%%%%%%%%%%%%%%%%%%



\chapter {Boundary Element Methods}

\section {Overview}

Framework for Boundary Element Methods in 2D, Namespace \lstinline$bm$

\begin {itemize}
\item Boundary Discretization\\
(framework, polygons, rectangles)
\item Boundary Integral Operators\\
(laplacian, elasticity)
\item Utilities
\end {itemize}

\section {Model Problem}

Consider a harmonic function $u$, i.e. $\Delta u = 0$ on $\Omega$.\\
Let $\psi_y (x) = -\frac{1}{2\pi }\ln \left| x-y\right|$ be the fundamental solution to the Laplacian in $\R^2$ centered at $y$.
Now we have for $y \in \partial \Omega$ the boundary integral equation ($\nu$ the outer normal)
\[ \frac {1}{2} u(y) = \int_{\partial \Omega} u(x) \partial_\nu \psi_y (x) - \partial_\nu u(x) \psi_y(x) \, ds_x \,, \]
that relates Dirichlet and Neumann boundary conditions.

This allows us to solve boundary value problems by considering an integral equation only on the boundary.

\section {Boundary Discretization}

Consider Integral Equations of the type
\[ (K u)(y) = f(y) \,, \qquad K u = \int_\Gamma u(x) \psi_y(x) \, ds_x \]
with some integral kernel $\psi_y$ that is usually nonlocal and
possibly singular at $y \in \Gamma$.

Ansatz: Approximate $u$ in some basis: $u (x) = \sum_j u_j \phi_j (x)$. Choose some points
$y_i \in \Gamma$, where we will assume the integral equation to be solved exactly. This yields
a linear system that is to be solved:
\[ \sum_j K_{ij} u_j = f(y_i) \,, \qquad K_{ij} = \int_\Gamma \phi_j (x) \psi_{y_i} (x) \, ds_x \]

\subsection {Abstract Framework}

For the target application, the boundary consists of numerous closed fragments.

So basic building block is a \lstinline$Particle$, i.e. a closed part of the boundary.

Particles are assumed to be polygons, that means one can iterate through the segments.

The set of all particles is organized in a list.



Particles have Iterators over segments\\
Barton--Nackman Interface, file \lstinline$particle.h$

\begin{myverbatim}
template <class Implementation, class IteratorType,
 class ConstIteratorType, class SegmentType,
 class ConstSegmentType, class DataType>
  class Particle {
  public:
    IteratorType beginSegment ();
    IteratorType endSegment ();
    ConstIteratorType beginSegment () const;
    ConstIteratorType endSegment () const;
    int getNumberOfSegments () const;
    void getLengths (Vector<DataType>& lengths) const;
    // ...
}
\end{myverbatim}



General segment Interface, file \lstinline$segment.h$

\begin{myverbatim}
template <class Implementation, class DataType>
  class Segment {
  public:
    aol::Vec2<DataType> getStart () const;
    aol::Vec2<DataType> getEnd () const;
    aol::Vec2<DataType> getDirection () const;
    aol::Vec2<DataType> getNormal () const;
    DataType getLength () const;
}
\end{myverbatim}



Usually one wants to store the real information in the particles.
Segments only contain a reference to a particle and an index. Most basic operations (including iterating
over them) are quite easy and efficient.

\begin{myverbatim}
template <class Implementation, class ParticleType>
  class ReferenceSegment
  : public Segment<Implementation,
   typename ParticleType::DataType> {
  public:
    typedef typename ParticleType::IndexType IndexType;
    bool operator == (const ReferenceSegment& segment) const;
    bool isFollowedBy (const ReferenceSegment& segment) const;
  protected:
    ParticleType* _particle;
    IndexType _index;
}
\end{myverbatim}



Boundary is a list of particles, file \lstinline$boundary.h$

\begin{myverbatim}
template <class ParticleType>
  class Boundary
  : public list<ParticleType> {
  public:
    typedef typename ParticleType::DataType DataType;
    AllSegmentIterator<ParticleType> beginSegment ();
    AllSegmentIterator<ParticleType> endSegment ();
    ConstAllSegmentIterator<ParticleType> beginSegment () const;
    ConstAllSegmentIterator<ParticleType> endSegment () const;
    int getNumberOfSegments () const;
    void getLengths (Vector<DataType>& lengths) const;
    // ...
}
\end{myverbatim}



The segment iterators on the boundary iterate over the particles, then over the segments
within. For this general concept of nested iterators, there is a utility class. Due to templatization, this is abstract
with respect to the const-ness of segments.

\begin{myverbatim}
template <class BoundaryType, class ParticleIteratorType,
 class SegmentIteratorType, class SegmentType>
  class PipingIterator {
  public:
    // Iterator interface

  protected:
    ParticleIteratorType _particleBegin,
                         _particleEnd, _particleCurrent;
    SegmentIteratorType _segmentBegin,
                        _segmentEnd, _segmentCurrent;
}
\end{myverbatim}



Example: Implementation of \lstinline$PipingIterator::operator++$

\begin{myverbatim}
  template <...>
    inline PipingIterator<...>& PipingIterator<...>::operator ++ ()
    {
      ++_segmentCurrent;
      if (_segmentCurrent == _segmentEnd) {
        ++_particleCurrent;
        updateSegment (false);
      }

      return *this;
    }
\end{myverbatim}



\begin{myverbatim}
  // Update segment iterators after particle changed
  template <...>
    inline void PipingIterator<...>::updateSegment (bool end)
    {
      if (_particleCurrent == _particleEnd) return;

      _segmentBegin = _particleCurrent->beginSegment ();
      _segmentEnd = _particleCurrent->endSegment ();
      _segmentCurrent = end ? _segmentEnd : _segmentBegin;
    }
\end{myverbatim}



Implementation of \lstinline$AllSegmentIterator$ and \lstinline$ConstAllSegmentIterator$
is now trivial with almost no code:

\begin{myverbatim}
template <class ParticleType>
  class AllSegmentIterator
  : public PipingIterator<Boundary<ParticleType>,
   typename Boundary<ParticleType>::iterator,
   typename ParticleType::SegmentIteratorType,
   typename ParticleType::SegmentType> {
  public:
    typedef typename ParticleType::SegmentType SegmentType;
    SegmentType& operator * () {
      return *_segmentCurrent;
    }
}
\end{myverbatim}

\subsection {General Implementation: Polygons}

The particle contains a list of points and
some interface functions. File \lstinline$parametric.h$

\begin{myverbatim}
template <class _DataType = double, class _IndexType = int>
  class ParaParticle
  : public Particle<ParaParticle<_DataType,_IndexType>,
   ParaSegmentIterator<_DataType,_IndexType>,
   ConstParaSegmentIterator<_DataType,_IndexType>,
   ParaSegment<_DataType,_IndexType>,
   ConstParaSegment<_DataType,_IndexType>,_DataType> {
   public:
     aol::Vec2<DataType>& getPoint (IndexType index);
   private:
     std::vector<aol::Vec2<DataType> > _points;
}
\end{myverbatim}



The \lstinline$ParaSegment$ class must contain the code for the
\lstinline$Segment$ interface using the list of points.

\begin{myverbatim}
template <class _DataType = double, class IndexType = int>
  class ParaSegment
  : public ReferenceSegment<ParaSegment<_DataType,IndexType>,
   ParaParticle<_DataType,IndexType> > {
  protected:
    void updatePoints ()
    {
      _start = _particle->getPoint (_index);
      _end = _particle->getPoint (_index + 1);
    }
}
\end{myverbatim}

\subsection {Space- and Time-Optimized Implementation: Rectangles}

A rectangle needs only to store two points. File \lstinline$rectangle.h$

\begin{myverbatim}
template <class _DataType = double>
  class RectParticle
  : public Particle<RectParticle<_DataType>,
   RectSegmentIterator<_DataType>,
   ConstRectSegmentIterator<_DataType>,
   RectSegment<_DataType>,
   ConstRectSegment<_DataType>,_DataType> {
  private:
    aol::Vec2<DataType> _lowerLeft, _upperRight;
}
\end{myverbatim}



\begin{myverbatim}
template <class DataType>
void RectSegment<DataType>::updatePoints () {
  aol::Vec2<DataType> ll = _particle->getLowerLeft ();
  aol::Vec2<DataType> ur = _particle->getUpperRight ();

  DataType top = ur.y (), left = ll.x ();
  DataType bottom = ll.y (), right = ur.x ();

  switch (_index) {
  case TOP:
    _start = aol::Vec2<DataType> (right, top);
    _end = aol::Vec2<DataType> (left, top);
    break;
  // ...
  }
}
\end{myverbatim}

\section {Boundary Integral Operators}

Discrete Integral operators are full matrices, constructed with some\\
\lstinline$LocalOperatorType$
that defines the evaluation of the integral of the kernel and some basis function:
\[ A_{ij} = \int_\Gamma \psi_{y_i} (x) \phi_j (x) \, ds_x \]

\begin{myverbatim}
template <class LocalOperatorType>
  class IntegralOperator
  : public aol::FullMatrix<typename LocalOperatorType
                           ::ParticleType::DataType> {
  IntegralOperator (const LocalOperatorType& localOp,
                    const Boundary<typename LocalOperatorType
                                   ::ParticleType>& boundary);
}
\end{myverbatim}

\subsection {Concrete Operators}

Our ansatz is based on integrating the kernel explicitly, so that we must provide
the kernel function itself and the analytical integrals needed for the matrix construction.
The basic local operator considers piecewise constant ansatz functions.

\begin{myverbatim}
template <class _ParticleType>
  class SingleLayerPotential
  : public LocalOperator<SingleLayerPotential<_ParticleType>,
                         _ParticleType> {

  public:
    typedef _ParticleType ParticleType;
    typedef typename ParticleType::DataType DataType;
    typedef typename ParticleType::ConstSegmentType
                                   ConstSegmentType;

    using LocalOperator<...>::evaluate;
    using LocalOperator<...>::evaluateLocally;

    DataType evaluate (aol::Vec2<DataType> point,
                       aol::Vec2<DataType> center) const;
    DataType evaluateLocally (const ConstSegmentType& integrate,
                              const aol::Vec2<DataType>& evaluate,
                              bool isstart, bool isinterior, bool isend) const;
  };
\end{myverbatim}

\lstinline$evaluate (p, c)$ return $\psi_c(p)$

\lstinline$evaluateLocally (i, e, ...)$ returns $\int_\Gamma \psi_{e} (x) \phi_i (x) \, ds_s$, where $\psi_e$ is centered at the middle of the segment $e$,
$\phi_i$ is the characteristic function of the segment $i$ (which then is obviously the integrational domain), and the bools have to be set appropriately
when the singularity of $\psi_e$ is somewhere on $i$.



There are other types of operators for piecewise linear ansatz functions and for different types of integral kernels.

\begin {itemize}

\item Single layer, double layer, and hypersingular operator

\item Laplacian and anisotropic elasticity

\item Piecewise constant and linear ansatz functions

\end {itemize}

\section {Utilities}

\subsection {Parameter Parser}

Parser for parameter files of the following type
\begin{myverbatim}
int     numerOfSteps  100
double  tau           1E-3
string  filename      data
bool    doit          1
\end{myverbatim}

Extensible to other types, by simply provinding an appropriate \lstinline$operator >>$
and adding one line to \lstinline$ParameterParser::read$
\begin{myverbatim}
map <string,TypePtr> types;
types["bool"]                  = &typeid (bool);
types["aol::Vector<double>"]   = &typeid (aol::Vector<double>);
\end{myverbatim}



Usage in main program is simple: With no runtime argument, the parser reads from
the default file \lstinline$parameter$, otherwise from the file with the provided filename.

\begin {myverbatim}
int main (int argc, char* argv[])
{
  aol::ParameterParser parameter (argc, argv);
  double tau;
  parameter.get ("tau", tau);
}
\end{myverbatim}

On initialization, the type of variables is identified by the string in the \lstinline$types$ map,
and the value is read to a string.
Then \lstinline$ParameterParser::get$ is overloaded for any type. It ensures that it is called exactly
for the right type by a \lstinline$typeid$ comparison and the converts the data via \lstinline$operator >>$.

\subsection {IO Utilities}

Functions for binary IO with streams\\
(encapsulates some ugly casting), file \lstinline$genmesh.h$

\begin{myverbatim}
//! Binary input from stream,
//! nothing is done about byte order
template <class DataType>
  void readbinary (istream& is, DataType& data);

//! Binary output to stream,
//! nothing is done about byte order
template <class DataType>
  void writebinary (ostream& os, const DataType& data);
\end{myverbatim}

\subsection {Memory Usage}

For space complexity estimates: memory usage of running program\\
(adds the right values of the \lstinline$mallinfo$ structure), file \lstinline$genmesh.h$

\begin{myverbatim}
//! Returns number of bytes used by program
int memusage ();
\end{myverbatim}

%%% Local Variables:
%%% mode: latex
%%% TeX-master: "manual"
%%% End:
