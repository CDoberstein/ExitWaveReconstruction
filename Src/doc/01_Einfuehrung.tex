\chapter{Einf"uhrung}

\section{Referenzen}

Mit welchen Begriffen sollte man freih"andig umgehen k"onnen, wenn man diese
Anleitung vollst"andig verstehen will? Welche B"ucher k"onnen dabei helfen?

In die Literaturlisten wird jedes Werk aufgenommen, das ein Mitglied des
QuocMesh-Kollektivs als hilfreich empfunden hat. Es sollte zu jedem Werk
Schwerpunkt und ggf. Begrenzungen innerhalb der hier genannten Schlagworte
genannt werden; jedes gelistete Werk sollte in der Mathematik-Bibliothek Bonn
verf"ugbar gemacht werden.

\subsection{Zur Mathematik}

\paragraph{Begriffe:}
Finite Elemente,
endlichdimensionale Approximation,
Approximationsr"aume,
lineare, multilineare und quadratische nodale Basisfunktionen,
Gau"s-Christoffel-Quadratur (ein- und mehrdimensional),
schwache Formulierung partieller Differentialgleichungen,
Variationsformulierung,
Matrix-Vektor-Formulierung linearer partieller Differentialgleichungen,
iterative L"oser f"ur lineare Gleichungssysteme (CG, Bi-CG, GMRES,
allesamt mit und ohne Pr"akonditionierung),
Multigrid-Verfahren,
Newton-Verfahren (klassisch, Quasi-Newton, vereinfachter Newton),
Gradientenabstieg,
Gradientenflu"s

\paragraph{Literatur:}
\begin{description}
\item[Peter Deuflhard, Andreas Hohmann:] Numerische Mathematik I. Berlin:
	deGruyter $^3 2003$.

	\textit{Klassische Stoffanordnung, sehr detailreich mit vielen
	Prim"arquellen. Inhalt: Lineare Gleichungssysteme, Fehleranalyse,
	Lin. Ausgleichsprobleme, Nichtlin. Gleichungssysteme, Lineare Eigenwertprobleme,
	Drei-Term-Rekursion (darin alleinstehend), Interpolation und
	Approximation, Iterative L"oser, Quadratur.
	Erste Wahl f"ur die Vertiefung der Grundlagen.}
%
\item[Martin Hanke-Bourgeois:] Grundlagen der Numerischen Mathematik und des
	Wissenschaftlichen Rechnens. Teubner $^2 2006$.

	\textit{Von Beginn der Numerischen Mathematik bis zu Finiten Elementen
	alles enthalten. Allerdings manches ungew"ohnlich sortiert und dargestellt,
	Orthogonalpolynome sehr in den Vordergrund ger"uckt. Weniger detailliert
	als Deuflhard, daf"ur ausgearbeitete Algorithmen in Pseudocode.
	Inhalt: Fehleranalyse, Lineare Gleichungssysteme, Ausgleichsrechnung,
	Nichtlineare Gleichungen, Eigenwerte, Interpolation, Orthogonalpolynome,
	Quadratur. Splines, Fourierreihen, Wavelets, Mathematische Modellierung,
	Gew"ohnliche Differentialgleichungen: Anfangs- und Randwertprobleme, Partielle
	Differentialgleichungen: Elliptische, parabolische, hyperbolische DGl
	einschlie"slich Mehrgitterverfahren und adaptiver Gitter.}
%
\end{description}

\subsection{Zur Programmiersprache C++}

\paragraph{Begriffe:}
Dynamische Speicherverwaltung,
Namensr"aume,
Ausnahmen (fangen und werfen),
RTTI,
STL (Strings, Streams, auto\_ptr
sowie s"amtliche Container: vector, list, set, hash\_set)
Definition und Deklaration,
statische Klassen-Member,
Operator"uberladung,
Vererbung,
Polymorphismus,
Liskov'sches Substitutionsprinzip,
virtuelle und rein virtuelle (abstrakte) Methoden,
Interface und Implementierung,
Templates,
Template-Spezialisierungen,

\paragraph{Literatur:}

\begin{description}
\item[Ulrich Breymann:] C++. Einf"uhrung und professionelle Programmierung.
	Hanser $^9 2007$.

	\textit{Sehr angenehm lesbare Einf"uhrung in C++. Inhalt:
	Variablen, Kontrollstrukturen, Funktionen, Klassen, Zeiger,
	Templates, Operatoren, Ausnahmen, Dateien und Streams.}
%
\item[Andr� Willms:] C++. Einstieg f"ur Anspruchsvolle. Pearson $2005$.

	\textit{Der Name ist sehr treffend. Guten C-Programmierern, die
	zwar den Sprachumfang von C++ kennen, aber noch kaum objektorientiert
	denken, wird viel "uber die OO-Denkweise nahegebracht. Daneben tausend
	ungeahnte Details aus dem Schatzk"astlein des C++-Standards, die (wie
	alles unn"utze Wissen) ihre Macht an v"ollig ungeahnter Stelle ausspielen.
	Inhalt: Grundlagen: Datentypen, Anweisungen, Deklarations vs. Definition,
	cv-Qualifizierung, Funktions-"uberladung. Klassen: Zugriffsrechte,
	Kon- und Destruktoren, Zeiger vs. Konstanten, statische Elemente,
	unvollendet konstruierte Objekte, verschachtelte Klassendefinitionen,
	der "`this"'-Zeiger, Singleton-Ent"-wurfsmuster.
	Dynamische Speicherverwaltung: Zeiger auf Felder, Funktionen, Klassenelemente,
	new und delete auf Rohspeicher, Allokatoren, auto\_ptr, new-Handler.
	Operator-"uberladung: Rechen- und kombinierte Operatoren, Vergleichs-,
	Shift-, Klammeroperatoren, Umwandlungsoperatoren, ++ und -\;-.
	Namensbereiche, Vererbung, Ausnahmen, Templates, Designfragen
	bzgl. Vererbung (Schnittstelle vs. Implementierung, verschiedene Paradigmen).
	Danach viele Beispiele zum OO-Design.}
%
\item[Helmut Herold, J"org Arndt:] C-Programmierung unter Linux, Unix,
	Windows. N"urnberg: SuSE Press $2004$.

	\textit{Alles, was wir je "uber C wissen wollten. Gro"se Ausf"uhrlichkeit,
	auch in den Standard-Konstrukten. Inhalt: Alles, was zu C und nicht zu
	C++ geh"ort (Beispiel: vollst"andige Beschreibung des \texttt{printf}-%
	Formatstrings, variable Argumentlisten u.~"a.), manchmal in
	verwirrender Reihenfolge. Macht nichts, da ausf"uhrliches Inhaltsverzeichnis
	und gutes Register.}
\end{description}

\paragraph{Zur Header/Implementierungs-Trennung bei Templates}

Es folgt nichts Verwunderliches, was nicht in anderen B"uchern schon st"unde.
Nur leider ist es trotzdem nicht jedem bekannt, also wiederholen wir's kurz.

Templatisierte Klassen m"ussen zur Compile-Zeit vollst"andig dargestellt werden
k"onnen. Das hei"st: Irgendwann, nur auf jeden Fall gleichzeitig, mu"s der
Compiler (a) die Implementierung einer templatisierten Klasse kennen, (b)
wissen, f"ur welche Template-Parameter er Code erzeugen und dann kompilieren soll.
Bei nicht-templatisierten Klassen ist letzteres kein Problem, also kann
die Implementierung einer Klasse (wie "ublich) in einer eigenen \texttt{cpp}-Datei
geschehen. Diese L"osung klappt f"ur templatisierte Klassen nicht ohne weiteres,
denn beim Kompilieren der Klassen-Implementation wei"s der Compiler
nicht unbedingt, f"ur welche Template-Parameter er Code erzeugen soll.
Es gibt aus dem Dilemma zwei "ubliche Auswege:

\textbf{Deklaration und Definition (Implementierung) gemeinsam:} Die Methoden der
	Klasse werden in der gleichen Datei (genauer: der gleichen Compile-Unit)
	implementiert, in der sie auch deklariert werden. Entweder direkt
	im Klassenrumpf:
	\begin{myverbatim}
	template <typename RealType>
	class Vector {
		RealType * _data;
		...
	public:
		RealType get ( int i ) const {
			return _data[i];
		}
	};
	\end{myverbatim}

	oder au"serhalb der Klassendeklaration:

	\begin{myverbatim}
	template <typename RealType>
	class Vector {
		RealType * _data;
		...
	public:
		RealType get ( int i ) const;
	};
	...
	template <typename RealType>
	RealType Vector<RealType>::get ( int i ) const {
		return _data[i];
	}
	\end{myverbatim}

\textbf{Definition aller ben"otigten Template-Instantiierungen innerhalb
	der implementierenden Unit:} Der Compiler m"u"ste ja gar nicht
	Code f"ur die gew"unschten Template-Parameter erzeugen, wenn er
	ihnen begegnet, sondern mu"s nur beim Linken sicher sein, da"s er
	den Code irgendwann einmal erzeugt hat. Warum nicht f"ur alle
	gew"unschten Template-Parameter an einem gemeinsamen Ort, n"amlich
	in der Unit, in der die Template-Klasse implementiert ist?

	Die Definition der zu erzeugenden Template-Instantiierungen
	gleicht der Syntax von Template-Spezialisierungen, nur da"s
	kein Rumpf folgt: Die Datei \texttt{vec.cpp} enth"alt u.~a.
	\begin{myverbatim}
	template class aol::Vector<int>;
	template class aol::Vector<double>;
	\end{myverbatim}
	In den "ubrigen Units, in denen der \id{aol::Vector}
	verwendet wird, d"urfen dann nur diejenigen Template-Parameter
	verwendet werden, f"ur die in \texttt{vec.cpp} Code
	erzeugt wurde.

Eine Mischung der beiden Varianten ist m"oglich: Man kann sowohl eine vollst"andige
Implementierung der Template-Klasse angeben, so da"s eine Unit, die diese
Datei includet, selbst Template-Instantiierungen vornehmen kann, und trotzdem
noch Template-Instanzen angeben, die auf jeden Fall erzeugt werden sollen.

Die erste Variante wird in den QuocMeshes immer verwendet, wenn nicht
abzusehen ist, f"ur welche Template-Parameter die Klasse wohl
erzeugt werden k"onnte, und auch stets dann, wenn die Un"ubersichtlichkeit,
Deklaration und Definition in einer Datei zu haben, noch handhabbar ist.
F"ur sehr gro"se, meist nur "uber \texttt{RealType} templatisierte Klassen
(beispielsweise Vektoren), wird zuweilen die zweite M"oglichkeit verwendet.
Bei Neuprogrammierung ist die erste Variante mit Trennung von
Deklaration und Implementierung, vorzuziehen (so geschehen beim
DTGrid-Code).

\subsection{Zu den verwendeten Tools}

\paragraph{Verwendete Programmierumgebungen:}
\begin{enumerate}
\item GCC kommandozeilenbasiert mit wechselnden Editoren\\Debugging mit ddd
\item Eclipse mit C++-Plugin, dahinter GCC\\Eingebaute Schnittstelle zu ddd
\item Visual Studio Express 2007\\Eingebauter Debugger
\item Dev-C++
\end{enumerate}
Dabei ist die Voll-Optimierung des GCC wirkungsvoller (der erzeugte
Code also schneller) als der Microsoft-Compiler.

\paragraph{Grape.}
Die Visualisierungssoftware Grape ist sehr weit ausgereift, wird allerdings
im Kern seit einigen Jahren nicht mehr fortentwickelt. Sie l"auft nur unter
Linux, unter Windows gibt es Darstellungsprobleme (permanentes Flackern).
Grape wird beim Compilieren des \texttt{makefile.selection.default} stets
erstellt.

\paragraph{vtkFOX.}
Das \textit{visualization toolkit} (vtk) ist an das FOX-Toolkit zur
betriebssystemunabh"angigen Erstellung graphischer Benutzeroberfl"achen
angedockt. Dadurch nutzen wir die Vorteile der einfachen GUI-Programmierung
mit dem FOX-Toolkit und die OpenGL-Unterst"utzung des vtk. Der
Funktionsumfang ist wesentlich geringer als von Grape, daf"ur
befindet sich vtkFOX unter aktiver Entwicklung (Stand M"arz 2008).

%%%%%%%%%%%%%%%%%%%%%%%%%%%%%%%%%%%%%%%%%%%%%%%%%%%%%%%%%%%%%%%%%%%%%%%%%%%%%%%
%%%%%%%%%%%%%%%%%%%%%%%%%%%%%%%%%%%%%%%%%%%%%%%%%%%%%%%%%%%%%%%%%%%%%%%%%%%%%%%
%%%%%%%%%%%%%%%%%%%%%%%%%%%%%%%%%%%%%%%%%%%%%%%%%%%%%%%%%%%%%%%%%%%%%%%%%%%%%%%

\section{Tricks und Verfahrensweisen}

\subsection{Der Barton-Nackman-Trick}

\subsection{Traits}

\subsection{Programmierstil und -Konventionen}

Hausarbeit erfordert eine straffe Organisation -- wenn sie effektiv sein
soll. Oberster Grundsatz hierbei hei"st: Erst denken, dann handeln.%
\footnote{Loriot: Pappa ante portas.}

% *******************************************************************************
%     File of conventions for programming in the quocmesh-library.
% *******************************************************************************

The following rules apply to all modules, examples and tools. You should observe them in projects, too.

\paragraph{class- and filenames}

\begin{itemize}
  \item class names: use \verb|CamelCase|, starting with upper case letter
  \item member function: use \verb|camelCase|, starting with lower case letter
  \item file name: use \verb|camelCase|, starting with lower case letter (exemption: start with upper case,
        if first word is a proper name, e. g. \verb|ArmijoSearch.h|), \\
        use only the following characters:
        \begin{itemize}
          \item upper and lower case letters (in particular no umlauts) (\texttt{a..z, A..Z})
          \item digits (\texttt{0..9})
          \item underscore and hyphen (\texttt{\_,-})
          \item periods/full stops (\texttt{.})
        \end{itemize}
  \item please use correct English names
  \item avoid name conflicts with system header files (e. g. stl headers)
  \item include each new module header file to corresponding selfTest
\end{itemize}


\paragraph{templates}

\begin{itemize}
  \item template parameters: must contain lowercase letters (\verb|realType| and \verb|RealType|
        are okay, but \verb|REAL| is not)
  \item Naming standard for re-exported template parameters: \\
        \verb|template< typename _DataType > ... typedef _DataType DataType;|
  \item For template parameters to be reexported: use \verb|_SomeType| as template parameter and
        \verb|public typedef _SomeType SomeType|
  \item If declaration and implementation are separate, the template parameters must have the
        same name in both cases, that is \verb|_SomeType|. In the implementation, both versions
        may be used.
\end{itemize}


\paragraph{preprocessor directives}

\begin{itemize}
    \item after \verb|#define|, USE CAPITAL LETTERS
          \begin{verbatim}
#ifndef __BLA_H
#define __BLA_H
// contents of bla.h
#endif         \end{verbatim}
    \item \#ifdef and similar preprocessor directives are not indented.
    \item include guards must be used in all headers, the format is \verb|__AOL_H|
    \item openmp critical sections should be named according to the scheme
\begin{verbatim}
namespace_class_{method,other useful identifier}[_number, if needed]
\end{verbatim}
      e.g.
\begin{verbatim}
#pragma omp critical (aol_RandomGenerator_getNextRnd)
\end{verbatim}
\end{itemize}


\paragraph{structure of externals}

\begin{itemize}
\item standard externals should contain
  \begin{itemize}
  \item \texttt{makefile.local} that sets include and link paths
  \item provide an include header, marked as a system header
\begin{verbatim}
#ifdef __GNUC__
#pragma GCC system_header
#endif
\end{verbatim}
    to prevent compiler warnings for external code
  \item a short description
  \end{itemize}
\item if the external is selected in
  \texttt{makefile.selection.default}, define
  \texttt{USE$\_$EXTERNAL$\_$...} is set automatically.
\item nonstandard externals may contain (small amounts of) code that
  is compiled automatically (if necessary) by the make mechanism
  (\texttt{go} and \texttt{clean} scripts) or by an appropriate visual
  c++ project
\end{itemize}

\paragraph{using externals}
\begin{itemize}
\item all modules (except for those obviously fully depending on an
  external) must compile without the external being used
\item Programs that use external code also have to compile if the
  corresponding external is switched off. The executables should then
  give a useful and informative error message like ''This program
  can't be used without (\textit{corresponding external})''. To achieve
  this, enclose your header and cpp-files in the following
  ifdef-construction (external is GRAPE in this example):
\begin{verbatim}
#ifdef USE_EXTERNAL_GRAPE
    ... (code that uses externals) ...
#endif
\end{verbatim}
In case of executables add the follwing else-part (or a similar one):
\begin{verbatim}
#ifdef USE_EXTERNAL_GRAPE
  ... (code that uses externals) ...
#else
  int main ( int, char** ) {
    cerr << "Without grape external, this program is useless" << endl;
    return ( 0 ) ;
  }
#endif
\end{verbatim}
\end{itemize}


\paragraph{style (indentation, spaces~\ldots)}

\begin{itemize}
    \item 2 spaces are used for indentation (no tabs, not 4 spaces etc.)
    \item preprocessor directives are not indented at all.
    \item \texttt{public:} and similar are not indented relative to the class.
          In both cases, that is the current astyle standard.
    \item brackets (placement in lines and spacing around brackets) are used according to the following scheme:
          \begin{verbatim}
dummy_method ( aol::Vector<RealType> &vec, RealType factor ) {
  for ( int i = 0; i < vec.size(); ++i ) {
    vec[i] = factor * vec[i];
  }
}              \end{verbatim}
          and can be enforced automatically by using util/indent which in turn uses astyle
\end{itemize}


\paragraph{Name convention for methods that import or export data, e. g. aol::Mat A, B inversion (same for transposition etc.)}

\begin{itemize}
  \item \texttt{void A.invert()}: writes $A = A^{-1}$
  \item \texttt{B = A.inverse()}: compute and return inverse of $A$, do not modify $A$
  \item \texttt{A.invertFrom(B)}: $A = B^{-1}$, $B$ unmodified
  \item \texttt{A.invertTo(B)}: $B = A^{-1}$, $A$ unmodified
\end{itemize}


\paragraph{miscellaneous}

\begin{itemize}
  \item comments have to be written in english (except in your own projects, there you can do whatever you want)
  \item use special characters only in your own projects and only on your own risk
\end{itemize}


\paragraph{data sets}

\begin{itemize}
  \item don't commit any data sets (images etc.) except very small data sets for examples or selfTests
        (keep those in directory \texttt{examples/testdata}, files here must be sufficiently free to be usable under quoc license)
\end{itemize}



\paragraph{No convention on~\ldots}

\begin{itemize}
  \item the position of member variables, they may be at the beginning or at the end of a class
  \item No general rule on whether implementation should be inside or outside class definition.
\end{itemize}


\paragraph{Rules for subversion}

\begin{itemize}
  \item Moving code and changing code (e. g. moving implementation out of class and changing it)
        should be committed separately to allow diffing.
  \item use svn:ignore to ignore files that will automatically occur when compiling and typical temp files of editors and IDEs,
        not for personal temporary copies like aol.hold
\end{itemize}


\paragraph{Very special things}
\begin{itemize}
  \item use \texttt{aol::Abs} instead of \texttt{fabs}.
        But: template specialization is necessary for unsigned data types when needed.
  \item Instead of \texttt{M$\_$PI} the expressions \texttt{aol::PI} or
        \texttt{aol::NumberTrait<RealType>::pi} should be used. Analogously for other mathematical
        constants (if not available define own NumberTrait).
  \item \texttt{apply(x, x)} is nowhere forbidden, but produces (mostly) unpredictable output.
        Apply should check for this and throw exception or contain comment that \texttt{apply(x, x)}
        works, we will not change this in all old apply methods now.
  \item Use \texttt{for}-loops where possible, even simple things like \\
        \texttt{a[0] = expression ( 0 ); a[1] = expression ( 1 )} \\
        should be done in a \texttt{for}-loop.
  \item In methods like \texttt{getMinValue()} or \texttt{getMaxValue()} don't initialize the
        first value with $\pm$ infinity, but with \texttt{vector[0]} (otherwise, if the size of
        the vector is $0$, it might happen that $\pm$ infinity is returned). \\
        If it's not really really obvious that in no case anything can ever go wrong with the
        \texttt{[]}-operator of the vector, use \texttt{get} and \texttt{set} (then bounds-checking
        is applied in the debug-mode).
  \item use \texttt{NON\_PARALLEL\_STATIC} if \texttt{static} variables should not be static when using
        parallelization (due to conflicting write access), e.g.\ if \texttt{static} is only used for
        performance reasons. If they always need to be static, prevent parallel write access.
\end{itemize}




%%%%%%%%%%%%%%%%%%%%%%%%%%%%%%%%%%%%%%%%%%%%%%%%%%%%%%%%%%%%%%%%%%%%%%%%%%%%%%%
%%%%%%%%%%%%%%%%%%%%%%%%%%%%%%%%%%%%%%%%%%%%%%%%%%%%%%%%%%%%%%%%%%%%%%%%%%%%%%%
%%%%%%%%%%%%%%%%%%%%%%%%%%%%%%%%%%%%%%%%%%%%%%%%%%%%%%%%%%%%%%%%%%%%%%%%%%%%%%%

\section{Praktische Entwicklung}

\anmerkung{Dieser Abschnitt soll alle Hinweise enthalten, die bei der Benutzung der
folgenden Tools in der Arbeitsgruppe zu beachten sind. Scheidelinie stets:
"`Gilt dieses oder jenes nur in Zusammenhang mit der QuocMeshBibl.?"' Sonst
nur Verweis auf Literatur bzw. das Wiki.}

\subsection{Subversion}

\anmerkung{Verweis aufs Wiki, Angabe mindestens einer ersch"opfenden Literaturquelle
zu Idee von und Umgang mit CVS.\\
Benimmregeln zum Einchecken: Keine nicht-winzigen Daten, nie ohne Kommentar,
kein Commit ohne Kompilieren der \textit{default}-Tests.}

\subsection{make}

\anmerkung {
		  Verweis auf eine gut lesbare makefile-Dokumentation\\
		  Aufbau, Speicherort,
		  Erzeugung, Aufgabe und Funktion der unterschiedlichen makefiles,\\
		  Aufgabe, Funktion, Unterschiede der Targets\\
		  Unterschiede bei Aufruf in verschiedenen Verzeichnissen.}

\subsection{Compiler, Debugging-Tools, Entwicklungsumgebungen}

\anmerkung{Womit programmieren wir, welche Erfahrung gibt es mit anderen
		  Umgebungen, Vor-/Nachteile der Compiler.}
