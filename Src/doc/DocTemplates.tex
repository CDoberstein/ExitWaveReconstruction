%%%%%%%%%%%%%%%%%%%%%%%%%%%%%%%%%%%%%%%%%%%%%%%%%%%%%%%%%%%%%%%%%%%%%%%%%%%%%%%%%%
%%%%%%%%%%%%%%%%%%%%%%%%%%%%%%%%%%%%%%%%%%%%%%%%%%%%%%%%%%%%%%%%%%%%%%%%%%%%%%%%%%
%%%%%%%%%%%%%%%%%%%%%%%%%%%%%%%%%%%%%%%%%%%%%%%%%%%%%%%%%%%%%%%%%%%%%%%%%%%%%%%%%%
%%%%%%%%%%%%%%%%%%%%%%%%%%%%%%%%%%%%%%%%%%%%%%%%%%%%%%%%%%%%%%%%%%%%%%%%%%%%%%%%%%


%                 Chapter Templates, Traits, Interfaces, STL
%                 written Marc Droske


%%%%%%%%%%%%%%%%%%%%%%%%%%%%%%%%%%%%%%%%%%%%%%%%%%%%%%%%%%%%%%%%%%%%%%%%%%%%%%%%%%
%%%%%%%%%%%%%%%%%%%%%%%%%%%%%%%%%%%%%%%%%%%%%%%%%%%%%%%%%%%%%%%%%%%%%%%%%%%%%%%%%%
%%%%%%%%%%%%%%%%%%%%%%%%%%%%%%%%%%%%%%%%%%%%%%%%%%%%%%%%%%%%%%%%%%%%%%%%%%%%%%%%%%
%%%%%%%%%%%%%%%%%%%%%%%%%%%%%%%%%%%%%%%%%%%%%%%%%%%%%%%%%%%%%%%%%%%%%%%%%%%%%%%%%%
\chapter{Templates, Traits, Interfaces, STL}

\section{Generic (OO) programming concepts}

\begin{itemize}
\item Avoid code duplication.
\item Represent independent concepts separately.
\item Design your classes to be used as flexibly as possible.
\item Problem: type flexibility, robustness \& efficiency
\item \bfseries{templates} provide a mechanism for \emph{generic programming}
\end{itemize}


\section{Class parametrization via templates}
Consider a simple class to story some \id{int}s:
\begin{myverbatim}
class int_array {
  private:
    int data[1000];
  public:
    int get( int i ) { ... }
    void set( int i ) { ... }
    // dot-product of this array and other array
    int operator*( const int_array &other ) const { ... }
};
\end{myverbatim}

Later, we want to have an array of \id{float}s, \id{double}s, \id{complex} numbers,
\id{short[3]}s or other types (which can be classes) and specify a different length.


Consider the following parametrization of class \id{array} with a type \id{T}:

\begin{myverbatim}
template <typename T>
class array {
  private:
    T data[1000];
  public:
    T get( int i ) const { ... }
    void set( int i, T v ) { ... }
    T operator*( const array<T> &other ) const {
      T dot=static_cast<T>(0);
      for ( int i=0; i<1000; i++ ) {
        dot += this->get( i ) * other.get( i );
      }
      return dot;
    }
};
\end{myverbatim}


\section{Instantiation}

Instantiation of templated class is very simple:
\begin{myverbatim}
array<float> a, b;
array<int> c, d;
[...] // fill the arrays with really nice values.
float dotf = a * b;
int doti = c * d;
\end{myverbatim}

\begin{itemize}
\item \id{array<A>} and \id{array<B>} are distinct types if and only if \id{A} and \id{B} are distinct types!
\item \id{array} itself is not a class, but just a template of a class, which has to be parametrized by the type \id{T}.
\end{itemize}



\section{Parametrization by values}

We now want the array to be able to store data of different sizes.
\begin{myverbatim}
template <typename T, int SIZE = 1000>
class array {
  private:
    T data[ SIZE ];
  public:
    [...]
    T operator*( const array<T,SIZE> &other ) const {
      T dot=static_cast<T>(0);
      for ( int i=0; i<SIZE; i++ ) {
        dot += this->get( i ) * other.get( i );
      }
      return dot;
    }
};
\end{myverbatim}

{\bfseries{Template arguments have to be known at compile time.}}

Again, the type of the templated class is determined
by it's own type and all of it's template arguments..

\begin{myverbatim}
array<float,1000> a;
array<float,1001> b;
array<double,1000> c;
float dot = a * A; // fails!
float dot = a * B; // fails!

typedef float MYTYPE;
array<MYTYPE> d;

float dot = a * d;
// works.. typedefs and default args are resolved
\end{myverbatim}



\section{(Member) Function Templates}
Allow functions to be passed parameters of various types:
\begin{myverbatim}
template <typename T>
T sqr( const T& v ) { return v*v; }

class A {
  [...]
  A operator*( const A& other ) const { [...]; }
};

A a;

sqr( 4. );
sqr( 2 );
sqr( a ); // hooray!
\end{myverbatim}

or we want to have a class to fill some arrays..
\begin{myverbatim}
template <class T, int SIZE = 1000>
class array {
  [...];
  T& operator[]( int i ) { return data[i]; }
  const T& operator[]( int i ) const { return data[i]; }
};

template <typename ARR_TYPE>
void fill_array( ARR_TYPE &a ) {
  a[0] = 2.;  a[1] = 3.;  a[2] = 5.;  a[3] = 7.;  a[4] = 11.;
}

double d[5];
array<float,5> f;
fill_array( d ); // works..
fill_array( f ); // works too.. ;)
\end{myverbatim}



\section{Template specializations}

Maybe we want to define a special behaviour of the class for
some special template arguments:

\begin{myverbatim}
template <int SIZE = 1000>
class array<bool> {
  private:
    int data[ (SIZE / sizeof(int)) + (SIZE % sizeof(int) == 0) ? 0 : 1 ];
  public:
    bool get( int i ) const { /* fancy bitwise storage goes here */ };
    void set( int i, bool T ) { [...] }
};
cerr << sizeof( array<float> ) << endl;
// 1000 * sizeof(float);

cerr << sizeof( array<bool> ) << endl;
// 1000 / sizeof(int);
\end{myverbatim}


\section{Traits} Use template specializations to map compile-time types or values to
other types or values.
\begin{myverbatim}
template <typename T> class avg_trait {  // default
public:
  typedef T AVG_TYPE;
};

template<> class avg_trait<int> {  // special behaviour
public:
  typedef float AVG_TYPE;
};

template <typename T, int SIZE = 1000>
class array_with_average : public array<T,SIZE> {
public:
  avg_trait<T>::AVG_TYPE average( ) const { [...]; }
};
\end{myverbatim}


\section{Type promoters with traits:}
\begin{myverbatim}
template<typename T1, typename T2>
Vector<???> operator+(const Vector<T1> &, Vector<T2> &);
\end{myverbatim}
How to determine the return value?
\begin{myverbatim}
template <typename T1, typename T2>
struct promote_trait { };  // empty..

template <> struct promote_trait<int,float> {
  typedef float T_promote;
};
template <> struct promote_trait<char,double> {
  typedef double T_promote;
};

template<typename T1, typename T2>
Vector<promote_trait<T1,T2>::T_promote>
operator+(const Vector<T1> &, Vector<T2> &);
\end{myverbatim}


\section{Definition of member functions outside classes:}
\begin{myverbatim}
template <typename T>
class A {
public:
  void foo( const T& ) const;
};

template <typename T>
void A<T>::foo( const T& t ) const {
  [...];
}
\end{myverbatim}
It is possible to define templated member functions in \id{.cpp} files,
but the compiler has to known for which types they have to be compiled.
\begin{myverbatim}
template class A<int>;
template class A<float>;
template class A<B>;
\end{myverbatim}
If a-priori list of template arguments is known, one should always define
member functions in the header files! \\
This also reduces the size of compiled object files, precompilation into a lib
is not possible though.


\section{Interfaces: Static vs. dynamic Polymorphism}

Motivation
\begin{enumerate}
\item huge variety of libraries and special classes, each having
advantages and disadvantages in different environments
\item simplify problem-related programming by the spefication
  of a high-level interface. the programmer who is implementing
actual problems only has to know the interface, not the details
about the different libraries
\item Usually there is a significant loss of efficiency due to the interface,
 due to wrapper functions, proxy-objects and structural differences.
\end{enumerate}

How can interfaces be designed efficiently and still be easy to use?
\textbf{Interfaces: Static vs. dynamic Polymorphism} \\
(cf. T. Veldhuizen, {\em Techniques for Scientific C++})\\
{Dynamic Polymorphism:}
\begin{myverbatim}
template <typename T> class Matrix { //
public:
  virtual T operator( int i, int j ) const = 0 { };
  virtual T frobeniusNorm( ) const { [..] // call get a lot of times; }
  virtual T mult( const Vector<T> &arg, Vector<T> &dest );
};

template <typename T> class FullMatrix {
public:
  virtual T operator( int i, int j ) const { [...] };
};

template <typename T> class SymmetricMatrix {
public:
  virtual T operator( int i, int j ) const { [...] };
};
\end{myverbatim}

Now we want to use the matrices dynamically..
\begin{myverbatim}
Matrix *matrices[5];
matrices[0] = new FullMatrix<double>;
matrices[1] = new SymmetricMatrix<double>;
// etc.

double sum_frob = 0;
for (int i=0; i<5; i++ ) {
  sum_frob += matrices[i]->frobeniusNorm();
}
\end{myverbatim}

{This will be slow!!} the virtual function \id{operator( int i, int j )} will be called
often, each time a pointer lookup is necessary. \\
Resort: Implement the (virtual) function \id{frobeniusNorm} on all derived classes,
but this involves to implement a lot of similar code..

\section{Static polymorphism:} \textbf{engines}
\begin{myverbatim}
template <typename T>
class FullMatrix_engine { // storage/get/set };
template <typename T>
class SymmetricMatrix_engine { // storage/get/set };

template <T_engine>
class Matrix {
private:
  T_engine engine;
public:
  [...]
  T frobeniusNorm() const {
    T r=0;
    for ( ... ) { r += sqr( engine( i, j ) ); }
  }
};
\end{myverbatim}

\section{Static polymorphism: } \textbf{The Barton and Nackman trick}
\begin{myverbatim}
// we still want to put matrices into a container
template <typename Imp>
class A_Interface {
protected:
  Imp &asImp() { return static_cast<Imp&>( *this ); }
  const Imp &asImp() const { return static_cast<const Imp&>( *this ); }
public:
  void foo( ) { asImp().foo(); }

  void bar( ) { foo(); foo(); }
};

class A : public A_Interface<A> {
public:
  void foo( ) { // do something }
};
\end{myverbatim}


\section{Static polymorphism: } \textbf{The Barton and Nackman trick}
{\small
\begin{myverbatim}
// we still want to put matrices into a container
template <typename T, typename Imp>
class MatrixInterface {
protected:
  Imp &asImp() { return static_cast<Imp&>( *this ); }
  const Imp &asImp() const { return static_cast<const Imp&>( *this ); }
public:
  T operator( int i, int j ) const { return asImp()( i, j ); }
  T frobeniusNorm( ) const { return asImp().frobeniusNorm(); }
};

template <typename T, typename Imp>
class MatrixDefault : MatrixInterface<T, Imp>{
  T frobeniusNorm( ) const {
    T r=0;
    for ( ... ) { r += sqr((*this)( i, j ))); }
  }
};
\end{myverbatim}

Actual implementations:
\begin{myverbatim}
template <typename T>
class FullMatrix : public MatrixDefault<T,FullMatrix<T> > {
  T data[1000][1000];
public:
  T operator( int i, int j ) const { return data[i][j]; }
};

template <typename T>
class SymmetricMatrix : public MatrixDefault<T,FullMatrix<T> > {
public:
  T operator( int i, int j ) const { [..] }
  T frobeniusNorm( ) const { [..] // do some tricks here.. ; }
};

\end{myverbatim}
}


\section{Advantages}
\begin{enumerate}
\item Interface is clearly defined and more transparent.
\item No instanciations of auxiliary classes necessary.
\item Allows more conventional inheritance hierarchy.
\item Methods can be selectively overloaded in derived classes, i.~e. it is easy to
implement and change standard behaviour in a larger hierarchy.
\end{enumerate}

\textbf{Warning: Typical error} If the member function of the derived class is not exactly of the
same type (e.~g. forgot to declare as \id{const}), the call to
\id{asImp().function()} get's lost in infinite recursions..


\subsection{Rules of thumb}

\begin{itemize}
\item \id{virtual} functions \textbf{can} be slow, but the difference to inlined functions is miniscule. {When the function body is longer than a few lines, the difference may not be noticed.} Then dynamic polymorphism should be chosen over templated polymorphism for the sake of maintainability and lower compilation time.
\item What parts can I implement with static polymorphism and which not? \\
Static polymorphism has it's limitations. Everything that is known to you, the programmer, when you write the program, i.~e., at the stage of compile time, can be implemented with static polymorphism. Not more!
\end{itemize}


\section{The standard template library (STL)}


\begin{itemize}
\item Library of classes, algorithms and iterators for the generic storage and processing of instances.
\item It is entirely parametrizable by templates.
\item The heart of the STL are container classes, i.~e., classes to dynamically manage sets of objects.
\end{itemize}
Example:
\begin{myverbatim}
// create a vector of 1000 integers
vector<int> vec_of_ints( 1000 );
vec_of_ints[0] = 2;
vec_of_ints[1] = 3;
vec_of_ints[2] = vec_of_ints[0] + vec_of_ints[1];
\end{myverbatim}


\subsection{STL concepts}
\begin{itemize}
\item \textbf{storage classes/containers}. The main container classes in the STL are \id{vector}, \id{list}, \id{deque}, \id{set}, \id{map}, \id{hash\_set}, \id{hash\_map}, \id{stack}, etc. They are parametrized by the type of objects they should store and possibly by a memory manager.
\item \textbf{iterators} provide easy to use functional for the sequential traversal of the elements of a container or a subset. Iterators represent the method of choice for accessing elements of containers.
\item \textbf{algorithms}. The STL contains a variety of auxiliary algorithms which are often based on iterators, such as sorting routines, search, reversal, merging, splicing, heap operations, filling, shuffling etc. etc. The STL is considered state-of-the-art with respect to efficiency and flexibility and should be used wherever it is applicable.
\end{itemize}


\centerline{\large Overview on container classes}
\begin{itemize}
\item \textbf{vectors} represent linearly organized sequences of objects, which provides (unlike most others) \textbf{random access} to it's elements.
\item \textbf{lists} are doubly-linked lists, i.~e., access to the predecessor and successor, as well as insertion at arbitrary positions is an O(1) operation.
\item \textbf{deque} is like a vector but supports constant time insertion at the beginning.
\item \textbf{sets} represent the unstructered storage of (unique) elements. It is well suited for set operations, like $\cap$, $\cup$, $\setminus$. Sets are always sorted.
\item \textbf{maps} are parametrized by the types \id{Key} and \id{Data}, which can be understood as domain resp. range-types of a mathematical mapping. Keys are unique.
\item \textbf{stacks} provide a {\em last-in-first-out} (LIFO) in constant time class.
\item \textbf{hash\_sets} and \textbf{hash\_maps} are principally like \textbf{sets}, \textbf{maps}, but thanks to the additional specification of a hashing-function,
access to elements is in constant time (depending on the choice of the hash-function).

\item \textbf{(hash\_)multisets} and \textbf{(hash\_)multimaps} differ from the non-multi variants by the fact that they can contain multiple elements being equal, i.~e., elements are mapped to subsets of the range set, not atoms.

\item \textbf{string} classes.

\end{itemize}


\centerline{\large Iterators}

Sequential traversal of half-open intervals: \id{[begin,.........,end)}
\begin{itemize}
\item \textbf{Input iterators} read-access, but not necessarily write-access.
\item \textbf{Ouput iterators} write-access, but not necessarily read-access.
\item \textbf{Forward iterators} traversal from begin to end
\item \textbf{Reverse iterators} traversal in reverse order
\item \textbf{Bidirectonal iterators} traversal in both directions possible.
\item \textbf{Random-Access iterators} also allow index arithmetic.

\end{itemize}


\subsection{list-example}

\begin{myverbatim}
template <typename T_cont>
void dump( const T_cont& c ) {
  for ( typename T_cont::const_iterator it=c.begin();
    it!=c.end(); ++it ) { cerr << *it << " "; }
  cerr << endl;
}

list<int> l;
l.push_back( 8 ); l.push_back( 2 ); l.push_back( 6 );
dump( l );
list<int>::iterator it=l.begin();
l.insert( ++++it, 1 ); dump( l );
l.sort( );             dump( l );
Output:
8 2 6
8 2 1 6
1 2 6 8
\end{myverbatim}

\subsection{map-example}
{\small
\begin{myverbatim}
map<const char*, int, ltstr> months;
months["january"] = 31;
[...]
months["december"] = 31;

cout << "june -> " << months["june"] << endl;
map<const char*, int, ltstr>::iterator cur  = months.find("june");
map<const char*, int, ltstr>::iterator prev = cur;
map<const char*, int, ltstr>::iterator next = cur;
++next;
--prev;
cout << "Previous (alphabetically) is " << (*prev).first << endl;
cout << "Next (alphabetically) is " << (*next).first << endl;
\end{myverbatim}
Output:
\begin{myverbatim}
june -> 30
Previous (alphabetically) is july
Next (alphabetically) is march
\end{myverbatim}
}


\subsection{Memory management}

The STL has efficient built-in memory-management routines, which can
be customized (rarely-necessary).

Actually reserved memory may be larger than the amount needed by
the stored elements.
\id{vector} provides various member functions.
\begin{itemize}
\item \id{size( )} returns actual length of vector.
\item \id{capacity( )} returns how many elements fit into the reserved
memory. larger or equal than size.
\item \id{reserve( size\_t n )} ensures there is enough memory for \id{n} elements.
\item \id{resize( n, T t=T() )} resizes the vector such that length becomes \id{n}
\end{itemize}


%%% Local Variables:
%%% mode: latex
%%% TeX-master: "manual"
%%% End:
