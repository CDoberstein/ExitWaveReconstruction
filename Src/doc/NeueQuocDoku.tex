%   Diese Dokumentation soll Stueck fuer Stueck aufgebaut werden, um die fruehere
%   manual.tex irgendwann ersetzen zu koennen.
%
%   Das Unternehmen wurde begonnen im Jahre des Herrn 2008, im dritten Jahr
%   des Pontifikats Benedikts XVI. Moege eine glueckliche Entwicklung dieser
%   unserer Erde ihm eine glueckliche Zukunft schenken.
%
%
% General properties
\documentclass [twoside, a4paper, 12pt] {scrbook}
\addtokomafont{paragraph}{\rmfamily}

\usepackage {graphicx}

\usepackage {amssymb, latexsym}
\usepackage {amsmath}
\usepackage {color}
\usepackage {dsfont}
%\usepackage [english] {babel}
\usepackage{german}
\usepackage[latin1]{inputenc}
\usepackage {times}

\usepackage{booktabs}
\usepackage{longtable}
\usepackage{enumerate}

\usepackage {listings}
\lstset {language=C++, tabsize=2,
         keywordstyle=\color{rot}\bfseries,
         basicstyle=\ttfamily\small,
         commentstyle=\rmfamily\itshape,
         aboveskip=1cm,
         numbers=left, numberstyle=\tiny, stepnumber=1, numbersep=0.5cm,
         emph={Vector, Vec2, aol, qc, bm,
               list,
               std, vector,string, map,istream, ostream, Op, FILE, String, quoc,
               Exception,},
         emphstyle=\color{blau}\bfseries,
         emph={[2]IndexType, RangeType, DomainType,
               DataType, IndexType,},
         emphstyle={[2]\color{gruen}\bfseries}}

% Page layout
\typearea [12mm] {13}
\pagestyle {headings}
\frenchspacing
\setlength {\parindent} {0mm}

\definecolor {rot} {rgb} {0.4, 0, 0}
\definecolor {blau} {rgb} {0, 0, 0.4}
\definecolor {gruen} {rgb} {0, 0.4, 0}

\newcommand		\R			{\mathds {R}}
\newcommand		\id [1]		{{\lstinline$#1$}}
\renewcommand	\div		{\mathrm{div}}
\newcommand		\missing [1]{\textbf{[missing:}{#1} \textbf{]}}
\newcommand		\dx			{\,\mathrm{d}x}
\newcommand		\V			{{\mathcal V}}
\renewcommand	\phi		{\varphi}

\newcommand \cls [1] {\framebox(3,1.5){#1}}
\newcommand \clsklein [1] {\framebox(3,1){#1}}
\setlength{\unitlength}{8mm}

\DeclareMathOperator \Span {span}

\lstnewenvironment {myverbatim} {} {}

\newcommand \anmerkung [1] {{\bfseries \color{red} [#1]}}

\begin {document}


\chapter*{Vorwort und Gliederung}
% Autor Stefan W. von Deylen

Die QuocMesh-Bibliothek ist das Werk derzeitiger und ehemaliger Mitarbeiter
der Arbeitsgruppe Prof. Martin Rumpf, Bonn (fr"uher Duisburg). Sie ist eine
Klassenbibliothek zur Numerischen Simulation mithilfe Finiter Elemente,
vor allem auf 2D- und 3D-Gittern mit quadratischen bzw. w"urfelf"ormigen
Elementen (\textbf{Qu}adtrees bzw. \textbf{Oc}trees, daher der Name).

Die Mitarbeiter an der Bibliothek werden im folgenden als "`das QuocMesh-%
Kollektiv"' referenziert.

Diese Anleitung versucht, gleichzeitig vielen Herren zu dienen und alles in
einem zu sein: Einf"uhrung, Nachschlagewerk, Kurzreferenz f"ur die wichtigsten
Funktionen. F"ur einen Ersteinstieg ist es daher nicht empfehlenswert,
ganz vorn anzufangen. Hierf"ur ist das Kapitel \ref{sec:Schnelleinstieg} gedacht.

\tableofcontents

\chapter{Einf"uhrung}

\section{Referenzen}

Mit welchen Begriffen sollte man freih"andig umgehen k"onnen, wenn man diese
Anleitung vollst"andig verstehen will? Welche B"ucher k"onnen dabei helfen?

In die Literaturlisten wird jedes Werk aufgenommen, das ein Mitglied des
QuocMesh-Kollektivs als hilfreich empfunden hat. Es sollte zu jedem Werk
Schwerpunkt und ggf. Begrenzungen innerhalb der hier genannten Schlagworte
genannt werden; jedes gelistete Werk sollte in der Mathematik-Bibliothek Bonn
verf"ugbar gemacht werden.

\subsection{Zur Mathematik}

\paragraph{Begriffe:}
Finite Elemente,
endlichdimensionale Approximation,
Approximationsr"aume,
lineare, multilineare und quadratische nodale Basisfunktionen,
Gau"s-Christoffel-Quadratur (ein- und mehrdimensional),
schwache Formulierung partieller Differentialgleichungen,
Variationsformulierung,
Matrix-Vektor-Formulierung linearer partieller Differentialgleichungen,
iterative L"oser f"ur lineare Gleichungssysteme (CG, Bi-CG, GMRES,
allesamt mit und ohne Pr"akonditionierung),
Multigrid-Verfahren,
Newton-Verfahren (klassisch, Quasi-Newton, vereinfachter Newton),
Gradientenabstieg,
Gradientenflu"s

\paragraph{Literatur:}
\begin{description}
\item[Peter Deuflhard, Andreas Hohmann:] Numerische Mathematik I. Berlin:
	deGruyter $^3 2003$.

	\textit{Klassische Stoffanordnung, sehr detailreich mit vielen
	Prim"arquellen. Inhalt: Lineare Gleichungssysteme, Fehleranalyse,
	Lin. Ausgleichsprobleme, Nichtlin. Gleichungssysteme, Lineare Eigenwertprobleme,
	Drei-Term-Rekursion (darin alleinstehend), Interpolation und
	Approximation, Iterative L"oser, Quadratur.
	Erste Wahl f"ur die Vertiefung der Grundlagen.}
%
\item[Martin Hanke-Bourgeois:] Grundlagen der Numerischen Mathematik und des
	Wissenschaftlichen Rechnens. Teubner $^2 2006$.

	\textit{Von Beginn der Numerischen Mathematik bis zu Finiten Elementen
	alles enthalten. Allerdings manches ungew"ohnlich sortiert und dargestellt,
	Orthogonalpolynome sehr in den Vordergrund ger"uckt. Weniger detailliert
	als Deuflhard, daf"ur ausgearbeitete Algorithmen in Pseudocode.
	Inhalt: Fehleranalyse, Lineare Gleichungssysteme, Ausgleichsrechnung,
	Nichtlineare Gleichungen, Eigenwerte, Interpolation, Orthogonalpolynome,
	Quadratur. Splines, Fourierreihen, Wavelets, Mathematische Modellierung,
	Gew"ohnliche Differentialgleichungen: Anfangs- und Randwertprobleme, Partielle
	Differentialgleichungen: Elliptische, parabolische, hyperbolische DGl
	einschlie"slich Mehrgitterverfahren und adaptiver Gitter.}
%
\end{description}

\subsection{Zur Programmiersprache C++}

\paragraph{Begriffe:}
Dynamische Speicherverwaltung,
Namensr"aume,
Ausnahmen (fangen und werfen),
RTTI,
STL (Strings, Streams, auto\_ptr
sowie s"amtliche Container: vector, list, set, hash\_set)
Definition und Deklaration,
statische Klassen-Member,
Operator"uberladung,
Vererbung,
Polymorphismus,
Liskov'sches Substitutionsprinzip,
virtuelle und rein virtuelle (abstrakte) Methoden,
Interface und Implementierung,
Templates,
Template-Spezialisierungen,

\paragraph{Literatur:}

\begin{description}
\item[Ulrich Breymann:] C++. Einf"uhrung und professionelle Programmierung.
	Hanser $^9 2007$.

	\textit{Sehr angenehm lesbare Einf"uhrung in C++. Inhalt:
	Variablen, Kontrollstrukturen, Funktionen, Klassen, Zeiger,
	Templates, Operatoren, Ausnahmen, Dateien und Streams.}
%
\item[Andr� Willms:] C++. Einstieg f"ur Anspruchsvolle. Pearson $2005$.

	\textit{Der Name ist sehr treffend. Guten C-Programmierern, die
	zwar den Sprachumfang von C++ kennen, aber noch kaum objektorientiert
	denken, wird viel "uber die OO-Denkweise nahegebracht. Daneben tausend
	ungeahnte Details aus dem Schatzk"astlein des C++-Standards, die (wie
	alles unn"utze Wissen) ihre Macht an v"ollig ungeahnter Stelle ausspielen.
	Inhalt: Grundlagen: Datentypen, Anweisungen, Deklarations vs. Definition,
	cv-Qualifizierung, Funktions-"uberladung. Klassen: Zugriffsrechte,
	Kon- und Destruktoren, Zeiger vs. Konstanten, statische Elemente,
	unvollendet konstruierte Objekte, verschachtelte Klassendefinitionen,
	der "`this"'-Zeiger, Singleton-Ent"-wurfsmuster.
	Dynamische Speicherverwaltung: Zeiger auf Felder, Funktionen, Klassenelemente,
	new und delete auf Rohspeicher, Allokatoren, auto\_ptr, new-Handler.
	Operator-"uberladung: Rechen- und kombinierte Operatoren, Vergleichs-,
	Shift-, Klammeroperatoren, Umwandlungsoperatoren, ++ und -\;-.
	Namensbereiche, Vererbung, Ausnahmen, Templates, Designfragen
	bzgl. Vererbung (Schnittstelle vs. Implementierung, verschiedene Paradigmen).
	Danach viele Beispiele zum OO-Design.}
%
\item[Helmut Herold, J"org Arndt:] C-Programmierung unter Linux, Unix,
	Windows. N"urnberg: SuSE Press $2004$.

	\textit{Alles, was wir je "uber C wissen wollten. Gro"se Ausf"uhrlichkeit,
	auch in den Standard-Konstrukten. Inhalt: Alles, was zu C und nicht zu
	C++ geh"ort (Beispiel: vollst"andige Beschreibung des \texttt{printf}-%
	Formatstrings, variable Argumentlisten u.~"a.), manchmal in
	verwirrender Reihenfolge. Macht nichts, da ausf"uhrliches Inhaltsverzeichnis
	und gutes Register.}
\end{description}

\paragraph{Zur Header/Implementierungs-Trennung bei Templates}

Es folgt nichts Verwunderliches, was nicht in anderen B"uchern schon st"unde.
Nur leider ist es trotzdem nicht jedem bekannt, also wiederholen wir's kurz.

Templatisierte Klassen m"ussen zur Compile-Zeit vollst"andig dargestellt werden
k"onnen. Das hei"st: Irgendwann, nur auf jeden Fall gleichzeitig, mu"s der
Compiler (a) die Implementierung einer templatisierten Klasse kennen, (b)
wissen, f"ur welche Template-Parameter er Code erzeugen und dann kompilieren soll.
Bei nicht-templatisierten Klassen ist letzteres kein Problem, also kann
die Implementierung einer Klasse (wie "ublich) in einer eigenen \texttt{cpp}-Datei
geschehen. Diese L"osung klappt f"ur templatisierte Klassen nicht ohne weiteres,
denn beim Kompilieren der Klassen-Implementation wei"s der Compiler
nicht unbedingt, f"ur welche Template-Parameter er Code erzeugen soll.
Es gibt aus dem Dilemma zwei "ubliche Auswege:

\textbf{Deklaration und Definition (Implementierung) gemeinsam:} Die Methoden der
	Klasse werden in der gleichen Datei (genauer: der gleichen Compile-Unit)
	implementiert, in der sie auch deklariert werden. Entweder direkt
	im Klassenrumpf:
	\begin{myverbatim}
	template <typename RealType>
	class Vector {
		RealType * _data;
		...
	public:
		RealType get ( int i ) const {
			return _data[i];
		}
	};
	\end{myverbatim}

	oder au"serhalb der Klassendeklaration:

	\begin{myverbatim}
	template <typename RealType>
	class Vector {
		RealType * _data;
		...
	public:
		RealType get ( int i ) const;
	};
	...
	template <typename RealType>
	RealType Vector<RealType>::get ( int i ) const {
		return _data[i];
	}
	\end{myverbatim}

\textbf{Definition aller ben"otigten Template-Instantiierungen innerhalb
	der implementierenden Unit:} Der Compiler m"u"ste ja gar nicht
	Code f"ur die gew"unschten Template-Parameter erzeugen, wenn er
	ihnen begegnet, sondern mu"s nur beim Linken sicher sein, da"s er
	den Code irgendwann einmal erzeugt hat. Warum nicht f"ur alle
	gew"unschten Template-Parameter an einem gemeinsamen Ort, n"amlich
	in der Unit, in der die Template-Klasse implementiert ist?

	Die Definition der zu erzeugenden Template-Instantiierungen
	gleicht der Syntax von Template-Spezialisierungen, nur da"s
	kein Rumpf folgt: Die Datei \texttt{vec.cpp} enth"alt u.~a.
	\begin{myverbatim}
	template class aol::Vector<int>;
	template class aol::Vector<double>;
	\end{myverbatim}
	In den "ubrigen Units, in denen der \id{aol::Vector}
	verwendet wird, d"urfen dann nur diejenigen Template-Parameter
	verwendet werden, f"ur die in \texttt{vec.cpp} Code
	erzeugt wurde.

Eine Mischung der beiden Varianten ist m"oglich: Man kann sowohl eine vollst"andige
Implementierung der Template-Klasse angeben, so da"s eine Unit, die diese
Datei includet, selbst Template-Instantiierungen vornehmen kann, und trotzdem
noch Template-Instanzen angeben, die auf jeden Fall erzeugt werden sollen.

Die erste Variante wird in den QuocMeshes immer verwendet, wenn nicht
abzusehen ist, f"ur welche Template-Parameter die Klasse wohl
erzeugt werden k"onnte, und auch stets dann, wenn die Un"ubersichtlichkeit,
Deklaration und Definition in einer Datei zu haben, noch handhabbar ist.
F"ur sehr gro"se, meist nur "uber \texttt{RealType} templatisierte Klassen
(beispielsweise Vektoren), wird zuweilen die zweite M"oglichkeit verwendet.
Bei Neuprogrammierung ist die erste Variante mit Trennung von
Deklaration und Implementierung, vorzuziehen (so geschehen beim
DTGrid-Code).

\subsection{Zu den verwendeten Tools}

\paragraph{Verwendete Programmierumgebungen:}
\begin{enumerate}
\item GCC kommandozeilenbasiert mit wechselnden Editoren\\Debugging mit ddd
\item Eclipse mit C++-Plugin, dahinter GCC\\Eingebaute Schnittstelle zu ddd
\item Visual Studio Express 2007\\Eingebauter Debugger
\item Dev-C++
\end{enumerate}
Dabei ist die Voll-Optimierung des GCC wirkungsvoller (der erzeugte
Code also schneller) als der Microsoft-Compiler.

\paragraph{Grape.}
Die Visualisierungssoftware Grape ist sehr weit ausgereift, wird allerdings
im Kern seit einigen Jahren nicht mehr fortentwickelt. Sie l"auft nur unter
Linux, unter Windows gibt es Darstellungsprobleme (permanentes Flackern).
Grape wird beim Compilieren des \texttt{makefile.selection.default} stets
erstellt.

\paragraph{vtkFOX.}
Das \textit{visualization toolkit} (vtk) ist an das FOX-Toolkit zur
betriebssystemunabh"angigen Erstellung graphischer Benutzeroberfl"achen
angedockt. Dadurch nutzen wir die Vorteile der einfachen GUI-Programmierung
mit dem FOX-Toolkit und die OpenGL-Unterst"utzung des vtk. Der
Funktionsumfang ist wesentlich geringer als von Grape, daf"ur
befindet sich vtkFOX unter aktiver Entwicklung (Stand M"arz 2008).

%%%%%%%%%%%%%%%%%%%%%%%%%%%%%%%%%%%%%%%%%%%%%%%%%%%%%%%%%%%%%%%%%%%%%%%%%%%%%%%
%%%%%%%%%%%%%%%%%%%%%%%%%%%%%%%%%%%%%%%%%%%%%%%%%%%%%%%%%%%%%%%%%%%%%%%%%%%%%%%
%%%%%%%%%%%%%%%%%%%%%%%%%%%%%%%%%%%%%%%%%%%%%%%%%%%%%%%%%%%%%%%%%%%%%%%%%%%%%%%

\section{Tricks und Verfahrensweisen}

\subsection{Der Barton-Nackman-Trick}

\subsection{Traits}

\subsection{Programmierstil und -Konventionen}

Hausarbeit erfordert eine straffe Organisation -- wenn sie effektiv sein
soll. Oberster Grundsatz hierbei hei"st: Erst denken, dann handeln.%
\footnote{Loriot: Pappa ante portas.}

% *******************************************************************************
%     File of conventions for programming in the quocmesh-library.
% *******************************************************************************

The following rules apply to all modules, examples and tools. You should observe them in projects, too.

\paragraph{class- and filenames}

\begin{itemize}
  \item class names: use \verb|CamelCase|, starting with upper case letter
  \item member function: use \verb|camelCase|, starting with lower case letter
  \item file name: use \verb|camelCase|, starting with lower case letter (exemption: start with upper case,
        if first word is a proper name, e. g. \verb|ArmijoSearch.h|), \\
        use only the following characters:
        \begin{itemize}
          \item upper and lower case letters (in particular no umlauts) (\texttt{a..z, A..Z})
          \item digits (\texttt{0..9})
          \item underscore and hyphen (\texttt{\_,-})
          \item periods/full stops (\texttt{.})
        \end{itemize}
  \item please use correct English names
  \item avoid name conflicts with system header files (e. g. stl headers)
  \item include each new module header file to corresponding selfTest
\end{itemize}


\paragraph{templates}

\begin{itemize}
  \item template parameters: must contain lowercase letters (\verb|realType| and \verb|RealType|
        are okay, but \verb|REAL| is not)
  \item Naming standard for re-exported template parameters: \\
        \verb|template< typename _DataType > ... typedef _DataType DataType;|
  \item For template parameters to be reexported: use \verb|_SomeType| as template parameter and
        \verb|public typedef _SomeType SomeType|
  \item If declaration and implementation are separate, the template parameters must have the
        same name in both cases, that is \verb|_SomeType|. In the implementation, both versions
        may be used.
\end{itemize}


\paragraph{preprocessor directives}

\begin{itemize}
    \item after \verb|#define|, USE CAPITAL LETTERS
          \begin{verbatim}
#ifndef __BLA_H
#define __BLA_H
// contents of bla.h
#endif         \end{verbatim}
    \item \#ifdef and similar preprocessor directives are not indented.
    \item include guards must be used in all headers, the format is \verb|__AOL_H|
    \item openmp critical sections should be named according to the scheme
\begin{verbatim}
namespace_class_{method,other useful identifier}[_number, if needed]
\end{verbatim}
      e.g.
\begin{verbatim}
#pragma omp critical (aol_RandomGenerator_getNextRnd)
\end{verbatim}
\end{itemize}


\paragraph{structure of externals}

\begin{itemize}
\item standard externals should contain
  \begin{itemize}
  \item \texttt{makefile.local} that sets include and link paths
  \item provide an include header, marked as a system header
\begin{verbatim}
#ifdef __GNUC__
#pragma GCC system_header
#endif
\end{verbatim}
    to prevent compiler warnings for external code
  \item a short description
  \end{itemize}
\item if the external is selected in
  \texttt{makefile.selection.default}, define
  \texttt{USE$\_$EXTERNAL$\_$...} is set automatically.
\item nonstandard externals may contain (small amounts of) code that
  is compiled automatically (if necessary) by the make mechanism
  (\texttt{go} and \texttt{clean} scripts) or by an appropriate visual
  c++ project
\end{itemize}

\paragraph{using externals}
\begin{itemize}
\item all modules (except for those obviously fully depending on an
  external) must compile without the external being used
\item Programs that use external code also have to compile if the
  corresponding external is switched off. The executables should then
  give a useful and informative error message like ''This program
  can't be used without (\textit{corresponding external})''. To achieve
  this, enclose your header and cpp-files in the following
  ifdef-construction (external is GRAPE in this example):
\begin{verbatim}
#ifdef USE_EXTERNAL_GRAPE
    ... (code that uses externals) ...
#endif
\end{verbatim}
In case of executables add the follwing else-part (or a similar one):
\begin{verbatim}
#ifdef USE_EXTERNAL_GRAPE
  ... (code that uses externals) ...
#else
  int main ( int, char** ) {
    cerr << "Without grape external, this program is useless" << endl;
    return ( 0 ) ;
  }
#endif
\end{verbatim}
\end{itemize}


\paragraph{style (indentation, spaces~\ldots)}

\begin{itemize}
    \item 2 spaces are used for indentation (no tabs, not 4 spaces etc.)
    \item preprocessor directives are not indented at all.
    \item \texttt{public:} and similar are not indented relative to the class.
          In both cases, that is the current astyle standard.
    \item brackets (placement in lines and spacing around brackets) are used according to the following scheme:
          \begin{verbatim}
dummy_method ( aol::Vector<RealType> &vec, RealType factor ) {
  for ( int i = 0; i < vec.size(); ++i ) {
    vec[i] = factor * vec[i];
  }
}              \end{verbatim}
          and can be enforced automatically by using util/indent which in turn uses astyle
\end{itemize}


\paragraph{Name convention for methods that import or export data, e. g. aol::Mat A, B inversion (same for transposition etc.)}

\begin{itemize}
  \item \texttt{void A.invert()}: writes $A = A^{-1}$
  \item \texttt{B = A.inverse()}: compute and return inverse of $A$, do not modify $A$
  \item \texttt{A.invertFrom(B)}: $A = B^{-1}$, $B$ unmodified
  \item \texttt{A.invertTo(B)}: $B = A^{-1}$, $A$ unmodified
\end{itemize}


\paragraph{miscellaneous}

\begin{itemize}
  \item comments have to be written in english (except in your own projects, there you can do whatever you want)
  \item use special characters only in your own projects and only on your own risk
\end{itemize}


\paragraph{data sets}

\begin{itemize}
  \item don't commit any data sets (images etc.) except very small data sets for examples or selfTests
        (keep those in directory \texttt{examples/testdata}, files here must be sufficiently free to be usable under quoc license)
\end{itemize}



\paragraph{No convention on~\ldots}

\begin{itemize}
  \item the position of member variables, they may be at the beginning or at the end of a class
  \item No general rule on whether implementation should be inside or outside class definition.
\end{itemize}


\paragraph{Rules for subversion}

\begin{itemize}
  \item Moving code and changing code (e. g. moving implementation out of class and changing it)
        should be committed separately to allow diffing.
  \item use svn:ignore to ignore files that will automatically occur when compiling and typical temp files of editors and IDEs,
        not for personal temporary copies like aol.hold
\end{itemize}


\paragraph{Very special things}
\begin{itemize}
  \item use \texttt{aol::Abs} instead of \texttt{fabs}.
        But: template specialization is necessary for unsigned data types when needed.
  \item Instead of \texttt{M$\_$PI} the expressions \texttt{aol::PI} or
        \texttt{aol::NumberTrait<RealType>::pi} should be used. Analogously for other mathematical
        constants (if not available define own NumberTrait).
  \item \texttt{apply(x, x)} is nowhere forbidden, but produces (mostly) unpredictable output.
        Apply should check for this and throw exception or contain comment that \texttt{apply(x, x)}
        works, we will not change this in all old apply methods now.
  \item Use \texttt{for}-loops where possible, even simple things like \\
        \texttt{a[0] = expression ( 0 ); a[1] = expression ( 1 )} \\
        should be done in a \texttt{for}-loop.
  \item In methods like \texttt{getMinValue()} or \texttt{getMaxValue()} don't initialize the
        first value with $\pm$ infinity, but with \texttt{vector[0]} (otherwise, if the size of
        the vector is $0$, it might happen that $\pm$ infinity is returned). \\
        If it's not really really obvious that in no case anything can ever go wrong with the
        \texttt{[]}-operator of the vector, use \texttt{get} and \texttt{set} (then bounds-checking
        is applied in the debug-mode).
  \item use \texttt{NON\_PARALLEL\_STATIC} if \texttt{static} variables should not be static when using
        parallelization (due to conflicting write access), e.g.\ if \texttt{static} is only used for
        performance reasons. If they always need to be static, prevent parallel write access.
\end{itemize}




%%%%%%%%%%%%%%%%%%%%%%%%%%%%%%%%%%%%%%%%%%%%%%%%%%%%%%%%%%%%%%%%%%%%%%%%%%%%%%%
%%%%%%%%%%%%%%%%%%%%%%%%%%%%%%%%%%%%%%%%%%%%%%%%%%%%%%%%%%%%%%%%%%%%%%%%%%%%%%%
%%%%%%%%%%%%%%%%%%%%%%%%%%%%%%%%%%%%%%%%%%%%%%%%%%%%%%%%%%%%%%%%%%%%%%%%%%%%%%%

\section{Praktische Entwicklung}

\anmerkung{Dieser Abschnitt soll alle Hinweise enthalten, die bei der Benutzung der
folgenden Tools in der Arbeitsgruppe zu beachten sind. Scheidelinie stets:
"`Gilt dieses oder jenes nur in Zusammenhang mit der QuocMeshBibl.?"' Sonst
nur Verweis auf Literatur bzw. das Wiki.}

\subsection{Subversion}

\anmerkung{Verweis aufs Wiki, Angabe mindestens einer ersch"opfenden Literaturquelle
zu Idee von und Umgang mit CVS.\\
Benimmregeln zum Einchecken: Keine nicht-winzigen Daten, nie ohne Kommentar,
kein Commit ohne Kompilieren der \textit{default}-Tests.}

\subsection{make}

\anmerkung {
		  Verweis auf eine gut lesbare makefile-Dokumentation\\
		  Aufbau, Speicherort,
		  Erzeugung, Aufgabe und Funktion der unterschiedlichen makefiles,\\
		  Aufgabe, Funktion, Unterschiede der Targets\\
		  Unterschiede bei Aufruf in verschiedenen Verzeichnissen.}

\subsection{Compiler, Debugging-Tools, Entwicklungsumgebungen}

\anmerkung{Womit programmieren wir, welche Erfahrung gibt es mit anderen
		  Umgebungen, Vor-/Nachteile der Compiler.}

\chapter{Was kann die Bibliothek}

\anmerkung{"Ubersicht �ber die Problemkreise, f�r deren L"osung sich die QuocMesh-Bibl. anbietet}

\chapter{Gliederung der Bibliothek}

\anmerkung{
  Gliederung in Module, Projekte, Tools, Examples, Externals, \dots, Wiederfinden
  der Struktur in der Verzeichnisstruktur, Erkl"arung der einzelnen Gliederungsteile,
  Aufbau der einzelnen Gliederungsteile (in Projekten hat jeder ein eigenes Verzeichnis
  mit Selbstverantwortung, Module sind untergliedert in \dots, usw.),
  Aufgaben und Inhalte der einzelnen Tools, Module usw., Selbsttests, Nutzung der
  Examples, Aufgaben und Inhalte der Namensr"aume und ihre Beziehung zu den Modulen,
  zu jedem Punkt Verweis auf das entsprechende Kapitel.
}

\chapter{Schnelleinstieg}
\label{sec:Schnelleinstieg}

\anmerkung{
  Einmal alle Befehle/Schritte aufgelistet, die nach dem ersten Einloggen im
  Rechnerraum bis zum Speichern einer W"armeleitungsl�sung n"otig sind:\\
  SVN-Vorbereitung (checkout, update usw.), Projekt anlegen, .cpp-Datei schreiben,
  Module einbinden (durch header-Dateien), compilieren/debuggen,
  Programm starten, zum repository hinzuf�gen
}

\chapter{Allgemeine Mathematik}

\anmerkung{Default-Einstellung f"ur alle Container ist deep copy! Ausnahmen nennen:
DiscreteFunction, auto\_container auf Wunsch.}\\
\anmerkung{Alle folgenden Punkte im Schema: Welche gibt es, Unterschiede, wof"ur jeweils
gedacht/zu benutzen, wie zu finden/einzubinden}

%%%%%%%%%%%%%%%%%%%%%%%%%%%%%%%%%%%%%%%%%%%%%%%%%%%%%%%%%%%%%%%%%%%%%%%%%%%%%%%
%%%%%%%%%%%%%%%%%%%%%%%%%%%%%%%%%%%%%%%%%%%%%%%%%%%%%%%%%%%%%%%%%%%%%%%%%%%%%%%
%%%%%%%%%%%%%%%%%%%%%%%%%%%%%%%%%%%%%%%%%%%%%%%%%%%%%%%%%%%%%%%%%%%%%%%%%%%%%%%

\section{Vektoren}

\paragraph{Multivektoren.}

\paragraph{Speichermanagement "uber den Vektor-Manager.}

%%%%%%%%%%%%%%%%%%%%%%%%%%%%%%%%%%%%%%%%%%%%%%%%%%%%%%%%%%%%%%%%%%%%%%%%%%%%%%%
%%%%%%%%%%%%%%%%%%%%%%%%%%%%%%%%%%%%%%%%%%%%%%%%%%%%%%%%%%%%%%%%%%%%%%%%%%%%%%%
%%%%%%%%%%%%%%%%%%%%%%%%%%%%%%%%%%%%%%%%%%%%%%%%%%%%%%%%%%%%%%%%%%%%%%%%%%%%%%%

\section{Operatoren}

\paragraph{Der BlockOp.}

%%%%%%%%%%%%%%%%%%%%%%%%%%%%%%%%%%%%%%%%%%%%%%%%%%%%%%%%%%%%%%%%%%%%%%%%%%%%%%%
%%%%%%%%%%%%%%%%%%%%%%%%%%%%%%%%%%%%%%%%%%%%%%%%%%%%%%%%%%%%%%%%%%%%%%%%%%%%%%%
%%%%%%%%%%%%%%%%%%%%%%%%%%%%%%%%%%%%%%%%%%%%%%%%%%%%%%%%%%%%%%%%%%%%%%%%%%%%%%%

\section{Matrizen}

\newcommand \clsallgdiag [3]
{%
  \framebox(#1,#2)
  {%
    \begin{minipage}{#1\unitlength}
    \begin{center}
    \footnotesize
    #3
    \end{center}
    \end{minipage}
  }
}
\newcommand \clsdiag[2][0.75] {\clsallgdiag{2.5}{#1}{#2}}

\newcommand \abstrakt {\itshape}

\setlength{\unitlength}{10mm}
\begin{picture}(15,8)
\put( 7   , 7   ){\clsdiag{\abstrakt Op}}
\put( 8.25, 6.5 ){\vector(0,1){0.5}}
\put( 3.75, 5.75){\line(0,1){0.75}}
\put( 6.25, 5.75){\line(0,1){0.75}}
\put( 9.75, 5.75){\line(0,1){0.75}}
\put(12.75, 5.75){\line(0,1){0.75}}
\put( 3.75, 6.5 ){\line(1,0){9}}
%
\put( 2.5 , 5   ){\clsdiag{\abstrakt Matrix}}
\put( 1.25, 2.75){\line(0,1){0.75}}
\put( 5.75, 2.75){\line(0,1){0.75}}
\put( 1.25, 3.5 ){\line(1,0){4.5}}
\put( 3.75, 3.5){\vector(0,1){1.5}}
%
\put( 0   , 2   ){\clsdiag{\abstrakt GenBandMatrix}}
\put( 1.25, 0.75){\vector(0,1){1.25}}
\put(-0.5 ,-0.75){\clsallgdiag{3}{1.5}{S9\_2D\_BandMatrix\\S27\_3D\_BandMatrix}}
\put( 4.5 , 2   ){\clsdiag{\abstrakt GenSparseMatrix}}
\put( 4.25, 0.75){\line(0,1){0.75}}
\put( 7.25, 0.75){\line(0,1){0.75}}
\put( 4.25, 1.5 ){\line(1,0){3}}
\put( 5.75, 1.5){\vector(0,1){0.5}}
\put( 3   , 0   ){\clsdiag{SparseMatrix}}
\put( 6   ,-0.75){\clsdiag[1.5]{UniformGrid\\SparseMatrix}}
%
%
\put( 5.5 , 4.25){\clsdiag[1.5]{MyFastUniform\\GridMatrix}}
\put( 8.5 , 5   ){\clsdiag{UGBMatrix}}
%
%
\put(11.5 , 5   ){\clsdiag{\abstrakt GenSparseOp}}
\put(11.25, 2.75){\line(0,1){0.75}}
\put(14.25, 2.75){\line(0,1){0.75}}
\put(11.25, 3.5 ){\line(1,0){3}}
\put(12.75, 3.5){\vector(0,1){1.5}}
%
\put(10   , 1.25){\clsdiag[1.5]{FastUniform\\GridMatrix}}
\put(13   , 1.25){\clsdiag[1.5]{UniGrid\\CSR\_Matrix}}
%
\end{picture}
\vspace{2cm}

\subsection{General explanations}
\paragraph{Why are not all matrix classes derived from \texttt{Matrix}?}
aol::Matrix is an abstract basis class for matrices that provides
virtual methods get, set and add. These methods are mainly used for
assembling matrices, so their efficiency may be critical.

For faster assembly, some ``matrices'' are not derived from
aol::Matrix, thus lacking some methods.

\paragraph{Which interfaces are defined by \texttt{Matrix},
\texttt{GenBandMatrix}, \texttt{GenSparseMatrix} and \texttt{GenSparseOp}? How do they differ?}
aol::GenBandMatrix is an abstract basis class for band matrices. Only
entries in bands parallel (but not necessarily adjacent) to the main
diagonal are stored, the storage is organized row-wise for
cache-efficiency in matrix-vector multiplication.

aol::GenSparseMatrix is an abstract basis class for sparse matrices
that are organized in rows. They contain a vector of pointers to
aol::Rows which may be filled with different row objects.

\subsection{The matrix classes in detail}
Note that remarks about the speed of matrices may significantly depend
on the platform, the compiler and its settings, so you should run the
sparsebench benchmark to see which matrix works best in your
environment.

\subsubsection{aol::FullMatrix}
%%%%%%%%%%%%%%%%%%%%%%%%%%%%%%%
\paragraph{For which applications is this matrix class designed?}
Dense matrices.

\paragraph{Advantages}
``Fast'' access.

\paragraph{Disadvantages}
Storage of all entries, even if they are zero.

\paragraph{Internal structure and how it works}
Two-dimensional array of all entries.


\subsubsection{aol::SparseMatrix}
%%%%%%%%%%%%%%%%%%%%%%%%%%%%%%%%%
\paragraph{For which applications is this matrix class designed?}
Unstructured sparse matrix.

\paragraph{Advantages}
Most flexible class for sparse matrices.

\paragraph{Disadvantages}
Least efficient because no sparsity structure is known implicitely.

\paragraph{Internal structure and how it works}
Each row stores a (sorted) stl-vector of aol::RowEntries containing a
pair (column index, value).



\subsubsection{qc::S9\_2D\_BandMatrix, qc::S27\_3D\_BandMatrix}
%%%%%%%%%%%%%%%%%%%%%%%%%%%%%%%%%%%%%%%%%%%%%%%%%%%%%%%%%%%%%%%
\paragraph{For which applications is this matrix class designed?}
For applications on an uniform rectangular (2D) or an uniform hexagonal (3D) mesh,
which result in 9 (2D) or 27 (3D) entries per row.

\paragraph{Advantages}
Fast access and matrix-vector multiplication, fast row-wise access.

\paragraph{Disadvantages}
Small memory overhead (two integers per row).

\paragraph{Internal structure and how it works}
Only bands containing nonzero entries are stored, entries are stored
row-wise. Lookup tables for the offsets are created when creating
matrix.

\subsubsection{aol::TriBandMatrix, aol::LQuadBandMatrix}
%%%%%%%%%%%%%%%%%%%%%%%%%%%%%%%%%%%%%%%%%%%%%%%%%%%%%%%%
\paragraph{For which applications is this matrix class designed?}
Tridiagonal matrices and Quadridiagonal matrices (two lower, one upper
diagonal) adjacent to main diagonal.


\subsubsection{qc::UniformGridSparseMatrix}
%%%%%%%%%%%%%%%%%%%%%%%%%%%%%%%%%%%%%%%%%%%
\paragraph{For which applications is this matrix class designed?}
For applications on an uniform rectangular (2D) or an uniform hexagonal (3D) mesh,
which result in 9 (2D) or 27 (3D) entries per row.

\paragraph{Advantages}
The \texttt{applyAdd}-method is about 2 times (2D) or
1.5 times (3D) faster than the simple SparseMatrix.

\paragraph{Disadvantages}
The \texttt{assemble}-method is not faster, but even slightly slower.

\paragraph{Internal structure and how it works}
This matrix is derived from \texttt{GenSparseMatrix}, the main
difference is: The rows that belong to inner nodes are special
\texttt{UniformGridSparseRows} (implemented in rows.h) which are just
arrays of the specified DataType with 9 (2D) or 27 (3D) entries.  The
dimension is read from the grid and set automatically. The methods
\texttt{get} and \texttt{set} allow for the uniform structure of the
grid and compute the belonging array-index to the desired
matrix-entry.

Furtheron the \texttt{realloc}-method makes use of the uniform grid
structure.


\subsubsection{qc::FastUniformGridMatrix}
%%%%%%%%%%%%%%%%%%%%%%%%%%%%%%%%%%%%%%%%%
\paragraph{For which applications is this matrix class designed?}
For applications on an uniform rectangular (2D) or an uniform
hexagonal (3D) mesh, which result in 9 (2D) or 27 (3D) entries per
row.

\paragraph{Advantages}
Since this matrix is not derived from aol::Matrix, it doesn't use
the virtual get and set methods, but its own faster ones.
The \texttt{assemble}-method is much faster than the one from the
UniformGridSparseMatrix (about 2.5 times in 2D and 3D), the
\texttt{apply}-method is at least a bit faster (about 1.4 times in 2D
and 3D).

\paragraph{Disadvantages}
Not derived from aol::Matrix

\paragraph{Internal structure and how it works}
The \texttt{FastUniformGridMatrix} is derived from the
\texttt{GenSparseOp} and \textit{not} from the \texttt{GenSparseMatrix}.
This makes it possible that it has its own once more optimized
methods.  The rows are again stored as simple DataType-arrays, but the
access is organised with some kind of hierarchical principle: The rows
are divided into three blocks, which consist of three diagonals (in
2D) or once again of three blocks which then are divided into three
diagonals (in 3D). To access the value of an index $(i,j)$ the
belonging block and the index in this block are computed.

Furtheron the \texttt{applyAdd}-method also doesn't need to use the
\texttt{mult}-method from the \texttt{rows}, as it is the case for all
matrices that are derived from the \texttt{GenSparseMatrix}. Instead
it is implemented in a hierarchical way which uses the block structure
described above.

\subsubsection{qc::UGBMatrix}
%%%%%%%%%%%%%%%%%%%%%%%%%%%%%
\paragraph{For which applications is this matrix class designed?}
For applications which produce a band matrix and work on the grids of
types RectangularGrid or GridDefinition.

\paragraph{Advantages}
Very cache-efficient matrix-vector multiplication.

\paragraph{Disadvantages}
Not an aol::Matrix.

\paragraph{Internal structure and how it works}
The method used here is a sparse banded block multiplication scheme
which takes advantage of data localization in the cache of the
processor. With use of the intel compiler icc/ecc and the pragma
directives pragma ivdep and efficient vectorization of the code can be
achieved by setting the define VECTORIZE\_INTEL during
compilation. \\
The template parameter blocksize defines the size of the blocks which
are used in the multiplication method. blocksize should be chosen such
that sizeof(DataType)*columns*blocksize is less than the size of the
largest cache of the processor.


\subsubsection{aol::CSR\_Matrix}
%%%%%%%%%%%%%%%%%%%%%%%%%%%%%%%
\paragraph{For which applications is this matrix class designed?}
Unstructured sparse matrices.

\paragraph{Advantages}
Efficient matrix-vector multiplication.

\paragraph{Disadvantages}
Not an aol::Matrix. No (or very slow) random access. Only works for
double precision data type.

\paragraph{Internal structure and how it works}
A matrix stored in compressed sparse row storage format, a standard
format (see MKL).

\subsubsection{qc::UniGridCSR\_Matrix}
%%%%%%%%%%%%%%%%%%%%%%%%%%%%%%%%%%%%%
\paragraph{For which applications is this matrix class designed?}
For applications on an uniform rectangular (2D) or an uniform
hexagonal (3D) mesh, which result in 9 (2D) or 27 (3D) entries per
row.

\subsubsection{MyFastUniformGridMatrix}
%%%%%%%%%%%%%%%%%%%%%%%%%%%%%%%%%%%%%%%
This matrix class is obsolete and will be (or has been) merged with
FastUniformGridMatrix.

%%%%%%%%%%%%%%%%%%%%%%%%%%%%%%%%%%%%%%%%%%%%%%%%%%%%%%%%%%%%%%%%%%%%%%%%%%%%%%%
%%%%%%%%%%%%%%%%%%%%%%%%%%%%%%%%%%%%%%%%%%%%%%%%%%%%%%%%%%%%%%%%%%%%%%%%%%%%%%%
%%%%%%%%%%%%%%%%%%%%%%%%%%%%%%%%%%%%%%%%%%%%%%%%%%%%%%%%%%%%%%%%%%%%%%%%%%%%%%%

\section{L"oser f"ur lineare Gleichungssysteme}

\anmerkung{bish. Anleitung verbessern, Liste h"ubscher, aber beibehalten}
\anmerkung{Referenz f"ur Algorithmen angeben. Wenn Notation selbst erdacht:
Algorithmus hinschreiben.\\
Voraussetzungen an Matrix klarer und deutlicher\\
Benutzung der L"oser vor Auflistung}

\anmerkung{Was ist mit Martins QR-Zerlegung, alte Anl. Seite 69?}

\section{Vorkonditionierer}

\anmerkung{wie L"oser: bisherige Liste behalten, nur aufh"ubschen}

%%%%%%%%%%%%%%%%%%%%%%%%%%%%%%%%%%%%%%%%%%%%%%%%%%%%%%%%%%%%%%%%%%%%%%%%%%%%%%%
%%%%%%%%%%%%%%%%%%%%%%%%%%%%%%%%%%%%%%%%%%%%%%%%%%%%%%%%%%%%%%%%%%%%%%%%%%%%%%%
%%%%%%%%%%%%%%%%%%%%%%%%%%%%%%%%%%%%%%%%%%%%%%%%%%%%%%%%%%%%%%%%%%%%%%%%%%%%%%%

\section{Newton-Verfahren}

\anmerkung{neu verfassen f"ur Benjamins Newton-Verf.\\
Dabei Klassendiagramm wg. un"ubersichtlicher Vererbung\\
Ablaufdiagramm aus Stefans Kurzdoku Pkt. 1.5}

%%%%%%%%%%%%%%%%%%%%%%%%%%%%%%%%%%%%%%%%%%%%%%%%%%%%%%%%%%%%%%%%%%%%%%%%%%%%%%%
%%%%%%%%%%%%%%%%%%%%%%%%%%%%%%%%%%%%%%%%%%%%%%%%%%%%%%%%%%%%%%%%%%%%%%%%%%%%%%%
%%%%%%%%%%%%%%%%%%%%%%%%%%%%%%%%%%%%%%%%%%%%%%%%%%%%%%%%%%%%%%%%%%%%%%%%%%%%%%%

\section{Gradientenabstiegsverfahren}
\anmerkung{ebenfalls Klassendiagramm. GradientFlow-Erkl"arung,
alte Anl. Seite 81, wiederverwenden}

%%%%%%%%%%%%%%%%%%%%%%%%%%%%%%%%%%%%%%%%%%%%%%%%%%%%%%%%%%%%%%%%%%%%%%%%%%%%%%%
%%%%%%%%%%%%%%%%%%%%%%%%%%%%%%%%%%%%%%%%%%%%%%%%%%%%%%%%%%%%%%%%%%%%%%%%%%%%%%%
%%%%%%%%%%%%%%%%%%%%%%%%%%%%%%%%%%%%%%%%%%%%%%%%%%%%%%%%%%%%%%%%%%%%%%%%%%%%%%%

\section{Gew"ohnliche DGl}

\anmerkung{Motivation (Ortsdiskret. "uber FE, Zeit "uber fin. Diff. f"uhrt zu
gro"sen DGl-Systemen)}

\paragraph{Die Klasse Timestep}
\anmerkung{alte Anl. Seite 75, aber allgemeiner. Parabolische pDGl nur als
Beispiel.}

\chapter{Finite Elemente}

\section{Gitter, Konfiguratoren, Basisfunktionen}
\lstset{aboveskip=\medskipamount}
% Autor von Deylen

Wir betrachten beispielhaft das Helmholtz-Problem: Finde f"ur
festes $\alpha \in \R$ ein $u: \bar \Omega \to \R$ mit
\begin{eqnarray*}
 u - \alpha \Delta u & = & f  \quad \mbox{ in } \Omega \\
\partial_\nu u &=& 0 \quad \mbox{ auf } \partial \Omega
\end{eqnarray*}

Diese Gleichheit weichen wir f"ur die schwachen Formulierung auf zu
\[
	\int_\Omega (u - \alpha \Delta u) \phi = \int_\Omega f\,\phi \qquad
	\text{f"ur alle } \phi \in H^{1,2}(\Omega).
\]
Durch partielle Integration ergibt dies
\[
	\int_\Omega  u \, \phi + \alpha \int_\Omega \nabla u \cdot \nabla \phi = \int_{\partial \Omega }
	\underbrace{\nabla u \cdot \nu}_{=0}\, \phi + \int f \, \phi
\]
Wir k"urzen $F(\phi) := \int f \, \phi$ ab.

Gesucht ist nun eine Approximation $U \in \V_h$ von $u$ f"ur einen
endlichdimensionalen Ansatzraum $\V_h = \Span (\phi_1, \dots, \phi_N)$:
\[
	U = \sum_j \bar U_j\, \phi_j
\]
Dies f"ur $u$ eingesetzt ergibt (Integrale von jetzt an immer "uber $\Omega$)
\[
	\int \sum \bar U_j \, \phi_j \, \phi_i + \alpha
	\int \sum \bar U_j \, \nabla \phi_j \cdot \nabla \phi_i = F(\phi_i)
	\qquad \text{f"ur alle $i$}
\]
Integration und Summenbildung l"a"st sich vertauschen, es bleiben
nur die Integrale "uber $\phi_j \, \phi_i$ respektive die Gradienten
"ubrig. Sie fassen wir in Matrizen zusammen:
\[
	M_{ij} := \int \phi_i\, \phi_j, \qquad
	L_{ij} := \int \nabla \phi_i \, \nabla \phi_j
\]
Dies ergibt ein lineares Gleichungssystem:
\[
	(M + \alpha L) \bar U = \bar F
\]

Will man diese L"osungsmethode praktisch implementieren, lassen sich die
zu kl"arenden Fragen (mit einigem guten Willen) aufteilen in:
\begin{description}
\item[Ansatzfunktionen:]
	Generell wird man $\phi_i$ mit kompaktem Tr"ager verwenden, die einen
	Spline-Raum zur Approximation von $u$ aufspannen, dessen Elemente dieses
	Ansatzraumes global stetig, stetig differenzierbar oder (seltener)
	von h"oherer globaler Glattheit sein werden. Die $\phi_i$ werden stets
	als nodale Basis dieses Raums gew"ahlt. Beispiele:
	\begin{enumerate}
	\item	Hutfunktionen f"ur global stetige Splines (Tr"ager sind
			in 2D sechs Dreiecke und in 3D acht Tetraeder)
	\item	in 2D bilineare, in 3D trilineare, allgemein
			multilineare Funktionen
			f"ur global stetig differenzierbare Splines
			(Tr"ager sind die vier (2D) bzw. acht (3D)
			umgebenden Einheitskuben)
	\end{enumerate}

	Die �blichen Ansatzfunktionen der QuocMeshes sind multilinear, die
	Elemente sind also Quader. Es gibt auch Unterst�tzung f�r lineare
	Ansatzfunktionen auf simplizialen Elementen.
%
\item[Gitter:]
	Soll das Gitter uniform sein, d. h. sollen
	alle im Rechenbereich liegenden Punkte eines Gitternetzes
	$\frac 1 h \mathds Z^d \subset \R^d$ benutzt werden, nur eine
	Auswahl davon, oder darf das Gitter auch unregelm"a"sig
	verfeinert sein?

	In den QuocMeshes ist das Gitter stets
	uniform, au"ser bei der DT-Grid-Struktur sind stets alle
	Gitterpunkte des Einheitsw"urfels Aufpunkte je einer Basisfunktion.
\item[Quadratur:]
	Die Integration "uber die einzelnen Elemente sollte
	unabh"angig von der Wahl der Basisfunktionen implementiert
	werden. -- Zwar k"onnte das Programm automatisch nach Wahl
	der Basisfunktion eine Quadraturregel bestimmen, die die
	o. g. Integrale exakt berechnet; dieser Ansatz st"o"st
	allerdings an seine Grenzen, sobald nichtlineare
	Operatoren betrachtet werden. Deswegen ist es m�glich,
	die Ordnung der Quadraturregel unabh�ngig von den
	Ansatzfunktionen zu bestimmen.

	Die QuocMeshes verwenden Mittelpunkts- und
	Gau"squadraturen, deren Ordnung
	meist so gew"ahlt wird, da"s die linearen Operatoren
	exakt integriert werden bzw. bei nichtlinearen
	Operatoren die Fehlerordnung gen"ugend hoch ist.
\end{description}

Entscheidungen �ber diese drei Ingredienzen des FE-Algorithmus werden
mithilfe von \textbf{Gittern} und \textbf{Konfiguratoren} getroffen.
Dabei speichert ein Gitter nur die topologische und geometrische Anordnung
der Elemente, der Konfigurator w�hlt Basisfunktionen und eine Quadraturregel
aus. Und wie das geht -- zeigen wir Euch nach der n�chsten Maus\footnote
{sc. Abschnitts�berschrift.}

%%%%%%%%%%%%%%%%%%%%%%%%%%%%%%%%%%%%%%%%%%%%%%%%%%%%%%%%%%%%%%%%%%%%%%%%%%%%%%%
%%%%%%%%%%%%%%%%%%%%%%%%%%%%%%%%%%%%%%%%%%%%%%%%%%%%%%%%%%%%%%%%%%%%%%%%%%%%%%%
%%%%%%%%%%%%%%%%%%%%%%%%%%%%%%%%%%%%%%%%%%%%%%%%%%%%%%%%%%%%%%%%%%%%%%%%%%%%%%%

\section{Darstellung am Beispiel}

\subsection*{Beispiel 1: MassOp zur Integration}

Ein anderes, mit den QuocMeshes erstelltes Programm, habe ein 2D-Array von
Funktionswerten berechnet, die nun eingelesen und als Werte einer
Funktion $u$ an den Knoten eines W�rfelgitters verstanden werden sollen.
Dazwischen soll $u$ bilinear interpoliert werden. Der hierf�r ben�tigte
Konfigurator ist \lstinline!qc::QuocConfiguratorTraitMultiLin!. Er
bekommt als Template-Argumente
\begin{enumerate}
\item	den gew�nschten Datentyp f�r Flie�komma-Zahlen (\lstinline!RealType!)
\item	die Dimension, hier \lstinline!qc::QC_2D!
\item	die gew�nschte Quadraturregel. Es gibt keine Gau�-Quadratur zweiter
		Ordnung, mit zwei St�tzstellen erhalten wir bereits Ordnung 3.
		Also verwenden wir die Klasse \\
		\lstinline!aol::GaussQuadrature<double, qc::QC_2D, 3>!.
\end{enumerate}

\lstset{numbers=left, firstnumber=last}
\begin{lstlisting}[firstnumber=0]
typedef double RealType;
typedef aol::GaussQuadrature<RealType, qc::QC_2D, 3> QuadType;
typedef qc::QuocConfiguratorTraitMultiLin<RealType,
                                       qc::QC_2D, QuadType> ConfType;

int main() {
\end{lstlisting}
F�r dieses Beispiel kennen wir die Gr��e des Datensatzes, er sei eine
$(2^5 + 1) \times (2^5 + 1)$-Matrix von \lstinline!double!-Werten:
\begin{lstlisting}
  qc::ScalarArray2d<RealType> u ( "fct_data.dat" );
  qc::GridDefinition grid ( 5, qc::QC_2D );
  ConfType configurator ( grid );
\end{lstlisting}
Wir wollen jetzt
\[
	\int_\Omega u = \int_\Omega \sum_i \bar U_i \phi_i
\]
berechnen. Da die multilinearen Basisfunktionen $\phi_i$ eine Zerlegung
der Eins bilden, ist dies gleich
\[
	\qquad \qquad = \int_\Omega \sum_{i,j} \bar U_i \phi_i \phi_j = M\, \bar U \cdot \bar 1,
\]
wobei $\bar 1$ den Vektoren bezeichnet, dessen Eintr�ge alle $1$ sind.
Die Massematrix $M$ bekommen wir durch den \lstinline!MassOp!. Der
Rest des Programms ist lineare Algebra:
\begin{lstlisting}
  aol::Vector<RealType> M_u ( u.size() );
  aol::MassOp<ConfType> massOp ( grid, aol::ASSEMBLED );
  massOp.apply ( u, M_u );

  aol::Vector<RealType> ones ( u.size() );
  ones.setAll ( 1. );
  cout << "Integral ueber u: " << M_u * ones << endl;
}
\end{lstlisting}
\lstset{firstnumber=auto}

In diesem Beispiel speichert der \lstinline!MassOp! selbst die assemblierte
Matrix. Im n�chsten Beispiel werden wir dem Konstruktor kein zweites
Argument �bergeben. Dann wird das Default-Argument \lstinline!aol::ONTHEFLY!
benutzt. Es sorgt daf�r, da� der Operator bei jedem \lstinline!apply!-Aufruf
seine Quadratur

\subsection*{Beispiel 2: Helmholtz-Problem mit Neumann-Randwerten}

In diesem Beispiel laden wir die Randdaten zum L�sen der am Kapitelanfang
beschriebenen partiellen Differentialgleichung. Wir assemblieren
die Systemmatrix $M + \alpha L$ in einer Matrix und �bergeben diese an
einen CG-L�ser. Die \lstinline!typedef!s �bernehmen wir. Den Typ
\lstinline!ScalarArray2d! brauchen wir gar nicht explizit zu verwenden,
er steht durch den Konfigurator als \lstinline!ConfType::ArrayType!
bereits zur Verf�gung. Im Gegensatz zum \lstinline!Vector! speichert er
seine Inhalte mit Informationen �ber $x$- und $y$-Ausdehnung.

\begin{lstlisting}
typedef double RealType;
typedef aol::GaussQuadrature<RealType, qc::QC_2D, 3> QuadType;
typedef qc::QuocConfiguratorTraitMultiLin<RealType,
                                       qc::QC_2D, QuadType> ConfType;

int main() {
  ConfType::ArrayType f ( "rhs_data.dat" );
  qc::GridDefinition grid ( 5, qc::QC_2D );
  ConfType configurator ( grid );

  aol::StiffOp<ConfType> stiffOp ( grid );
  aol::MassOp<ConfType>  massOp ( grid );
  ConfType::MatrixType mat ( grid );
  massOp.assembleAddMatrix( mat );
  RealType alpha; cin >> alpha; mat *= alpha;
  stiffOp.assembleAddMatrix ( mat, alpha );

  ConfType::ArrayType rhs ( f, aol::STRUCT_COPY ),
                        u ( f, aol::STRUCT_COPY );
  mass.apply( f, rhs );

  aol::CGInverse<aol::Vector<RealType> > solver( mat );
  solver.apply( rhs, u );
}
\end{lstlisting}

\subsection*{Schlu�wort zu den Beispielen}

An dieser Stelle w�re eine Liste sch�n, die die m�glichen Gitter, Konfiguratoren
und Quadraturregeln aufz�hlt. Eine solche List ist aber unm�glich zu warten,
daher verweisen wir auf das Prinzip �rtlicher Lokalit�t in der \textit{doxygen}-%
Dokumentation und den Bibliotheks-Headern: Eng verwandte Klassen haben meistens
�hnliche Namen, stehen also in der \textit{doxygen}-Klassenliste eng
beieinander, und meistens sind sie auch in einer gemeinsamen Datei deklariert.

Antworten auf die Frage "`Was mu� jedes Gitter / jeder Konstruktor k�nnen?"'
versuchen die ersten Abschnitte des Kapitels \ref{sec:Gitterarten} zu geben.

%%%%%%%%%%%%%%%%%%%%%%%%%%%%%%%%%%%%%%%%%%%%%%%%%%%%%%%%%%%%%%%%%%%%%%%%%%%%%%%
%%%%%%%%%%%%%%%%%%%%%%%%%%%%%%%%%%%%%%%%%%%%%%%%%%%%%%%%%%%%%%%%%%%%%%%%%%%%%%%
%%%%%%%%%%%%%%%%%%%%%%%%%%%%%%%%%%%%%%%%%%%%%%%%%%%%%%%%%%%%%%%%%%%%%%%%%%%%%%%

\section{Maskierung (Dirichlet-Randwerte)}

Das Poisson-Problem
\begin{eqnarray*}
 - \Delta u & = & 0 \quad \mbox{ in } \Omega \\
          u & = & u_\text{bd} \quad \mbox{ auf } \partial \Omega
\end{eqnarray*}
l��t sich FE-diskretisiert elegant schreiben als:
\[
	R_\text{int}\, L \, E_\text{int} \, \bar U_\text{int} = - R_\text{int}\, L\, E_\text{bd} \bar U_\text{bd}
\]
Dabei sind $U_\text{int}$ und $U_\text{bd}$ die Vektoren von Knotenwerten an
den inneren bzw. �u�eren Kanten, $E$ verl�ngert sie auf eine gemeinsame L�nge,
und $R_\text{int}$ wirft die Randknoten wieder fort. Die Steifigkeitsmatrix $L$
ist hier also f�r alle, innere und Randknoten, gemeint.

Zur praktischen Realisierung setzen wir
\[
	\bar U := E_\text{int} \, \bar U_\text{int}, \qquad
	\bar B := E_\text{int} \, \bar U_\text{bd}
\]
Jetzt haben alle verwendeten Vektoren eine gemeinsame L�nge.

Die Restriktion $R_{int}$ hat im Programm keinen direkten Platz.
Stattdessen ersetzen wir $R_{int}\, L$ auf der linken Seite
durch eine Steifigkeitsmatrix, bei der in Zeilen und Spalten,
die zu Randknoten geh�ren, alle Elemente gel�scht und nur das
Diagonalelement auf $1$ gesetzt wird. Das Ergebnis ist �quivalent
zu einer Steifigkeitsmatrix, die ohne diese Knoten assembliert wurde,
zuz�glich einiger "`nicht dazugeh�riger"' Zeilen und Spalten mittendrin.

Insbesondere werden beim L�sen eines lin. GlS $L_\text{gestr.} \bar U =
\bar F$ die zu Randknoten geh�rigen Werte in $\bar F$ direkt kopiert,
nur die Werte zu inneren Knoten nehmen am eigentlichen L�sen teil.

Die rechte
Seite behandeln wir jetzt, indem wir ohne weitere Vorbehandlung die
Steifigkeitsmatrix $\bar B$ multiplizieren und danach auf die
Randknoten wieder schreiben, was in $\bar U_\text{bd}$ stand.

Modulo einer in Wirklichkeit nicht stattgefundenen Umsortierung
der Knoten ist unser System jetzt folgendes:
\[
	\begin{pmatrix}
	L_\text{int} & 0 \\
	0 & 1 \\
	\end{pmatrix} \, \begin{pmatrix}
	                 \bar U_\text{int}\\
	                 \bar U_\text{bd} \\
	                 \end{pmatrix} = \begin{pmatrix}
	                                 - R_\text{int} \, L\, E_\text{bd} \bar U_\text{bd} \\
	                                 \bar U_\text{bd} \\
	                                 \end{pmatrix}
\]
Die Implementierung ist \textit{straight forward}, wenn man
die FE-Operator-Methode
\begin{lstlisting}[aboveskip=0.5\medskipamount, numbers=none]
assembleAddMatrix ( ConfType::MatrixType &, const ConfType::MaskType & )
\end{lstlisting}
kennt:
\begin{lstlisting}[numbers=left]
const qc::Dimension DIM = qc::QC_2D;
typedef double RealType;
typedef aol::GaussQuadrature<RealType, DIM, 3> QuadType;
typedef qc::QuocConfiguratorTraitMultiLin<RealType,
                                            DIM, QuadType > ConfType;
int main () {
  const int depth = 5;
  qc::GridDefinition grid ( depth, DIM );
  const int N = grid.getNumX();

  ConfType::ArrayType bdryValues ( grid );
  ConfType::MaskType bdryMask ( grid );
  bdryMask.setAll ( false );

  // note that only upper and lower bdry are set to Dirichlet
  // boundaries. On left and right side, we will get natural
  // Neumann boundary conditions.
  for ( int i = 0; i < N; ++i ) {
    DirichletMask.set ( i, 0 , true );
    bdryValues.set ( i,  0 , aol::NumberTrait<double>::zero );

    DirichletMask.set ( i, N - 1, true );
    bdryValues.set ( i, N - 1,  1 + grid.H() * i );
  }

  aol::StiffOp< ConfType > stiffOp ( grid );
  ConfType::MatrixType stiffOpMasked ( grid );
  stiffOp.assembleAddMatrix ( stiffOpMasked, bdryMask );

  ConfType::ArrayType rhs ( grid );
  stiffOp.apply ( bdryValues, rhs );
  rhs *= -1.;
  rhs.assignMasked ( bdryValues, bdryMask );

  ConfType::ArrayType u ( grid );
  aol::CGInverse<aol::Vector<RealType> > solver( mat );
  solver.apply( rhs, u );
}
\end{lstlisting}


%%%%%%%%%%%%%%%%%%%%%%%%%%%%%%%%%%%%%%%%%%%%%%%%%%%%%%%%%%%%%%%%%%%%%%%%%%%%%%%
%%%%%%%%%%%%%%%%%%%%%%%%%%%%%%%%%%%%%%%%%%%%%%%%%%%%%%%%%%%%%%%%%%%%%%%%%%%%%%%
%%%%%%%%%%%%%%%%%%%%%%%%%%%%%%%%%%%%%%%%%%%%%%%%%%%%%%%%%%%%%%%%%%%%%%%%%%%%%%%

\section{Vorhandene FE-Operatoren}

\originalTeX
Here, an overview over the finite element interface classes is given.
We will distinguish between linear operator interfaces (which additionally
to "apply(...)" and "applyAdd(...)" have a method "assembleAddMatrix(...)" to assemble a system matrix)
and nonlinear operators (which cannot assemble a matrix).
The first table enlists the linear operator interfaces, where the second column displays the matrix assembled in "assembleAddMatrix(...)"
(the result of "apply(...)" then is just this matrix multiplied by the vector of function values, passed to "apply(...)" as the first argument),
the left column gives the corresponding class names, and the right column enlists the methods to be overloaded.
The second table shows all nonlinear operator interfaces and has the same structure, only the middle column shows the result of the method "apply(...)".
In both cases, $\phi$ shall represent the discretized function, which is passed to "apply(...)" or "applyAdd(...)" as the first argument,
$\varphi_i$ represents the $i$th finite element base function, $w$, $f$, $A$ are functions, which have to be implemented by the overload method
in the derived class. An arrow over a function signifies that the function is vector-valued, two arrows
symbolize a matrix. The implementation of $f\left(\phi(x),\nabla\phi(x),x\right)$ gets $x$ and the function $\phi$ as argument, where $\phi$
itself may be evaluated at $x$, or its gradient, or both. $n_g$ shall denote the number of components of a vector function $\vec g$.

{
\renewcommand{\arraystretch}{2}
\resizebox{\textwidth}{!}{
\begin{tabular}{llll}
  \hline
    \textbf{FELin...Interface} & \textbf{expression} & \textbf{overload method} & \textbf{comment} \\
  \hline
    ScalarWeightedMass      & $\left(\int_\Omega w(x) \varphi_j(x) \varphi_i(x) dx\right)_{ij}$                                                            & $w$: getCoeff() \\
    ScalarWeightedStiff     & $\left(\int_\Omega w(x) \nabla\varphi_j(x) \cdot \nabla\varphi_i(x)dx \right)_{ij}$                                          & $w$: getCoeff() \\
    MatrixWeightedStiff     & $\left(\int_\Omega A(x)\nabla \varphi_j \cdot \nabla \varphi_i dx\right)_{ij}$                                               & $A$: getCoeffMatrix() & $A$ symmetric \\
    AsymMatrixWeightedStiff & $\left(\int_\Omega A(x)\nabla \varphi_j \cdot \nabla \varphi_i dx\right)_{ij}$                                               & $A$: getCoeffMatrix() & $A$ asymmetric \\
    ScalarWeightedMixedDiff & $\left(\int_\Omega w(x) \frac{\partial\varphi_j(x)}{\partial x_s} \frac{\partial\varphi_i(x)}{\partial x_t} dx\right)_{ij}$  & $w$: getCoeff() \\
    ScalarWeightedSemiDiff  & $\left(\int_\Omega \varphi_j(x) \frac{\partial\varphi_i(x)}{\partial x_k} w(x)\,dx\right)_{ij} $                             & $w$: getCoeff() & can also return the transpose \\
    VectorWeightedSemiDiff  & $\left(\int_\Omega \varphi_j(x) (\nabla \varphi_i(x) \cdot \vec{w}(x))\,dx\right)_{ij} $                                     & $\vec{w}$: getCoefficientVector() & can also return the transpose, \\
                            &                                                                                                                              & & $n_w=$domain dimension \\
  \hline
\end{tabular}
}
}

\vspace*{1cm}
{
\renewcommand{\arraystretch}{2}
\resizebox{\textwidth}{!}{
\begin{tabular}{llll}
  \hline
    \textbf{FENonlin...Interface} & \textbf{expression} & \textbf{overload method} & \textbf{comment} \\
  \hline
    Op                       & $\left(\int_\Omega f\left(\phi(x),\nabla\phi(x),x\right) \varphi_i(x) dx\right)_i $                                           & $f$: getNonlinearity() \\
    DiffOp                   & $\left(\int_\Omega \vec{f}\left(\phi(x),\nabla\phi(x),x\right)\cdot \nabla\varphi_i(x) dx\right)_i $                          & $\vec{f}$: getNonlinearity() \\
    VectorOp                 & $\left(\int_\Omega \vec{f}\left(\vec\phi(x),\nabla\vec\phi(x),x\right)\cdot\vec\varphi_i(x) dx\right)_i $                     & $\vec{f}$: getNonlinearity() & $n_f$, $n_\phi$ are template parameters \\
    VectorDiffOp             & $\left(\!\!\int_\Omega \!\!\vec{\vec{f}}\left(\vec\phi(x),\nabla\vec\phi(x),x\right):\nabla \vec\varphi_i(x) dx\!\!\right)_i$ & $\vec{\vec{f}}$:getNonlinearity() \\
    IntegrationScalar        & $\int_\Omega f\left(\phi(x),\nabla\phi(x),x\right) dx$                                                                        & $f$: evaluateIntegrand() \\
    IntegrationVector        & $\int_\Omega f\left(\vec\phi(x),\nabla\vec\phi(x),x\right) dx$                                                                & $f$: evaluateIntegrand() & $n_\phi$ is template parameter \\
    IntegrationVectorGeneral & $\int_\Omega f\left(\vec\phi(x),\nabla\vec\phi(x),x\right) dx$                                                                & $f$: evaluateIntegrand() & $n_\phi\leq$domain dimension is variable \\
  \hline
\end{tabular}
}
}
VectorFENonlinIntegrationVectorInterface can be used for vector valued integrands.

\germanTeX

%%%%%%%%%%%%%%%%%%%%%%%%%%%%%%%%%%%%%%%%%%%%%%%%%%%%%%%%%%%%%%%%%%%%%%%%%%%%%%%
%%%%%%%%%%%%%%%%%%%%%%%%%%%%%%%%%%%%%%%%%%%%%%%%%%%%%%%%%%%%%%%%%%%%%%%%%%%%%%%
%%%%%%%%%%%%%%%%%%%%%%%%%%%%%%%%%%%%%%%%%%%%%%%%%%%%%%%%%%%%%%%%%%%%%%%%%%%%%%%

\section{Weitere Bibliothekselemente}

\anmerkung{Punkte der bisherigen Doku "uberarbeiten:
zusammenh"angende S"atze, ggf. k"urzer, daf"ur Verweis auf bestehende Projekte}

\anmerkung{Angabe von Ansprechpartnern?}

\subsection{Faltung}

\subsection{$L^2$-Projektion}

\subsection{Mean curvature motion (MCM)}

\subsection{Wulff \& Frank}

\chapter{Gitterarten und ihre Konfiguratoren}
\label{sec:Gitterarten}

\anmerkung{Geh"oren BEM hierhin oder besser ins FE-Kapitel? Oder ein eigenes?}


%%%%%%%%%%%%%%%%%%%%%%%%%%%%%%%%%%%%%%%%%%%%%%%%%%%%%%%%%%%%%%%%%%%%%%%%%%%%%%%
%%%%%%%%%%%%%%%%%%%%%%%%%%%%%%%%%%%%%%%%%%%%%%%%%%%%%%%%%%%%%%%%%%%%%%%%%%%%%%%
%%%%%%%%%%%%%%%%%%%%%%%%%%%%%%%%%%%%%%%%%%%%%%%%%%%%%%%%%%%%%%%%%%%%%%%%%%%%%%%

\section{Vollbesetzte QuocMeshes}
% Autor von Deylen, Oct 2008

Die Knoten der Quoc-Gitter sind invers lexikographisch numeriert, der
Knoten $(i, j, k)$ in einem Gitter der Gr��e $(n_x, n_y, n_z)$ hat
also Index $i + n_x (j + n_y\,k)$. Quaderf�rmige Elemente werden
indiziert �ber die vordere untere linke Ecke indiziert. Zu simplizialen
Elementen siehe Abschnitt \ref{sec:SimplElemente}.

Bisher (Okt. 2008) wird die Gr��e fast aller Strukturen als
\lstinline!aol::Vec3<int>! angegeben, auch wenn ein Template-Argument
sagt, da� die Struktur zweidimensional benutzt wird. Dies soll
umgestellt werden auf \lstinline!aol::GridSize!. Diese Klasse
kann auch jetzt schon f�r das Setzen und Auslesen von Gittergr��en
verwendet werden. Zuweisung mit rechten Seite \lstinline!aol::Vec2<int>!
und \lstinline!aol::Vec3<int>! sind m�glich, genauso existieren
implizite Umwandlungsoperatoren in diese Klassen.

%%%%%%%%%%%%%%%%%%%%%%%%%%%%%%%%%%%%%%%%%%%%%%%%%%%%%%%%%%%%%%%%%%%%%%%%%%%%%%%

\subsection{Zusicherung f�r alle Gitter}
% Autor von Deylen, Oct 2008

\lstset{numbers=none, aboveskip=0.5\medskipamount}
F�r verschiedene Anwendungsbereiche gibt es verschiedene Gitter, unter
anderem:
\begin{lstlisting}
qc::GridDefinition
qc::RectangularGrid
qc::simplex::GridStructure
nb::NarrowBandGrid
\end{lstlisting}

Allgemeine Klasse wie FE-Operatoren oder Abstiegsverfahren sollten keine
Annahmen �ber die interne Struktur des �bergebenen Gitters treffen.
Wenn sie sich an die folgende Schnittstelle halten, ist die Verwendbarkeit
in allen Gittern gesichert.

Jedes Gitter implementiert das folgende Minimalger�st:
\begin{lstlisting}[numbers=left, stepnumber=5, aboveskip=\medskipamount]
template <...>
class Grid {
public:
  typedef ... ElementType;
  typedef ... FullElementIterator;
  typedef ... FullNodeIterator;
  typedef ... BeginIterType;
  typedef ... EndIterType;

  Grid ( const GridSize & );

  qc::Dimension getDimOfWorld() const;
  int getElementIndex ( const ElementType & ) const;

  int getNumX() const;
  int getNumY() const;
  int getNumZ() const;
  GridSize getSize() const;
  int getNumberOfNodes() const;
  double H() const;

  const BeginIterType & begin() const;
  const EndIterType & end() const;
};
\end{lstlisting}

Die Funktion \lstinline!getNumberOfNodes()! liefert dabei die Anzahl an Datenwerten,
die f�r ein Skalarfeld auf dem Gitter gespeichert werden m�ssen. F�r das
\lstinline!nb::NarrowBandGrid! bedeutet dies die Gr��e des Vollgitters, nicht
die Anzahl benutzter Knoten.

Die Funktion \lstinline!H()! liefert die maximale Entfernung zwischen zwei Gitterknoten
zur�ck.

Ein Iterator vom Typ \lstinline!IterT! mu� folgendes Minimalger�st implementieren:
\begin{lstlisting}
template <...>
class Iterator {
public:
  Iterator ( const BeginIterType & );
  bool operator!= ( const EndIterType & ) const
  ElementType & operator* ();
  ElementType * operator-> ();
};
\end{lstlisting}

%%%%%%%%%%%%%%%%%%%%%%%%%%%%%%%%%%%%%%%%%%%%%%%%%%%%%%%%%%%%%%%%%%%%%%%%%%%%%%%

\subsection{Zusicherung f�r alle Konfiguratoren}
% Autor von Deylen, Oct 2008

Jeder Konfigurator implementiert das Minimalger�st:
\begin{lstlisting}[numbers=left, stepnumber=5, aboveskip=\medskipamount]
template < ... >
class Configurator {
public:
  typedef ... RealType;
  typedef ... InitType;

  typedef ... ElementType;
  typedef ... ElementIteratorType;
  typedef ... BaseFuncSetType;
  typedef ... QuadType;

  typedef ... ArrayType;
  typedef ... VectorType;
  typedef ... MatrixType;
  typedef ... FullMatrixType;
  typedef ... MaskType;

  typedef ... DomVecType;
  typedef ... VecType;
  typedef ... MatType;

  Configurator ( const InitType & );

  int getNumLocalDofs ( const ElementType & ) const;
  int getNumGlobalDofs () const;
  int maxNumQuadPoints() const;

  const InitType & getInitializer() const;
  MatrixType * createNewMatrix() const;

  const BaseFuncSetType & getBaseFuncSet ( const ElementType & ) const;
  int localToGlobal ( const ElementType &, int localIndex ) const;
  RealType vol ( const ElementType & ) const;
  RealType H ( const ElementType & ) const;

  const InitType::BeginIterType & begin() const;
  const InitType::EndIterType & end() const;

  static const int                      maxNumLocalDofs;
  static const qc::Dimension            Dim;
  static const qc::Dimension            DomDim;
  static const aol::GridGlobalIndexMode IndexMode;
};
\end{lstlisting}

Dabei ist \lstinline!DomVecType! der Typ, in dem die Quadraturregel
Referenzkoordinaten f�r die Quadratur angibt, bei W�rfelgittern
also ein \lstinline!Vec<RealType, DomDim>!, bei simplizialen Gittern
(siehe unten) baryzentrische Koordinaten mit einer Komponenten mehr.

%%%%%%%%%%%%%%%%%%%%%%%%%%%%%%%%%%%%%%%%%%%%%%%%%%%%%%%%%%%%%%%%%%%%%%%%%%%%%%%

\subsection{Simpliziale Elemente}
\label{sec:SimplElemente}
% Autor von Deylen, Oct 2008

Die simplizialen Gitter in den QuocMeshes sind kein eigenst�ndiges
Gitterkonstrukt. Sie ben�tigen ein zugrundeliegendes W�rfelgitter,
das durch Unterteilung der W�rfel in Simplicia\footnote
{
	\textit{simplex, simplex, simplex} wird kurz-i-dekliniert.
	Schon ohne dieses Wissen w�re "`Simplices"' mit h�chster Sicherheit
	als falscher Plural des Neutrums zu erkennen. Es ist korrekter Plural
	des Maskulinums, und also gibt es drei grammatisch korrekte Varianten:\\
	das Simplex, die Simplicia\\
	der Simplex, die Simplices\\
	der/das Simplex, die Simplexe.
}
zu einem Simplex-Gitter wird.

In 2D wird jedes Quadrat in zwei Dreiecke geteilt, in 3D jeder W�rfel
in sechs Tetraeder. Die Unterteilung ist identisch zu der im CFE-Modul.

\paragraph{Koordinaten.}
Referenzkoordinaten innerhalb eines Elements sind baryzentrische
Koordinaten. Hierf�r existiert die Klasse \lstinline!BarCoord<int, RealType>!,
die auch �ber den \lstinline!VecDimTrait! zug�nglich ist:
\begin{lstlisting}
template <class _DataType, int dim>
class VecDimTrait {};

template <class _DataType>
class VecDimTrait<_DataType, 2> {
public:
  typedef aol::Vec2<_DataType> VecType;
  typedef aol::BarCoord<2, _DataType> BarCoordType;
};

template <class _DataType>
class VecDimTrait<_DataType, 3> {
public:
  typedef aol::Vec3<_DataType> VecType;
  typedef aol::BarCoord<3, _DataType> BarCoordType;
};
\end{lstlisting}

Hieraus wird deutlich, da� der Template-Parameter \lstinline!dim!
f�r \lstinline!BarCoord! nicht die L�nge des Vectors, sondern die
Dimension des umgebenden Raums angibt. Ein
\lstinline!BarCoord<3, double>! beschreibt Punkte im dreidimensionalen
Standardsimplex und besitzt also vier Komponenten.

Man kann sich streiten, ob das Durchschleusen des \lstinline!DataType!
richtig ist oder vielmehr \lstinline!RealTrait<DataType>::RealType! die
korrekte Wahl w�re, immerhin liegen die baryzentrischen Koordinaten
zwingend zwischen $0$ und $1$, sind mithin sinnvollerweise stets
reellwertig. Ganzzahlige baryzentrische Koordinaten sind genau an
den Ecken m�glich.

\paragraph{Namensraum.}
S�mtliche Klassen liegen im Unter-Namensraum \lstinline!qc::simplex!
im Modul \lstinline!qc!. In den hiesigen Code-Beispielen wird er
zur �bersichtlichkeit fortgelassen. Das entspricht dem sicherlich
sinnvollen \lstinline!using namespace qc::simplex;! am Anfang einer
\texttt{cpp}-Datei im Projektverzeichnis.

\paragraph{Gitter.}
Das Gitter bekommt als Template-Argument den Typ des W�rfel-Gitters.
Es wird auch durch �bergabe solch eines W�rfel-Gitters konstruiert:
\begin{lstlisting}
template <typename CubicGrid>
class GridStructure {
public:
  GridStructure ( CubicGrid & CubicGrid );
  ...
};
\end{lstlisting}

\paragraph{Iteratoren.}
Der Knoteniterator des W�rfelgitters wird durchgeschleust.

Der Elementiterator ist zusammengesetzt aus dem W�rfel-Elementinterator
und einem Simplex-in-W�rfel-Iterator. Er durchl�uft in �u�erer Schleife
alle W�rfelelemente, in innerer alle Simplicia eines W�rfels.

\paragraph{BaseFunctionSet.}
Die Basisfunktionen sind ohne R�ckgriff auf W�rfel-Klassen geschrieben.
Es besteht bisher nur Bedarf nach linearen Basisfunktionen, also sind nur
diese implementiert.

\paragraph{Konfiguratoren.}
Die Simplex-Gitter kennen keinen eigenen Index-Mapper. Ihre Funktion
\lstinline!localToGlobal()! �bersetzt erst lokale Simplex- in lokale
W�rfelindices und benutzt hinterher das \lstinline!localToGlobal()!
des passenden W�rfel-Konfigurators.

\subsubsection{Kompatiblit�t zu W�rfelgittern}

Code, der je nach Pr�prozessor-\lstinline!define! mit simplizialen
oder W�rfelelementen arbeitet, stellt an das W�rfelgitter folgende
Anforderungen (die von der \lstinline!qc::simplex::GridStructure!
erf�llt werden):
\begin{lstlisting}
typedef ... CubicGridType;
const CubicGridType getCubicGrid() const;
\end{lstlisting}

%%%%%%%%%%%%%%%%%%%%%%%%%%%%%%%%%%%%%%%%%%%%%%%%%%%%%%%%%%%%%%%%%%%%%%%%%%%%%%%
%%%%%%%%%%%%%%%%%%%%%%%%%%%%%%%%%%%%%%%%%%%%%%%%%%%%%%%%%%%%%%%%%%%%%%%%%%%%%%%
%%%%%%%%%%%%%%%%%%%%%%%%%%%%%%%%%%%%%%%%%%%%%%%%%%%%%%%%%%%%%%%%%%%%%%%%%%%%%%%

\section{D"unne B"ander}
% Autor von Deylen, Oct 2008

Narrow-Band-Gitter (NB-Gitter) sind in den QuocMeshes definiert als
\begin{enumerate}
\item	ein zugrundeliegendes volles Gitter zusammen mit
\item	einer Liste der verwendeten Elemente.
\end{enumerate}
Ein Knoten wird verwendet, wenn mindestens ein an ihn angrenzendes
Element verwendet wird.

Derzeit wird die Liste der verwendeten Elemente als Hash-Set gespeichert.

Das Vollgitter und die Dimension\footnote
{
	Da die \lstinline!qc::GridDefinition! bisher ihre Dimension
	nur als Variable speichert und nicht dar�ber templatisiert
	ist, mu� die Dimension zus�tzlich gegeben werden.
}
sind Template-Parameter.

\subsection{Erzeugen eines NB-Gitters}
Das NB-Gitter bekommt im Konstruktor ein fertig konstruiertes
Vollgitter-Objekt �bergeben. Folglich mu� im Projekt ein solches
deklariert werden. Zur Verwaltung der Elementliste stehen nur die
Funktionen
\begin{lstlisting}
bool exists ( const ElementType & ) const;
void insert ( const ElementType & );
void clear ();
\end{lstlisting}
zur Verf�gung. Im Projekt kann beispielsweise mit folgendem Code
ein �berall mindestens \lstinline!bandwidth! breites Band um
das Nullniveau der \textit{signed distance function}
\lstinline!distanceArray! gef�llt werden:

\begin{lstlisting}[numbers=left, stepnumber=5, aboveskip=\medskipamount]
template <typename ConfType, typename ArrayType,
          typename ElType, typename RealType>
bool elementCutsBand(const ConfType & configurator,
                     const ArrayType & distanceArray,
                     const ElType & element,
                     RealType bandWidth) {
  int numLocalDofs = configurator.getNumLocalDofs ( element );
  for (int i = 0; i < numLocalDofs; ++i)
    if (fabs(distanceArray[configurator.
                     localToGlobal ( element, i )]) < bandWidth)
      return true;
  return false;
}

template <typename RealType, typename NarrowBandGridType,
          typename ArrayType, typename ConfType>
void fillNarrowBandGrid
  ( const typename NarrowBandGridType::FullGridType & fullGrid,
    const ConfType & configurator,
    const ArrayType & levelFct,
    RealType bandWidth,
    NarrowBandGridType & narrowGrid ) {
  typedef typename NarrowBandGridType::FullGridType::
                                  FullElementIterator Iterator;
  for (Iterator iter = fullGrid.begin(); iter != fullGrid.end(); ++iter)
    if (elementCutsBand(configurator, levelFct, *iter, bandWidth))
      narrowGrid.insert(*iter);
}
\end{lstlisting}

\subsection{Konfiguratoren}

Bisher gibt es noch eigene NB-Konfiguratoren. Diese sollten nicht mehr
verwendet werden. Stattdessen kann eine "`H�lle"' um Vollgitter-Konfiguratoren
gest�lpt werden:
\begin{lstlisting}[numbers=left, aboveskip=\medskipamount]
template <typename FullGridConfType, typename NBGridType>
class FullGridConfiguratorHull {
public:
  FullGridConfiguratorHull ( const NBGridType & );
  void writeNodeExistMaskTo ( MaskType & ) const;
};
\end{lstlisting}
sowie mit allen Typen und Funktionen der allgemeinen Zusicherung.

\subsection{Kompatiblit�t zum Vollgitter}

Code, der je nach Pr�prozessor-\lstinline!define! mit vollem
oder NB-Gitter arbeitet, stellt an das Vollgitter folgende
Anforderungen (die vom \lstinline!nb::NarrowBandGrid!
erf�llt werden):
\begin{lstlisting}
typedef ... FullGridType;
const FullGridType getFullGrid() const;
\end{lstlisting}

\subsection{Weiteres}

\paragraph{Zus�tzliche Iteratoren.}
�ber die allgemeinen Zusicherungen hinaus gibt es die Funktionen:
\begin{lstlisting}
eiterator ebegin () const;
eiterator eend () const;
biterator bbegin () const;
biterator bend () const;
\end{lstlisting}

\paragraph{Simpliziale NB-Gitter.}
Das Vollgitter-Template-Argument darf auch ein simpliziales
Gitter sein.

\lstset{numbers=left, aboveskip=\medskipamount}

%%%%%%%%%%%%%%%%%%%%%%%%%%%%%%%%%%%%%%%%%%%%%%%%%%%%%%%%%%%%%%%%%%%%%%%%%%%%%%%
%%%%%%%%%%%%%%%%%%%%%%%%%%%%%%%%%%%%%%%%%%%%%%%%%%%%%%%%%%%%%%%%%%%%%%%%%%%%%%%
%%%%%%%%%%%%%%%%%%%%%%%%%%%%%%%%%%%%%%%%%%%%%%%%%%%%%%%%%%%%%%%%%%%%%%%%%%%%%%%

\section{SurfMeshes}

%%%%%%%%%%%%%%%%%%%%%%%%%%%%%%%%%%%%%%%%%%%%%%%%%%%%%%%%%%%%%%%%%%%%%%%%%%%%%%%
%%%%%%%%%%%%%%%%%%%%%%%%%%%%%%%%%%%%%%%%%%%%%%%%%%%%%%%%%%%%%%%%%%%%%%%%%%%%%%%
%%%%%%%%%%%%%%%%%%%%%%%%%%%%%%%%%%%%%%%%%%%%%%%%%%%%%%%%%%%%%%%%%%%%%%%%%%%%%%%

\section{DT-Grids}

\chapter{Nichtmathematische Inhalte der Bibliothek}

\section{Tools und Utilities}

\anmerkung{Martins Parameter-Parner (alte Anleitung Seite 69) nicht
vergessen}

\subsection{TimestepSaver}

\anmerkung{alte Anleitung Seite 78}

\section{Externals und Visualisierer}



\end{document}
