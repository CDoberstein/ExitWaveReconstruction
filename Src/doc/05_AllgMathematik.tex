\chapter{Allgemeine Mathematik}

\anmerkung{Default-Einstellung f"ur alle Container ist deep copy! Ausnahmen nennen:
DiscreteFunction, auto\_container auf Wunsch.}\\
\anmerkung{Alle folgenden Punkte im Schema: Welche gibt es, Unterschiede, wof"ur jeweils
gedacht/zu benutzen, wie zu finden/einzubinden}

%%%%%%%%%%%%%%%%%%%%%%%%%%%%%%%%%%%%%%%%%%%%%%%%%%%%%%%%%%%%%%%%%%%%%%%%%%%%%%%
%%%%%%%%%%%%%%%%%%%%%%%%%%%%%%%%%%%%%%%%%%%%%%%%%%%%%%%%%%%%%%%%%%%%%%%%%%%%%%%
%%%%%%%%%%%%%%%%%%%%%%%%%%%%%%%%%%%%%%%%%%%%%%%%%%%%%%%%%%%%%%%%%%%%%%%%%%%%%%%

\section{Vektoren}

\paragraph{Multivektoren.}

\paragraph{Speichermanagement "uber den Vektor-Manager.}

%%%%%%%%%%%%%%%%%%%%%%%%%%%%%%%%%%%%%%%%%%%%%%%%%%%%%%%%%%%%%%%%%%%%%%%%%%%%%%%
%%%%%%%%%%%%%%%%%%%%%%%%%%%%%%%%%%%%%%%%%%%%%%%%%%%%%%%%%%%%%%%%%%%%%%%%%%%%%%%
%%%%%%%%%%%%%%%%%%%%%%%%%%%%%%%%%%%%%%%%%%%%%%%%%%%%%%%%%%%%%%%%%%%%%%%%%%%%%%%

\section{Operatoren}

\paragraph{Der BlockOp.}

%%%%%%%%%%%%%%%%%%%%%%%%%%%%%%%%%%%%%%%%%%%%%%%%%%%%%%%%%%%%%%%%%%%%%%%%%%%%%%%
%%%%%%%%%%%%%%%%%%%%%%%%%%%%%%%%%%%%%%%%%%%%%%%%%%%%%%%%%%%%%%%%%%%%%%%%%%%%%%%
%%%%%%%%%%%%%%%%%%%%%%%%%%%%%%%%%%%%%%%%%%%%%%%%%%%%%%%%%%%%%%%%%%%%%%%%%%%%%%%

\section{Matrizen}

\newcommand \clsallgdiag [3]
{%
  \framebox(#1,#2)
  {%
    \begin{minipage}{#1\unitlength}
    \begin{center}
    \footnotesize
    #3
    \end{center}
    \end{minipage}
  }
}
\newcommand \clsdiag[2][0.75] {\clsallgdiag{2.5}{#1}{#2}}

\newcommand \abstrakt {\itshape}

\setlength{\unitlength}{10mm}
\begin{picture}(15,8)
\put( 7   , 7   ){\clsdiag{\abstrakt Op}}
\put( 8.25, 6.5 ){\vector(0,1){0.5}}
\put( 3.75, 5.75){\line(0,1){0.75}}
\put( 6.25, 5.75){\line(0,1){0.75}}
\put( 9.75, 5.75){\line(0,1){0.75}}
\put(12.75, 5.75){\line(0,1){0.75}}
\put( 3.75, 6.5 ){\line(1,0){9}}
%
\put( 2.5 , 5   ){\clsdiag{\abstrakt Matrix}}
\put( 1.25, 2.75){\line(0,1){0.75}}
\put( 5.75, 2.75){\line(0,1){0.75}}
\put( 1.25, 3.5 ){\line(1,0){4.5}}
\put( 3.75, 3.5){\vector(0,1){1.5}}
%
\put( 0   , 2   ){\clsdiag{\abstrakt GenBandMatrix}}
\put( 1.25, 0.75){\vector(0,1){1.25}}
\put(-0.5 ,-0.75){\clsallgdiag{3}{1.5}{S9\_2D\_BandMatrix\\S27\_3D\_BandMatrix}}
\put( 4.5 , 2   ){\clsdiag{\abstrakt GenSparseMatrix}}
\put( 4.25, 0.75){\line(0,1){0.75}}
\put( 7.25, 0.75){\line(0,1){0.75}}
\put( 4.25, 1.5 ){\line(1,0){3}}
\put( 5.75, 1.5){\vector(0,1){0.5}}
\put( 3   , 0   ){\clsdiag{SparseMatrix}}
\put( 6   ,-0.75){\clsdiag[1.5]{UniformGrid\\SparseMatrix}}
%
%
\put( 5.5 , 4.25){\clsdiag[1.5]{MyFastUniform\\GridMatrix}}
\put( 8.5 , 5   ){\clsdiag{UGBMatrix}}
%
%
\put(11.5 , 5   ){\clsdiag{\abstrakt GenSparseOp}}
\put(11.25, 2.75){\line(0,1){0.75}}
\put(14.25, 2.75){\line(0,1){0.75}}
\put(11.25, 3.5 ){\line(1,0){3}}
\put(12.75, 3.5){\vector(0,1){1.5}}
%
\put(10   , 1.25){\clsdiag[1.5]{FastUniform\\GridMatrix}}
\put(13   , 1.25){\clsdiag[1.5]{UniGrid\\CSR\_Matrix}}
%
\end{picture}
\vspace{2cm}

\subsection{General explanations}
\paragraph{Why are not all matrix classes derived from \texttt{Matrix}?}
aol::Matrix is an abstract basis class for matrices that provides
virtual methods get, set and add. These methods are mainly used for
assembling matrices, so their efficiency may be critical.

For faster assembly, some ``matrices'' are not derived from
aol::Matrix, thus lacking some methods.

\paragraph{Which interfaces are defined by \texttt{Matrix},
\texttt{GenBandMatrix}, \texttt{GenSparseMatrix} and \texttt{GenSparseOp}? How do they differ?}
aol::GenBandMatrix is an abstract basis class for band matrices. Only
entries in bands parallel (but not necessarily adjacent) to the main
diagonal are stored, the storage is organized row-wise for
cache-efficiency in matrix-vector multiplication.

aol::GenSparseMatrix is an abstract basis class for sparse matrices
that are organized in rows. They contain a vector of pointers to
aol::Rows which may be filled with different row objects.

\subsection{The matrix classes in detail}
Note that remarks about the speed of matrices may significantly depend
on the platform, the compiler and its settings, so you should run the
sparsebench benchmark to see which matrix works best in your
environment.

\subsubsection{aol::FullMatrix}
%%%%%%%%%%%%%%%%%%%%%%%%%%%%%%%
\paragraph{For which applications is this matrix class designed?}
Dense matrices.

\paragraph{Advantages}
``Fast'' access.

\paragraph{Disadvantages}
Storage of all entries, even if they are zero.

\paragraph{Internal structure and how it works}
Two-dimensional array of all entries.


\subsubsection{aol::SparseMatrix}
%%%%%%%%%%%%%%%%%%%%%%%%%%%%%%%%%
\paragraph{For which applications is this matrix class designed?}
Unstructured sparse matrix.

\paragraph{Advantages}
Most flexible class for sparse matrices.

\paragraph{Disadvantages}
Least efficient because no sparsity structure is known implicitely.

\paragraph{Internal structure and how it works}
Each row stores a (sorted) stl-vector of aol::RowEntries containing a
pair (column index, value).



\subsubsection{qc::S9\_2D\_BandMatrix, qc::S27\_3D\_BandMatrix}
%%%%%%%%%%%%%%%%%%%%%%%%%%%%%%%%%%%%%%%%%%%%%%%%%%%%%%%%%%%%%%%
\paragraph{For which applications is this matrix class designed?}
For applications on an uniform rectangular (2D) or an uniform hexagonal (3D) mesh,
which result in 9 (2D) or 27 (3D) entries per row.

\paragraph{Advantages}
Fast access and matrix-vector multiplication, fast row-wise access.

\paragraph{Disadvantages}
Small memory overhead (two integers per row).

\paragraph{Internal structure and how it works}
Only bands containing nonzero entries are stored, entries are stored
row-wise. Lookup tables for the offsets are created when creating
matrix.

\subsubsection{aol::TriBandMatrix, aol::LQuadBandMatrix}
%%%%%%%%%%%%%%%%%%%%%%%%%%%%%%%%%%%%%%%%%%%%%%%%%%%%%%%%
\paragraph{For which applications is this matrix class designed?}
Tridiagonal matrices and Quadridiagonal matrices (two lower, one upper
diagonal) adjacent to main diagonal.


\subsubsection{qc::UniformGridSparseMatrix}
%%%%%%%%%%%%%%%%%%%%%%%%%%%%%%%%%%%%%%%%%%%
\paragraph{For which applications is this matrix class designed?}
For applications on an uniform rectangular (2D) or an uniform hexagonal (3D) mesh,
which result in 9 (2D) or 27 (3D) entries per row.

\paragraph{Advantages}
The \texttt{applyAdd}-method is about 2 times (2D) or
1.5 times (3D) faster than the simple SparseMatrix.

\paragraph{Disadvantages}
The \texttt{assemble}-method is not faster, but even slightly slower.

\paragraph{Internal structure and how it works}
This matrix is derived from \texttt{GenSparseMatrix}, the main
difference is: The rows that belong to inner nodes are special
\texttt{UniformGridSparseRows} (implemented in rows.h) which are just
arrays of the specified DataType with 9 (2D) or 27 (3D) entries.  The
dimension is read from the grid and set automatically. The methods
\texttt{get} and \texttt{set} allow for the uniform structure of the
grid and compute the belonging array-index to the desired
matrix-entry.

Furtheron the \texttt{realloc}-method makes use of the uniform grid
structure.


\subsubsection{qc::FastUniformGridMatrix}
%%%%%%%%%%%%%%%%%%%%%%%%%%%%%%%%%%%%%%%%%
\paragraph{For which applications is this matrix class designed?}
For applications on an uniform rectangular (2D) or an uniform
hexagonal (3D) mesh, which result in 9 (2D) or 27 (3D) entries per
row.

\paragraph{Advantages}
Since this matrix is not derived from aol::Matrix, it doesn't use
the virtual get and set methods, but its own faster ones.
The \texttt{assemble}-method is much faster than the one from the
UniformGridSparseMatrix (about 2.5 times in 2D and 3D), the
\texttt{apply}-method is at least a bit faster (about 1.4 times in 2D
and 3D).

\paragraph{Disadvantages}
Not derived from aol::Matrix

\paragraph{Internal structure and how it works}
The \texttt{FastUniformGridMatrix} is derived from the
\texttt{GenSparseOp} and \textit{not} from the \texttt{GenSparseMatrix}.
This makes it possible that it has its own once more optimized
methods.  The rows are again stored as simple DataType-arrays, but the
access is organised with some kind of hierarchical principle: The rows
are divided into three blocks, which consist of three diagonals (in
2D) or once again of three blocks which then are divided into three
diagonals (in 3D). To access the value of an index $(i,j)$ the
belonging block and the index in this block are computed.

Furtheron the \texttt{applyAdd}-method also doesn't need to use the
\texttt{mult}-method from the \texttt{rows}, as it is the case for all
matrices that are derived from the \texttt{GenSparseMatrix}. Instead
it is implemented in a hierarchical way which uses the block structure
described above.

\subsubsection{qc::UGBMatrix}
%%%%%%%%%%%%%%%%%%%%%%%%%%%%%
\paragraph{For which applications is this matrix class designed?}
For applications which produce a band matrix and work on the grids of
types RectangularGrid or GridDefinition.

\paragraph{Advantages}
Very cache-efficient matrix-vector multiplication.

\paragraph{Disadvantages}
Not an aol::Matrix.

\paragraph{Internal structure and how it works}
The method used here is a sparse banded block multiplication scheme
which takes advantage of data localization in the cache of the
processor. With use of the intel compiler icc/ecc and the pragma
directives pragma ivdep and efficient vectorization of the code can be
achieved by setting the define VECTORIZE\_INTEL during
compilation. \\
The template parameter blocksize defines the size of the blocks which
are used in the multiplication method. blocksize should be chosen such
that sizeof(DataType)*columns*blocksize is less than the size of the
largest cache of the processor.


\subsubsection{aol::CSR\_Matrix}
%%%%%%%%%%%%%%%%%%%%%%%%%%%%%%%
\paragraph{For which applications is this matrix class designed?}
Unstructured sparse matrices.

\paragraph{Advantages}
Efficient matrix-vector multiplication.

\paragraph{Disadvantages}
Not an aol::Matrix. No (or very slow) random access. Only works for
double precision data type.

\paragraph{Internal structure and how it works}
A matrix stored in compressed sparse row storage format, a standard
format (see MKL).

\subsubsection{qc::UniGridCSR\_Matrix}
%%%%%%%%%%%%%%%%%%%%%%%%%%%%%%%%%%%%%
\paragraph{For which applications is this matrix class designed?}
For applications on an uniform rectangular (2D) or an uniform
hexagonal (3D) mesh, which result in 9 (2D) or 27 (3D) entries per
row.

\subsubsection{MyFastUniformGridMatrix}
%%%%%%%%%%%%%%%%%%%%%%%%%%%%%%%%%%%%%%%
This matrix class is obsolete and will be (or has been) merged with
FastUniformGridMatrix.

%%%%%%%%%%%%%%%%%%%%%%%%%%%%%%%%%%%%%%%%%%%%%%%%%%%%%%%%%%%%%%%%%%%%%%%%%%%%%%%
%%%%%%%%%%%%%%%%%%%%%%%%%%%%%%%%%%%%%%%%%%%%%%%%%%%%%%%%%%%%%%%%%%%%%%%%%%%%%%%
%%%%%%%%%%%%%%%%%%%%%%%%%%%%%%%%%%%%%%%%%%%%%%%%%%%%%%%%%%%%%%%%%%%%%%%%%%%%%%%

\section{L"oser f"ur lineare Gleichungssysteme}

\anmerkung{bish. Anleitung verbessern, Liste h"ubscher, aber beibehalten}
\anmerkung{Referenz f"ur Algorithmen angeben. Wenn Notation selbst erdacht:
Algorithmus hinschreiben.\\
Voraussetzungen an Matrix klarer und deutlicher\\
Benutzung der L"oser vor Auflistung}

\anmerkung{Was ist mit Martins QR-Zerlegung, alte Anl. Seite 69?}

\section{Vorkonditionierer}

\anmerkung{wie L"oser: bisherige Liste behalten, nur aufh"ubschen}

%%%%%%%%%%%%%%%%%%%%%%%%%%%%%%%%%%%%%%%%%%%%%%%%%%%%%%%%%%%%%%%%%%%%%%%%%%%%%%%
%%%%%%%%%%%%%%%%%%%%%%%%%%%%%%%%%%%%%%%%%%%%%%%%%%%%%%%%%%%%%%%%%%%%%%%%%%%%%%%
%%%%%%%%%%%%%%%%%%%%%%%%%%%%%%%%%%%%%%%%%%%%%%%%%%%%%%%%%%%%%%%%%%%%%%%%%%%%%%%

\section{Newton-Verfahren}

\anmerkung{neu verfassen f"ur Benjamins Newton-Verf.\\
Dabei Klassendiagramm wg. un"ubersichtlicher Vererbung\\
Ablaufdiagramm aus Stefans Kurzdoku Pkt. 1.5}

%%%%%%%%%%%%%%%%%%%%%%%%%%%%%%%%%%%%%%%%%%%%%%%%%%%%%%%%%%%%%%%%%%%%%%%%%%%%%%%
%%%%%%%%%%%%%%%%%%%%%%%%%%%%%%%%%%%%%%%%%%%%%%%%%%%%%%%%%%%%%%%%%%%%%%%%%%%%%%%
%%%%%%%%%%%%%%%%%%%%%%%%%%%%%%%%%%%%%%%%%%%%%%%%%%%%%%%%%%%%%%%%%%%%%%%%%%%%%%%

\section{Gradientenabstiegsverfahren}
\anmerkung{ebenfalls Klassendiagramm. GradientFlow-Erkl"arung,
alte Anl. Seite 81, wiederverwenden}

%%%%%%%%%%%%%%%%%%%%%%%%%%%%%%%%%%%%%%%%%%%%%%%%%%%%%%%%%%%%%%%%%%%%%%%%%%%%%%%
%%%%%%%%%%%%%%%%%%%%%%%%%%%%%%%%%%%%%%%%%%%%%%%%%%%%%%%%%%%%%%%%%%%%%%%%%%%%%%%
%%%%%%%%%%%%%%%%%%%%%%%%%%%%%%%%%%%%%%%%%%%%%%%%%%%%%%%%%%%%%%%%%%%%%%%%%%%%%%%

\section{Gew"ohnliche DGl}

\anmerkung{Motivation (Ortsdiskret. "uber FE, Zeit "uber fin. Diff. f"uhrt zu
gro"sen DGl-Systemen)}

\paragraph{Die Klasse Timestep}
\anmerkung{alte Anl. Seite 75, aber allgemeiner. Parabolische pDGl nur als
Beispiel.}
