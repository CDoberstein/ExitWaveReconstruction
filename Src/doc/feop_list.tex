Here, an overview over the finite element interface classes is given.
We will distinguish between linear operator interfaces (which additionally
to "apply(...)" and "applyAdd(...)" have a method "assembleAddMatrix(...)" to assemble a system matrix)
and nonlinear operators (which cannot assemble a matrix).
The first table enlists the linear operator interfaces, where the second column displays the matrix assembled in "assembleAddMatrix(...)"
(the result of "apply(...)" then is just this matrix multiplied by the vector of function values, passed to "apply(...)" as the first argument),
the left column gives the corresponding class names, and the right column enlists the methods to be overloaded.
The second table shows all nonlinear operator interfaces and has the same structure, only the middle column shows the result of the method "apply(...)".
In both cases, $\phi$ shall represent the discretized function, which is passed to "apply(...)" or "applyAdd(...)" as the first argument,
$\varphi_i$ represents the $i$th finite element base function, $w$, $f$, $A$ are functions, which have to be implemented by the overload method
in the derived class. An arrow over a function signifies that the function is vector-valued, two arrows
symbolize a matrix. The implementation of $f\left(\phi(x),\nabla\phi(x),x\right)$ gets $x$ and the function $\phi$ as argument, where $\phi$
itself may be evaluated at $x$, or its gradient, or both. $n_g$ shall denote the number of components of a vector function $\vec g$.

{
\renewcommand{\arraystretch}{2}
\resizebox{\textwidth}{!}{
\begin{tabular}{llll}
  \hline
    \textbf{FELin...Interface} & \textbf{expression} & \textbf{overload method} & \textbf{comment} \\
  \hline
    ScalarWeightedMass      & $\left(\int_\Omega w(x) \varphi_j(x) \varphi_i(x) dx\right)_{ij}$                                                            & $w$: getCoeff() \\
    ScalarWeightedStiff     & $\left(\int_\Omega w(x) \nabla\varphi_j(x) \cdot \nabla\varphi_i(x)dx \right)_{ij}$                                          & $w$: getCoeff() \\
    MatrixWeightedStiff     & $\left(\int_\Omega A(x)\nabla \varphi_j \cdot \nabla \varphi_i dx\right)_{ij}$                                               & $A$: getCoeffMatrix() & $A$ symmetric \\
    AsymMatrixWeightedStiff & $\left(\int_\Omega A(x)\nabla \varphi_j \cdot \nabla \varphi_i dx\right)_{ij}$                                               & $A$: getCoeffMatrix() & $A$ asymmetric \\
    ScalarWeightedMixedDiff & $\left(\int_\Omega w(x) \frac{\partial\varphi_j(x)}{\partial x_s} \frac{\partial\varphi_i(x)}{\partial x_t} dx\right)_{ij}$  & $w$: getCoeff() \\
    ScalarWeightedSemiDiff  & $\left(\int_\Omega \varphi_j(x) \frac{\partial\varphi_i(x)}{\partial x_k} w(x)\,dx\right)_{ij} $                             & $w$: getCoeff() & can also return the transpose \\
    VectorWeightedSemiDiff  & $\left(\int_\Omega \varphi_j(x) (\nabla \varphi_i(x) \cdot \vec{w}(x))\,dx\right)_{ij} $                                     & $\vec{w}$: getCoefficientVector() & can also return the transpose, \\
                            &                                                                                                                              & & $n_w=$domain dimension \\
  \hline
\end{tabular}
}
}

\vspace*{1cm}
{
\renewcommand{\arraystretch}{2}
\resizebox{\textwidth}{!}{
\begin{tabular}{llll}
  \hline
    \textbf{FENonlin...Interface} & \textbf{expression} & \textbf{overload method} & \textbf{comment} \\
  \hline
    Op                       & $\left(\int_\Omega f\left(\phi(x),\nabla\phi(x),x\right) \varphi_i(x) dx\right)_i $                                           & $f$: getNonlinearity() \\
    DiffOp                   & $\left(\int_\Omega \vec{f}\left(\phi(x),\nabla\phi(x),x\right)\cdot \nabla\varphi_i(x) dx\right)_i $                          & $\vec{f}$: getNonlinearity() \\
    VectorOp                 & $\left(\int_\Omega \vec{f}\left(\vec\phi(x),\nabla\vec\phi(x),x\right)\cdot\vec\varphi_i(x) dx\right)_i $                     & $\vec{f}$: getNonlinearity() & $n_f$, $n_\phi$ are template parameters \\
    VectorDiffOp             & $\left(\!\!\int_\Omega \!\!\vec{\vec{f}}\left(\vec\phi(x),\nabla\vec\phi(x),x\right):\nabla \vec\varphi_i(x) dx\!\!\right)_i$ & $\vec{\vec{f}}$:getNonlinearity() \\
    IntegrationScalar        & $\int_\Omega f\left(\phi(x),\nabla\phi(x),x\right) dx$                                                                        & $f$: evaluateIntegrand() \\
    IntegrationVector        & $\int_\Omega f\left(\vec\phi(x),\nabla\vec\phi(x),x\right) dx$                                                                & $f$: evaluateIntegrand() & $n_\phi$ is template parameter \\
    IntegrationVectorGeneral & $\int_\Omega f\left(\vec\phi(x),\nabla\vec\phi(x),x\right) dx$                                                                & $f$: evaluateIntegrand() & $n_\phi\leq$domain dimension is variable \\
  \hline
\end{tabular}
}
}
VectorFENonlinIntegrationVectorInterface can be used for vector valued integrands.
